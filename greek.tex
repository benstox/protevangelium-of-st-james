\pstart
\eledsection*{Αʹ}

\pend\setcounter{pstartL}{1}\pstart
Ἐν ταῖς ἱστορίαις τῶν δώδεκα φυλῶν τοῦ Ἰσραὴλ ἦν Ἰωακεὶμ πλούσιος σφόδρα, καὶ προσέφερε κυρίῳ τὰ δῶρα αὐτοῦ διπλᾶ λέγων ἐν ἑαυτῷ: Ἔσται τὸ τῆς περισσείας μου ἅπαντι τῷ λαῷ καὶ τὸ τῆς ἀφέσεως κυρίῳ τῷ θεῷ εἰς ἱλασμὸν ἐμοί.\footnote{Originally transcribed by Rolf Mainz in Bremen on an Atari computer (!). Transferred to PC format by Wieland Willker. Ported to UTF-8 by Alan Humm. The text is basically a mixture of codex C (Paris 1454, 10th CE, considered by Tischendorf as the best) and the text of Papyrus Bodmer 5, which was unknown to Tischendorf. It is not really a critical text, but it is nevertheless a good text and can be used for classroom purposes. Editions used: E. Strycker, \textit{La forme la plus ancienne du Protevangile de Jaques}, Brussels, 1961; H. R. Smid \textit{Protevangelium Jacobi}, Groningen, 1965. Retrieved from \textit{Jewish and Christian Literature}, \texttt{http://jewishchristianlit.com/Texts/NT/pgJasGrkUTF8.html} (accessed 21 February, 2020).}

\pend\pstart
ἐνἤγγισεν δὲ ἡ ἡμέρα κυρίου ἡ μεγάλη καὶ προσέφερον οἱ υἱοὶ Ἰσραὴλ τὰ δῶρα αὐτῶν, καὶ ἔστη κατενώπιον αὐτοῦ καὶ Ῥουβὴλ λέγων: οὐκ ἔξεστίν σοι πρώτῳ προσενεγκεῖν τὰ δῶρά σου, καθότι σπέρμα οὐκ ἐποίησας ἐν τῷ Ἰσραήλ.

\pend\pstart
καὶ ἐλυπήθη Ἰωακεὶμ καὶ ἀπίει εἰς τὸν οἶκον αὐτοῦ, καὶ ἐλθὼν εἰς τὴν δωδεκάφυλον τοῦ λαοῦ λέγει: ὄψομαι, εἰ ἐγὼ μόνος οὐκ ἐποίησα σπέρμα ἐν τῷ Ἰσραήλ. ἠρεύνησε δὲ καὶ εὗρε πάντας τοὺς δικαίους, ὅτι σπέρμα ἀνέστησαν ἐν τῷ Ἰσραὴλ, καὶ ἐμνήσθη τοῦ πατριάρχου Ἀβραάμ, ὅτι ἐν ταῖς ἐσχάταις αὐτοῦ ἡμέραις ἔδωκεν αὐτῷ ὁ θεὸς υἱὸν Ἰσαάκ.

\pend\pstart
καὶ ἐλυπεῖτο Ἰωακεὶμ σφόδρα καὶ οὐκ ἐφάνη τῇ γυναικὶ αὐτοῦ, ἀλλὰ ἔδωκεν ἑαυτὸν εἰς τὴν ἔρημον, καὶ ἔπηξε τὴν σκηνὴν αὐτοῦ ἐκεῖ καὶ ἐνήστευσεν ἡμέρας τεσσεράκοντα καὶ νύκτας τεσσεράκοντα λέγων ἑν ἑαυτῷ: οὐ καταβήσομαι οὔτε ἐπὶ βρωτὸν οὔτε ἐπὶ ποτόν, ἕως ἐπισκέψηταί με κύριος ὁ θεός μου, καὶ ἔσται μοι ἡ εὐχὴ βρόματα καὶ πόματα.

\pend\pstart
\eledsection*{Βʹ}

\pend\setcounter{pstartL}{1}\pstart
Ἡ δὲ γυνὴ δὲ αὐτοῦ Ἄννα δύο θρήνους ἐθρήνει καὶ δύο κοπετοὺς ἐκόπτετο λέγουσα: κόψομαι τὴν χηρίαν μου καὶ κόψομαι τὴν ἀτεκνίαν μου.

\pend\pstart
ἤγγισε δὲ ἡ ἡμέρα κυρίου ἡ μεγάλη καὶ εἶπεν Ἰουδὴθ ἡ παιδίσκη αὐτῆς πρὸς αὐτήν: ἕως πότε ταπεινοῖς τὴν ψυχήν σου; ἰδοὺ γὰρ ἤγγισεν ἡ ἡμέρα κυρίου ἡ μεγάλη καὶ οὐκ ἔξεστί σοι πενθεῖν. ἀλλὰ λάβε τοῦτο τὸ κεφαλοδέσμιον, ὅ ἔδωκέν μοι ἡ κυρία τοῦ ἔργου, καὶ οὐκ ἔξεστί μοι ἀναδήσασθαι αὐτό, καθότι παιδίσκη σού εἰμι καὶ χαρακτῆρα ἔχει βασιλικόν.

\pend\pstart
καὶ εἶπεν Ἄννα: ἀπόστηθι ἀπ' ἐμοῦ: καὶ ταῦτα οὐκ ἐποίησα, καὶ κύριος ὁ θεὸς ἐταπείνωσέν με σφόδρα. μήπως πανοῦργος ἔδωκέν σοι τοῦτο καὶ ἦλθες κοινωνῆσαί με τῇ ἁμαρτίᾳ σου; εἶπεν δὲ αὐτῇ Ἰουδὴθ ἡ παιδίσκη αὐτῆς: τί ἀράσωμαί σοι, καθότι οὐκ ἤκουσας τῆς φωνῆς μου; ἀπέκλεισεν κύριος ὁ θεὸς τὴν μήτραν σου τοῦ μὴ δοῦναί σοι καρπὸν ἐν τῷ Ἰσραήλ.

\pend\pstart
καὶ ἐλυπήθη Ἄννα σφόδρα καὶ περιείλετο τὰ ἱμάτια αὐτῆς τὰ πενθικὰ καὶ ἐσμήξατο τὴν κεφαλὴν αὐτῆς καὶ ἐνεδύσατο τὰ ἱμάτια αὐτῆς τὰ νυμφικὰ καὶ περὶ ὥραν ἐννάτην κατέβη εἰς τὸν παράδεισον αὐτῆς (τοῦ περιπατῆσαι). καὶ εἶδεν δάφνην καὶ ἐκάθισεν ὑποκάτω αὐτῆς καὶ ἐλιτάνευσε τῷ δεσπότῃ λέγουσα: ὁ θεὸς τῶν πατέρων μου, εὐλόγησόν με καὶ ἐπάκουσον τῆς δεήσεός μου, καθὼς ἐπήκουσας καὶ εὐλόγησας τὴν μητέραν Σάραν καὶ ἔδωκας αὐτῇ υἱὸν τὸν Ἰσαάκ.

\pend\pstart
\eledsection*{Γʹ}

\pend\setcounter{pstartL}{1}\pstart
Καὶ ἀτενίσασα Ἄννα εἰς οὐρανὸν εἶδεν καλιὰν στρουθίων ἐν τῇ δάφνῃ καὶ εὐθέως ἐποίησε θρῆνον ἐν ἑαυτῇ λέγουσα: οἴμοι, τίς με ἐγέννησεν, ποία δὲ μήτρα ἐξέφυσέν με, ὅτι κατάρα ἐγεννήθην ἐνώπιον τῶν υἱῶν Ἰσραήλ καὶ ὠνειδίσθην καὶ ἐξεμυκτηρίσθην ἐκβληθεῖσα ἐκ ναοῦ κυρίου τοῦ θεοῦ μου;

\pend\pstart
οἴμοι, τίνι ὁμοιώθην ἐγώ; οὐχ ὁμοιώθην ἐγὼ τοῖς πετεινοῖς τοῦ οὐρανοῦ, ὅτι καὶ τὰ πετεινὰ γόνιμά εἰσιν ἐνώπιόν σου, κύριε. οἴμοι, τίνι ὁμοιώθην ἐγώ; οὐχ ὁμοιώθην ἐγὼ τοῖς ἀλόγοις ζώοις, καὶ τὰ ἄλογα ζῶα γόνιμά εἰσιν ἐνώπιόν σου, κύριε.

\pend\pstart
οἴμοι, τίνι ὁμοιώθην ἐγώ; οὐχ ὁμοιώθην ἐγὼ τοῖς ὕδασι τούτοις, ὅτι καὶ τὰ ὕδατα γόνιμά εἰσιν ἐνώπιόν σου, κύριε. οἴμοι, τίνι ὁμοιώθην ἐγώ; οὐχ ὁμοιώθην ἐγὼ τῇ γῇ, ὅτι καὶ ἡ γῆ προφέρει τοὺς καρποὺς αὐτῆς κατὰ καιρὸν καί σε εὐλογεῖ, κύριε.

\pend\pstart
\eledsection*{Δʹ}

\pend\setcounter{pstartL}{1}\pstart
Καὶ ἰδοὺ ἄγγελος κυρίου ἐπέστη λέγων: Ἄννα, Ἄννα, εἰσήκουσε κύριος ὁ θεὸς τῆς δεήσεός σου, καὶ λήψῃ καὶ λαληθήσεται τὸ σπέρμα σου ἐν ὅλῃ τῇ οἰκουμένῃ. εἶπεν δὲ Ἄννα: ζῇ κύριος ὁ θεός μου: ἐὰν γεννήσω εἴτε ἄρρεν εἴτε θῆλυ, προσάξω αὐτὸ δῶρον κυρίῳ τῷ θεῷ μου καὶ ἔσται λειτουργοῦν αὐτῷ πάσας ἡμέρας τῆς ζωῆς αὐτοῦ.

\pend\pstart
καὶ ἰδοὺ ἤλθοσαν ἄγγελοι δύοι λέγοντες αὐτῇ: ἰδοὺ Ἰωακεὶμ ὁ ἀνήρ σου ἔρχεται μετὰ τῶν ποιμνίων αὐτοῦ. ἄγγελος γὰρ κυρίου κατέβη πρὸς αὐτὸν λέγων: Ἰωακείμ, Ἰωακείμ, εἰσήκουσε κύριος ὁ θεὸς τῆς δεήσεός σου. κατάβηθι ἐντεῦθεν. ἰδοὺ Ἄννα ἡ γυνή σου ἐν γαστρὶ λήψεται (εἴληφεν).

\pend\pstart
καὶ εὐθέως κατέβη Ἰωακεὶμ καὶ ἐκάλεσεν τοὺς ποιμένας αὐτοῦ λέγων: φέρετέ μοι ὧδε δώδεκα ἀμνάδας ἀσπίλους καὶ ἀμόμους εἰς θυσίαν κυρίῳ τῷ θεῷ μου, καὶ φέρετέ μοι δώδεκα μόσχους ἀσπίλους καὶ ἔσονται τοῖς ἱερεῦσι καὶ τῇ γερουσίᾳ, καὶ φέρετέ μοι ἑκατὸν χιμάρους καὶ ἔσονται αἱ ἑκατὸν χίμαροι παντὶ τῷ λαῷ.

\pend\pstart
καὶ ἰδοὺ ἥκει Ἰωακεὶμ μετὰ τῶν ποιμνίων αὐτοῦ. καὶ ἔστη Ἄννα πρὸς τῇ πύλῃ τοῦ οἴκου αὐτῆς καὶ εἶδεν τὸν Ἰωακεὶμ ἐρχόμενον μετὰ τῶν ποιμνίων αὐτοῦ. καὶ ἔδραμεν Ἄννα καὶ ἐκρεμάσθη ἐπὶ τὸν τράχηλον αὐτοῦ λέγουσα: νῦν οἶδα, ὅτι κύριος ὁ θεὸς εὐλόγησέ με σφόδρα: ἰδοὺ γὰρ ἡ χήρα οὐκέτι χήρα καὶ ἡ ἄτεκνος ἰδοὺ ἐν γαστρὶ λήψομαι εἴληφα . καὶ ἀνεπαύσατο Ἰωακεὶμ τὴν πρώτην ἡμέραν εἰς τὸν οἶκον αὐτοῦ.

\pend\pstart
\eledsection*{Εʹ}

\pend\setcounter{pstartL}{1}\pstart
Τῇ δὲ ἐπαύριον προσέφερε τὰ δῶρα αὐτοῦ λέγων ἐν ἑαυτῷ: ἐὰν κύριος ὁ θεὸς ἱλασθῇ μοι, τὸ πέταλον τοῦ ἱερέως φανερών μοι ποιήσει. καὶ προσέφερεν τὰ δῶρα αὐτοῦ Ἰωακεὶμ καὶ προσεῖχε τῷ πετάλῳ τοῦ ἱερέως, ὡς ἐπέβη ἐπὶ τὸ θυσιαστήριον κυρίου, καὶ ἁμαρτία οὐχ εὑρέθη ἐν αὐτῷ. καὶ εἶπεν Ἰωακείμ: νῦν οἶδα, ὅτι κύριος ὁ θεὸς ἱλάσθη μοι καὶ ἀφεῖλέν μου πάντα τὰ ἁμαρτήματα. καὶ κατέβη ἐκ ναοῦ κυρίου δεδικαιωμένος καὶ ἀπῆλθεν εἰς τὸν οἶκον αὐτοῦ χαίρων καὶ δοξάζων τὸν θεόν.

\pend\pstart
ἐπληρώθησαν δὲ οἱ μῆνες αὐτῆς. ἐν δὲ τῷ ἐνάτῳ μηνὶ ἐγέννησεν Ἄννα καὶ εἶπεν τῇ μαίᾳ: τί ἐγέννησα; ἡ δὲ εἶπεν: θῆλυ. καὶ εἶπεν Ἄννα: ἐμεγάλυνεν ἡ ψυχή μου τὴν ἡμέραν ταύτην καὶ ἀνέκλινεν αὐτήν. πληρωθεισῶν δὲ τῶν ἡμερῶν ἀπεσμήξατο Ἄννα καὶ ἔδωκεν μασθὸν τῇ παιδί. ἐκάλεσεν δὲ τὸ ὄνομα αὐτῆς Μαριάμ.

\pend\pstart
% \eledsection*{Ϛʹ}
\eledsection*{Ϝʹ}

\pend\setcounter{pstartL}{1}\pstart
Ἡμέρᾳ δὲ καὶ ἡμέρᾳ ἐκραταιοῦτο ἡ παῖς. γενομένης δὲ αὐτῆς ἑξαμήνου ἔστησεν αὐτὴν ἡ μήτηρ αὐτῆς χαμαὶ τοῦ πειράσαι, εἰ ἵσταται: καὶ περιπατήσασα ἑπτὰ βήματα ἦλθεν εἰς τὸν κόλπον τῆς μητρὸς αὐτῆς, καὶ ἀνήρπασεν αὐτὴν ἡ μήτηρ αὐτῆς λέγουσα: ζῇ κύριος ὁ θεός μου: οὐ μὴ περιπατήσῃς ἐν τῇ γῇ ταύτῃ, ἕως οὗ ἀπάξω σε ἐν τῷ ναῷ κυρίου. καὶ ἐποίησεν ἁγίασμα ἐν τῷ κοιτῶνι αὐτῆς καὶ πᾶν κοινὸν ἤ ἀκάθαρτον οὐκ εἴα διέρχεσθαι δι' αὐτῆς. καὶ ἐκάλεσε τὰς θυγατέρας τῶν Ἑβραίων τὰς ἀμιάντους, καὶ διεπλάνων αὐτήν.

\pend\pstart
ἐγένετο δὲ πρῶτος ἐνιαυτὸς τῇ παιδί, καὶ ἐποίησεν Ἰωακεὶμ δοχὴν μεγάλην καὶ ἐκάλεσεν τοὺς ἱερεῖς καὶ τοὺς γραμματεῖς καὶ τὴν γερουσίαν καὶ πάντα τὸν λαὸν Ἰσραήλ. καὶ προσήνεγκεν Ἰωακεὶμ τὴν παῖδα τοῖς ἱερεῦσι καὶ εὐλόγησαν αὐτὴν οἱ ἱερεῖς λέγοντες: ὁ θεὸς τῶν πατέρων ἡμῶν, εὐλόγησον τὴν παῖδα ταύτην καὶ δὸς αὐτῇ ὄνομα ὀνομαστὸν αἰώνιον ἐν πάσαις ταῖς γενεαῖς. καὶ εἶπεν ὁ λαός: γένοιτο, γένοιτο, ἀμήν. καὶ προσήνεγκεν Ἰωακεὶμ τὴν παῖδα τοῖς ἀρχιερεῦσι, καὶ εὐλόγησαν αὐτὴν λέγοντες: ὁ θεὸς τῶν ὑψωμάτων, ἐπίβλεψον ἐπὶ τὴν παῖδα ταύτην καὶ εὐλόγησον αὐτὴν ἐσχάτην εὐλογίαν, ἥτις διαδοχὴν οὐχ ἕξει.

\pend\pstart
καὶ ἀπήγαγον αὐτὴν ἐν τῷ ἁγιάσματι τοῦ κοιτῶνος αὐτῆς: καὶ λαβοῦσα Ἄννα ἔδωκε μασθὸν τῇ παιδὶ καὶ ᾖσεν ᾆσμα κυρίῳ τῷ θεῷ λέγουσα: ᾄσω ὠδὴν κυρίῳ τῷ θεῷ μου, ὅτι ἐπεσκέψατό με καὶ ἀφεῖλεν ἀπ' ἐμοῦ τὸν ὀνειδισμὸν τῶν ἐχθρῶν μου καὶ ἔδωκέ μοι καρπὸν δικαιοσύνης μονοούσιον αὐτῷ καὶ πολυπλούσιον. τίς ἀναγγελεῖ τοῖς υἱοῖς Ῥουβίμ , ὅτι Ἄννα θηλάζει; καὶ ἀνέπαυσεν αὐτὴν ἡ μήτηρ αὐτῆς ἐν τῷ ἁγιάσματι τοῦ κοιτῶνος αὐτῆς καὶ ἐξῆλθε καὶ διηκόνει αὐτοῖς. τελεσθέντος δὲ τοῦ δείπνου κατέβησαν εὐφραινόμενοι καὶ ἐδόξασαν τὸν θεὸν Ἰσραήλ.

\pend\pstart
\eledsection*{Ζʹ}

\pend\setcounter{pstartL}{1}\pstart
Τῇ δὲ παιδὶ προσετίθεντο οἱ μῆνες αὐτῆς. ἐγένετο δὲ διετὴς ἡ παῖς, καὶ εἶπεν Ἰωακείμ: ἀπάξωμεν αὐτὴν ἐν τῷ ναῷ κυρίου καὶ ἀποδῶμεν τὴν ἐπαγγελίαν, ἥν ἐπηγγειλάμεθα, μήπως ἀποστείλῃ κύριος ὁ θεὸς πρὸς ἡμᾶς καὶ γένηται ἀπρόσδεκτον τὸ δῶρον ἡμῶν. καὶ εἶπεν Ἄννα: ἀναμείνωμεν τὸ τρίτον ἔτος, ὅπως μὴ ζητήσῃ πατέρα ἤ μητέρα. καὶ εἶπεν Ἰωακείμ: ἀμήν, γένοιτο.

\pend\pstart
ἐγένετο δὲ τριετὴς ἡ παῖς, καὶ εἶπεν Ἰωακείμ: καλέσωμεν τὰς θυγατέρας τῶν Ἑβραίων τὰς ἀμιάντους, καὶ λαβέτωσαν ἀνὰ λαμπάδα, καὶ ἔστωσαν καιόμεναι, ἵνα μὴ ἐπιστραφῇ ἡ παῖς εἰς τὰ ὀπίσω καὶ αἰχμαλωτισθῇ ἡ καρδία αὐτῆς ἐκ ναοῦ κυρίου. καὶ ἐποίησαν οὕτως, ἕως οὗ ἀνέβησαν ἐν τῷ ναῷ κυρίου. καὶ ἐδέξατο αὐτὴν ὁ ἱερεὺς καὶ καταφιλήσας εὐλόγησε καὶ εἶπεν: ἐμεγάλυνε κύριος ὁ θεὸς τὸ ὄνομά σου ἐν πάσαις ταῖς γενεαῖς τῆς γῆς: (ἐπὶ σοὶ) ἐπ' ἐσχάτου τῶν ἡμερῶν φανερώσει κύριος ὁ θεὸς τὸ λύτρον τῶν υἱῶν Ἰσραήλ.

\pend\pstart
καὶ ἐκάθισεν αὐτὴν ἐπὶ τρίτου βαθμοῦ τοῦ θυσιαστηρίου, καὶ ἔβαλε κύριος ὁ θεὸς χάριν ἐπ' αὐτήν, καὶ κατεχόρευσε τοῖς ποσὶν αὐτοῖς, καὶ ἠγάπησεν αὐτὴν πᾶς οἶκος Ἰσραήλ.

\pend\pstart
\eledsection*{Ηʹ}

\pend\setcounter{pstartL}{1}\pstart
κατέβησαν δὲ οἱ γονεῖς αὐτῆς θαυμάζοντες καὶ ἐπαινοῦντες τὸν θεόν, ὅτι οὐκ ἐπεστράφη ἡ παῖς εἰς τὰ ὀπίσω. ἦν δὲ Μαριὰμ ὡσεὶ περιστερὰ νεμομένη ἐν τῷ ναῷ κυρίου καὶ ἐλάμβανε τροφὴν ἐκ χειρὸς ἀγγέλου.

\pend\pstart
γενομένης δὲ αὐτῆς δωδεκαετοῦς συμβούλιον ἐγένετο τῶν ἱερέων λεγόντων: ἰδοὺ Μαριὰμ γέγονε δωδεκαέτης ἐν τῷ ναῷ κυρίου: τί οὖν ποιήσωμεν αὐτήν, μήπως (ἐπέλθῃ αὐτῇ τὰ γυναικῶν καὶ) μιάνῃ τὸ ἁγίασμα κυρίου. καὶ εἶπον τῷ ἀρχιερεῖ: σὺ ἕστηκας ἐπὶ τὸ θυσιαστήριον θεοῦ: εἴσελθε καὶ πρόσευξαι περὶ αὐτῆς, καὶ ὅ ἄν φανερώσῃ σοι κύριος ὁ θεός, τοῦτο ποιήσωμεν.

\pend\pstart
καὶ εἰσῆλθεν ὁ ἱερεὺς λαβὼν τὸν δωδεκακόδωνα (ἱεροπρεπὲς ἱμάτιον) εἰς τὰ ἅγια τῶν ἁγίων καὶ ηὔξατο περὶ αὐτῆς. καὶ ἰδοὺ ἄγγελος κυρίου ἐπέστη αὐτῷ λέγων: Ζαχαρία, Ζαχαρία, ἔξελθε καὶ ἐκκλησίασον τοὺς χηρεύοντας τοῦ λαοῦ, καὶ ἐνεγκάτωσαν ἀνὰ ῥάβδον, καὶ εἰς ὅν ἐὰν δείξῃ κύριος ὁ θεὸς σημεῖον, τούτου ἔσται γυνή. καὶ ἐξῆλθον οἱ κήρυκες καθ' ὅλης τῆς περιχώρου τῆς Ἰουδαίας, καὶ ἤχησεν ἡ σάλπιγξ κυρίου, καὶ ἔδραμον πάντες.

\pend\pstart
\eledsection*{Θʹ}

\pend\setcounter{pstartL}{1}\pstart
Ἰωσὴφ δὲ ῥίψας τὸ σκέπαρνον ἔδραμε καὶ αὐτὸς εἰς τὴν συναγωγήν, καὶ συναχθέντες ὁμοῦ ἀπῆλθαν πρὸς τὸν ἱερέα. ἔλαβε δὲ πάντων τὰς ῥάβδους ὁ ἱερεὺς καὶ εἰσῆλθεν εἰς τὸ ἱερὸν καὶ ηὔξατο. τελέσας δὲ τὴν εὐχὴν ἐξῆλθε καὶ ἐπέδωκεν ἑνὶ ἑκάστῳ τὴν ἑαυτοῦ ῥάβδον, καὶ σημεῖον οὐκ ἦν ἐν αὐτοῖς. τὴν δὲ ἐσχάτην ῥάβδον ἔλαβεν ὁ Ἰωσήφ, καὶ ἰδοὺ περιστερὰ ἐξῆλθεν ἐκ τῆς ῥάβδου καὶ ἐπετάσθη ἐπὶ τὴν κεφαλὴν Ἰωσήφ. καὶ εἶπεν αὐτῷ ὁ ἱερεύς: σὺ κεκλήρωσαι τὴν παρθένον κυρίου παραλαβεῖν. παράλαβε αὐτὴν εἰς τήρησιν σεαυτῷ.

\pend\pstart
ἀντεῖπε δὲ Ἰωσὴφ λέγων: υἱοὺς ἔχω καὶ πρεσβύτης εἰμί, αὕτη δὲ νεωτέρα. μήπως κατάγελως γένωμαι τοῖς υἱοῖς Ἰσραήλ; εἶπεν δὲ αὐτῷ ὁ ἱερεύς: Ἰωσήφ, φοβήθητι κύριον τὸν θεὸν καὶ ὅσα ἐποίησε Δαθὰμ καὶ Κορὲ καὶ Ἀβηρών, πῶς ἐδιχάσθη ἡ γῆ καὶ κατεποντίσθησαν ἅπαντες διὰ τὴν ἀντιλογίαν αὐτῶν. καὶ νῦν φοβήθητι, Ἰωσήφ, μήπως ἔσται ταῦτα ἐν τῷ οἴκῳ σου.

\pend\pstart
καὶ φοβηθεὶς Ἰωσὴφ παρέλαβεν αὐτὴν εἰς τήρησιν. καὶ εἶπεν αὐτῇ: Μαρία, ἰδοὺ παρέλαβόν σε ἐκ ναοῦ κυρίου τοῦ θεοῦ μου καὶ νῦν καταλιμπάνω σε ἐν τῷ οἴκῳ μου, ἀπέρχομαι γὰρ οἰκοδομῆσαι τὰς οἰκοδομάς μου, καὶ ἐν τάχει ἥξω πρὸς σέ. κύριος ὁ θεὸς διαφυλάξει σε.

\pend\pstart
\eledsection*{Ιʹ}

\pend\setcounter{pstartL}{1}\pstart
Ἐγένετο δὲ συμβούλιον τῶν ἱερέων λεγόντων: ποιήσωμεν καταπέτασμα τῷ ναῷ κυρίου. καὶ εἶπεν ὁ ἱερεύς: καλέσατέ μοι ὧδε ἑπτὰ παρθένους ἀμιάντους ἐκ φυλῆς Δαυίδ. καὶ ἀπῆλθον οἱ ὑπηρέται καὶ εὕρησαν ἑπτά (εὗρον ἕξ). καὶ ἐμνήσθη ὁ ἱερεύς, ὅτι Μαρία ἐκ φυλῆς Δαυίδ ἐστι καὶ ἀμίαντός ἐστιν. καὶ ἀπῆλθαν οἱ ὑπηρέται καὶ ἤγαγον αὐτήν. καὶ εἰσήγαγεν αὐτὰς ὁ ἱερεὺς ἐν τῷ ναῷ κυρίου καὶ εἶπεν: λάχετέ μοι ὧδε, τίς νήσει τὸ χρυσίον καὶ τὸ ἀμίαντον καὶ τὸ βύσσινον καὶ τὸ σηρικοῦν καὶ τὸ ὑάκινθον καὶ τὸ κόκκινον καὶ τὴν ἀληθινὴν πορφύραν. καὶ ἔλαχεν τὴν Μαριὰμ τὸ κόκκινον καὶ ἡ ἀληθινὴ πορφύρα. καὶ λαβοῦσα ἀπῆλθεν εἰς τὸν οἶκον αὐτῆς. τῷ δὲ καιρῷ ἐκείνῳ Ζαχαρίας ἐσίγησεν. Μαριὰμ δὲ λαβοῦσα τὸ κόκκινον ἔκλωσεν.

\pend\pstart
\eledsection*{ΙΑʹ}

\pend\setcounter{pstartL}{1}\pstart
Καὶ λαβοῦσα κάλπιν ἐξῆλθεν γεμίσαι ὕδωρ, καὶ ἰδοὺ φωνὴ λέγουσα: χαῖρε κεχαριτωμένη, ὁ κύριος μετὰ σοῦ, εὐλογημένη σὺ ἐν γυναιξί. καὶ περιεβλέπετο δεξιὰ καὶ ἀριστερά, πόθεν αὕτη ἡ φωνὴ ὑπάρχει, καὶ ἔντρομος γενομένη ἀπῆλθεν εἰς τὸν οἶκον αὐτῆς. καὶ ἀναπαύσασα τὴν κάλπην ἔλαβε πάλιν τὴν πορφύραν καὶ ἐκάθισεν ἐπὶ τὸν θρόνον καὶ εἷλκεν αὐτήν.

\pend\pstart
καὶ ἰδοὺ ἄγγελος κυρίου ἐπέστη λέγων αὐτῇ: μὴ φοβοῦ, Μαριάμ, εὗρες γὰρ χάριν ἐνώπιον τοῦ θεοῦ καὶ συλλήψῃ ἐκ λόγου αὐτοῦ. ἀκούσασα δὲ Μαριὰμ διεκρίθη ἐν ἑαυτῇ λέγουσα: ἐγὼ συλλήψομαι, ὡς πᾶσα γυνὴ γεννᾷ;

\pend\pstart
καὶ λέγει πρὸς αὐτὴν ὁ ἄγγελος: οὐχ οὕτως, Μαριάμ: δύναμις γὰρ θεοῦ ἐπισκιάσει σοι, διὸ καὶ τὸ γεννόμενον (ἐκ σοῦ) ἅγιον κληθήσεται υἱὸς ὑψίστου, καὶ καλέσεις τὸ ὄνομα αὐτοῦ Ἰησοῦν: αὐτὸς γὰρ σώσει τὸν λαὸν αὐτοῦ ἀπὸ τῶν ἁμαρτιῶν αὐτῶν. καὶ εἶπεν Μαριάμ: ἰδοὺ ἡ δούλη κυρίου: γένοιτό μοι κατὰ τὸ ῥῆμά σου.

\pend\pstart
\eledsection*{ΙΒʹ}

\pend\setcounter{pstartL}{1}\pstart
Καὶ ἐποίησεν τὴν πορφύραν καὶ τὸ κόκκινον καὶ ἀπήνεγκεν αὐτὰ τῷ ἱερεῖ, καὶ εὐλόγησεν αὐτὴν ὁ ἱερεὺς καὶ εἶπεν: Μαριάμ, ἐμεγάλυνε κύριος ὁ θεὸς τὸ ὄνομά σου ἐν πάσαις ταῖς γενεαῖς τῆς γῆς καὶ ἔσῃ εὐλογημένη ὑπὸ κυρίου.

\pend\pstart
χαρὰν δὲ λαβοῦσα Μαριὰμ ἀπῆλθε πρὸς τὴν συγγενίδα αὐτῆς Ἐλισάβετ καὶ ἔκρουσε πρὸς τῇ θύρᾳ. καὶ ἀκούσασα Ἐλισάβετ ἔρριψε τὸ ἐν χερσὶν, καὶ δραμοῦσα ἤνοιξεν αὐτῇ καὶ εὐλόγησεν αὐτὴν καὶ εἶπεν: πόθεν μοι τοῦτο, ἵνα ἡ μήτηρ τοῦ κυρίου μου ἔλθῃ πρὸς ἐμέ; ἰδοὺ γὰρ τὸ ἐν ἐμοὶ βρέφος ἐσκίρτησε καὶ εὐλόγησέν σε. Μαριὰμ δὲ ἐπελάθετο τῶν μυστηρίων, ὧν εἶπεν πρὸς αὐτὴν Γαβριήλ, καὶ ἀτενίσασα εἰς τὸν οὐρανὸν εἶπεν: τίς εἰμι ἐγώ, ὅτι πᾶσαι αἱ γυναῖκες μακαριοῦσί με;

\pend\pstart
ἐποίησε δὲ τρεῖς μῆνας πρὸς τὴν Ἐλισάβετ καὶ ἀπῆλθεν εἰς τὸν οἶκον αὐτῆς. ἡμέρᾳ δὲ ἀφ' ἡμέρας ἡ γαστὴρ αὐτῆς ὀγκοῦτο, καὶ ἔκρυβεν ἑαυτὴν ἀπὸ τῶν υἱῶν Ἰσραήλ. ἦν δὲ ἐτῶν πεντεκαίδεκα, ὅτε τὰ μυστήρια ταῦτα ἐγένοντο.

\pend\pstart
\eledsection*{ΙΓʹ}

\pend\setcounter{pstartL}{1}\pstart
Ἐγένετο δὲ ἕκτος μὴν καὶ ἦλθεν Ἰωσὴφ ἀπὸ τῶν οἰκοδομῶν αὐτοῦ καὶ εἰσῆλθεν ἐν τῷ οἴκῳ αὐτοῦ καὶ εὗρε τὴν Μαριὰμ ὀγκωμένην. καὶ ἔτυψε τὸ πρόσωπον αὐτοῦ καὶ ἔρριψεν ἑαυτὸν χαμαὶ καὶ ἔκλαυσε λέγων: ποίῳ προσόπῳ ἀτενίσω πρὸς κύριον τὸν θεόν μου; τί δὴ εἴπω περὶ τῆς κόρης ταύτης, ὅτι παρθένον αὐτὴν παρέλαβον ἐκ ναοῦ κυρίου καὶ οὐκ ἐφύλαξα αὐτήν; τίς ὁ θηρεύσας με; τίς τὸ πονηρὸν τοῦτο ἐποίησεν ἐν τῷ οἴκῳ μου καὶ ἐμίανεν τὴν παρθένον; μήτι εἰς ἐμὲ ἀνεκεφαλαιόθη ἡ ἱστορία Ἀδάμ; ὥσπερ γὰρ Ἀδὰμ ἦν ἐν τῇ ὥρᾳ τῆς δοξολογίας αὐτοῦ καὶ ἦλθεν ὁ ὄφις καὶ εὗρεν τὴν Εὔαν μόνην καὶ ἐξηπάτησεν αὐτήν, οὕτως κἀμοί συνέβη.

\pend\pstart
καὶ ἀνέστη Ἰωσὴφ ἀπὸ τοῦ σάκκου καὶ ἐκάλεσε τὴν Μαριὰμ καὶ εἶπεν αὐτῇ: μεμελημένη τῷ θεῷ, τί τοῦτο ἐποίησας; τί ἐταπείνωσας τὴν ψυχήν σου; ἐπελάθου κυρίου τοῦ θεοῦ σου, ἡ ἀνατραφεῖσα εἰς τὰ ἅγια τῶν ἁγίων καὶ λαβοῦσα τροφὴν ἐκ χειρὸς ἀγγέλου καὶ χορεύσασα ἐν αὐτοῖς;

\pend\pstart
ἡ δὲ ἔκλαυσε πικρῶς λέγουσα: ζῇ κύριος ὁ θεός, καθότι καθαρά εἰμι ἐγὼ καὶ ἄνδρα οὐ γινώσκω. εἶπε δὲ αὐτῇ Ἰωσήφ: πόθεν οὖν ἐστι τοῦτο ἐν τῇ γαστρί σου; εἶπε δὲ αὐτῷ: ζῇ κύριος ὁ θεός μου, καθότι οὐ γινώσκω, πόθεν ἐστὶ τοῦτο τὸ ἐν τῇ γαστρί μου.

\pend\pstart
\eledsection*{ΙΔʹ}

\pend\setcounter{pstartL}{1}\pstart
Καὶ ἐφοβήθη Ἰωσὴφ σφόδρα καὶ ἠρέμησεν ἐξ αὐτῆς καὶ διελογίζετο, τί αὐτὴν ποιήσει, εἶπε δὲ ἐν ἑαυτῷ: ἐὰν αὐτῆς κρύψω τὸ ἁμάρτημα, εὑρεθήσομαι μαχόμενος τῷ νόμῳ κυρίου: καὶ ἐὰν αὐτὴν φανερὰν ποιήσω τοῖς υἱοῖς Ἰσραήλ, φοβοῦμαι, μήπως ἀγγελικόν ἐστι τὸ ἐν αὐτῇ καὶ εὑρεθήσομαι παραδιδοὺς αἷμα ἀθῷον εἰς κρίμα θανάτου. τί οὖν αὐτὴν ποιήσω; λάθρᾳ αὐτὴν ἀπολύσω ἀπ' ἐμοῦ. καὶ ταῦτα αὐτοῦ ἐνθυμουμένου κατέλαβεν αὐτὸν ἡ νύξ.

\pend\pstart
καὶ ἰδοὺ ἄγγελος κυρίου φαίνεται αὐτῷ κατ' ὄναρ λέγων: Ἰωσήφ (υἱὸς Δαυίδ), μὴ φοβηθῇς τὴν παῖδα ταύτην. τὸ γὰρ ἐν αὐτῇ γεννηθὲν ἐκ πνεύματός ἐστιν ἁγίου, καὶ καλέσεις τὸ ὄνομα αὐτοῦ Ἰησοῦν: αὐτὸς γὰρ σώσει τὸν λαὸν αὐτοῦ ἀπὸ τῶν ἁμαρτιῶν αὐτῶν. καὶ ἀνέστη Ἰωσὴφ ἀπὸ τοῦ ὕπνου καὶ ἐδόξασε τὸν θεὸν Ἰσραὴλ τὸν δόντα αὐτῷ τὴν χάριν ταύτην, καὶ ἐφύλασσε τὴν παῖδα.

\pend\pstart
\eledsection*{ΙΕʹ}

\pend\setcounter{pstartL}{1}\pstart
Ἠλθεν δὲ Ἄννας ὁ γραμματεὺς πρὸς αὐτὸν καὶ εἶπεν αὐτῷ: διὰ τί οὐκ ἐφάνης ἐν τῇ συναγωγῇ (συνόδῳ) ἡμῶν; καὶ εἶπεν αὐτῷ Ἰωσήφ: ὅτι κεκμηκὼς ἤμην ἐκ τῆς ὁδοῦ καὶ ἀνεπαυσάμην ἡμέραν μίαν . καὶ ἐστράφη Ἄννας καὶ εἶδεν τὴν παρθένον ὀγκωμένην.

\pend\pstart
καὶ ἀπελθὼν δρομαίως πρὸς τὸν (ἀρχ-)ἱερέα εἶπεν αὐτῷ: Ἰωσήφ, ὅν σὺ μαρτυρεῖς, ἠνόμησε σφόδρα. καὶ εἶπεν ὁ ἱερεύς: τί τοῦτο; καὶ εἶπεν Ἄννας: τὴν παρθένον, ἥν παρέλαβεν ἐκ ναοῦ κυρίου, ἐμίανεν αὐτήν. καὶ ἀποκριθεὶς ὁ ἱερεὺς εἶπεν αὐτῷ: Ἰωσὴφ; Ἰωσὴφ τοῦτο ἐποίησεν; καὶ εἶπεν Ἄννας: ἀπόστειλον ὑπηρέτας καὶ εὑρέσεις τὴν παρθένον ὀγκωμένην. καὶ ἀπῆλθον οἱ ὑπηρέται καὶ εὗρον αὐτήν, καθὼς εἶπεν, καὶ ἀπήγαγον ἅμα τῷ Ἰωσὴφ εἰς τὸ κριτήριον.

\pend\pstart
καὶ εἶπεν ὁ ἱερεύς: Μαριάμ, τί τοῦτο ἐποίησας καὶ ἐταπείνωσας τὴν ψυχήν σου καὶ ἐπελάθου κυρίου τοῦ θεοῦ σου, ἡ ἀνατραφεῖσα εἰς τὰ ἅγια τῶν ἁγίων καὶ λαβοῦσα τροφὴν ἐκ χειρὸς ἀγγέλων, σὺ ἡ ἀκούσασα τὸν ὕμνον αὐτῶν καὶ χορεύσασα ἐνώπιον αὐτῶν; τί τοῦτο ἐποίησας; ἡ δὲ ἔκλαυσε πικρῶς λέγουσα: ζῇ κύριος ὁ θεός, ὅτι καθαρά εἰμι ἐγὼ ἐνώπιον αὐτοῦ καὶ ἄνδρα οὐ γινώσκω.

\pend\pstart
καὶ εἶπεν ὁ ἀρχιερεύς: Ἰωσήφ, τί τοῦτο ἐποίησας; καὶ εἶπεν Ἰωσήφ: ζῇ κύριος ὁ θεός μου, ὅτι καθαρός εἰμι ἐξ αὐτῆς. καὶ εἶπεν ὁ ἀρχιερεύς: μὴ ψευδομαρτύρει, ἀλλὰ λέγε τὸ ἀληθές: ἔκλεψας τοὺς γάμους καὶ οὐκ ἐφανέρωσας τοῖς υἱοῖς Ἰσραήλ, καὶ οὐκ ἔκλινας τὴν κεφαλήν σου ὑπὸ τὴν κραταιὰν χεῖρα, ὅπως εὐλογηθῇ τὸ σπέρμα σου. καὶ Ἰωσὴφ ἐσίγησεν.

\pend\pstart
\eledsection*{ΙϜʹ}

\pend\setcounter{pstartL}{1}\pstart
Καὶ εἶπεν ὁ ἱερεύς: ἀπόδος τὴν παρθένον, ἥν παρέλαβες ἐκ ναοῦ κυρίου. καὶ περίδακρυς γενόμενος ὁ Ἰωσὴφ ἔστη. καὶ εἶπεν ὁ ἱερεύς: ποτιῶ ὑμᾶς τὸ ὕδωρ τῆς ἐλέγξεως κυρίου καὶ φανερώσει τὰ ἁμαρτήματα ὑμῶν ἐν ὀφθαλμοῖς ὑμῶν.

\pend\pstart
καὶ λαβὼν ὁ ἱερεὺς ἐπότισε τὸν Ἰωσὴφ καὶ ἔπεμψεν αὐτὸν εἰς τὴν ὀρεινήν: καὶ ἦλθεν ὁλόκληρος. ἐπότισεν δὲ καὶ τὴν παρθένον καὶ ἔπεμψεν καὶ αὐτὴν εἰς τὴν ὀρεινήν: καὶ ἦλθεν ὁλόκληρος, καὶ ἐθαύμασε πᾶς ὁ λαός, ὅτι ἁμαρτία οὐχ εὑρέθη ἐν αὐτοῖς.

\pend\pstart
καὶ εἶπεν ὁ ἱερεύς: εἰ κύριος ὁ θεὸς οὐκ ἐφανέρωσεν τὴν ἁμαρτίαν ὑμῶν, οὐδὲ ἐγὼ κρίνω ὑμᾶς καὶ ἀπέλυσεν αὐτούς. καὶ παρέλαβεν Ἰωσὴφ τὴν Μαριὰμ καὶ ἀπίει εἰς τὸν οἶκον αὐτοῦ χαίρων καὶ δοξάζων τὸν θεὸν τοῦ Ἰσραήλ.

\pend\pstart
\eledsection*{ΙΖʹ}

\pend\setcounter{pstartL}{1}\pstart
Κέλευσις δὲ ἐγένετο ἀπὸ (τοῦ Ἀόστου) Ἡρώδου τοῦ βασιλέως ἀπογράψασθαι, ὅσοι εἰσὶν ἐν Βηθλεὲμ τῆς Ἰουδαίας. (ἠναγκάζετο δὲ Ἰωσὴφ ἀπελθεῖν ἐκ Ναζαρὲτ εἰς τὴν Βηθλεὲμ καὶ εἶπεν) καὶ εἶπεν Ἰωσήφ: ἐγὼ ἀπογράψομαι τοὺς υἱούς μου. ταύτην δὲ τὴν παῖδα τί ποιήσω; πῶς αὐτὴν ἀπογράψομαι; γυναῖκα ἐμήν; ἐπαισχύνομαι. ἀλλὰ θυγατέρα; οἶδαν οἱ υἱοὶ Ἰσραήλ, ὅτι οὐκ ἔστιν θυγάτηρ μου. αὐτὴ ἡ ἡμέρα Κυρίου ποιήσει, ὡς βούλεται.

\pend\pstart
καὶ ἔστρωσεν τὸν ὄνον, καὶ ἐκάθισεν αὐτὴν καὶ ἧλκεν ὁ υἱὸς αὐτοῦ καὶ ἠκολούθησεν Σαμουήλ (αὐτός). καὶ ἤγγισαν ἐπὶ μίλιον τρίτον, καὶ ἐστράφη Ἰωσὴφ καὶ εἶδεν αὐτὴν στυγνὴν καὶ ἔλεγεν: ἴσως τὸ ἐν αὐτῇ χειμάζει αὐτήν. καὶ πάλιν ἐστράφη Ἰωσὴφ καὶ εἶδεν αὐτὴν γελοῦσαν καὶ εἶπεν: Μαριάμμη, τί ἐστίν σοι τοῦτο, ὅτι τὸ πρόσωπόν σου βλέπω ποτὲ μὲν γελοῦντα ποτὲ δὲ στυγνάζον; καὶ εἶπεν αὐτῷ: Ἰωσήφ, ὅτι δύο λαοὺς βλέπω ἐν τοῖς ὀφθαλμοῖς μου, ἔνα κλαίοντα καὶ κοπτόμενον καὶ ἔνα χαίροντα καὶ ἀγαλλιῶντα.

\pend\pstart
καὶ ἤλθωσεν ἀνὰ μέσον τῆς ὁδοῦ, καὶ εἶπεν αὐτῷ Μαριάμμη: κατάγαγέ με ἀπὸ τοῦ ὄνου, ὅτι (τ)ὸ ἐν ἐμοὶ ἐπείγει με προελθεῖν. καὶ κατήγαγεν αὐτὴν ἐκεῖ καὶ εἶπεν αὐτῇ: ποῦ σε ἀπάξω καὶ σκεπάσω σου τὴν ἀσχημοσύνην, ὅτι ὁ τόπος ἔρημός ἐστιν;

\pend\pstart
\eledsection*{ΙΗʹ}

\pend\setcounter{pstartL}{1}\pstart
Καὶ εὗρεν ἐκεῖ σπήλαιον καὶ εἰσήγαγεν αὐτὴν καὶ παρέστησεν αὐτῇ τοὺς υἱοὺς αὐτοῦ καὶ ἐξῆλθεν ζητῆσαι μαῖαν (Ἑβραίαν) ἐν χώρᾳ Βηθλεέμ.

\pend\pstart
ἐγὼ δὲ Ἰωσὴφ περιεπάτουν καὶ οὐ περιεπάτουν. καὶ ἀνέβλεψα εἰς τὸν πόλον τοῦ οὐρανοῦ καὶ εἶδον αὐτὸν ἑστῶτα, καὶ εἰς τὸν ἀέρα καὶ εἶδον αὐτὸν ἔκθαμβον, καὶ τὰ πετεινὰ τοῦ οὐρανοῦ ἠρεμοῦντα. καὶ ἐπέβλεψα ἐπὶ τὴν γῆν καὶ εἶδον σκάφην κειμένην καὶ ἐργάτας ἀνακειμένους, καὶ ἦσαν αἱ χεῖρες αὐτῶν ἐν τῇ σκάφῃ. καὶ οἱ μασόμενοι οὐκ ἐμασῶντο, καὶ οἱ αἴροντες οὐκ ἀνέφερον, καὶ οἱ προσφέροντες τῷ στόματι αὐτῶν οὐ προσέφερον. ἀλλὰ πάντων ἦν τὰ πρόσωπα ἄνω βλέποντα.

\pend\pstart
καὶ εἶδον ἐλαυνόμενα πρόβατα, καὶ τὰ πρόβατα ἑστήκει: καὶ ἐπῆρεν ὁ ποιμὴν τὴν χεῖρα αὐτοῦ τοῦ πατάξαι αὐτά, καὶ ἡ χεὶρ αὐτοῦ ἔστη ἄνω. καὶ ἀνέβλεψα ἐπὶ τὸν χείμαρρον τοῦ ποταμοῦ καὶ εἶδον ἐρίφους καὶ τὰ στόματα αὐτῶν ἐπικείμενα τῷ ὕδατι καὶ μὴ πίνοντα. καὶ πάντα ὑπὸ θῆξιν (θήζει, θίζει, θρίζιν, ἔκπληξιν) τῷ δρόμῳ ἀπηλαύνοντο.

\pend\pstart
\eledsection*{ΙΘʹ}

\pend\setcounter{pstartL}{1}\pstart
Καὶ εἶδον γυναῖκα καταβαίνουσαν ἀπὸ τῆς ὀρεινῆς καὶ εἶπέν μοι: ἄνθρωπε, ποῦ πορεύῃ; καὶ εἶπον αὐτῇ: μαῖαν ζητῶ. καὶ ἀποκριθεῖσά μοι εἶπεν: ἐξ Ἰσραήλ; καὶ εἶπον αὐτῇ: ναί, κυρία. καὶ εἶπέν μοι: τίς ἐστιν ἡ γεννήσασα ἐν τῇ σπηλαίῳ; καὶ εἶπον ἐγώ: ἡ μεμνηστευμένη μοι. καὶ εἶπέν μοι: οὐκ ἔστι σου γυνή; καὶ εἶπον αὐτῇ: Μαριάμ ἐστιν καὶ ἐκληρωσάμην αὐτὴν εἰς γυναῖκα, ἥτις ἀνετράφη εἰς τὰ ἅγια τῶν ἁγίων: καὶ οὐκ ἔστι μου γυνή, ἀλλὰ σύλληψιν ἔχει ἐκ πνεύματος ἁγίου. καὶ εἶπεν: εἰπέ μοι τὸ ἀληθές. καὶ εἶπον αὐτῇ: ἐλθὲ καὶ ἴδε. καὶ ἀπῆλθεν μετ' αὐτοῦ.

\pend\pstart
καὶ ἔστη ἐν τῷ τόπῳ τοῦ σπηλαίου, καὶ ἦν νεφέλη ἐπισκιάζουσα ἐπὶ τὸ σπήλαιον: καὶ εἶπεν ἡ μαῖα: ἐμεγαλύνθη ἡ ψυχή μου τῇ σήμερον ἡμέρᾳ, ὅτι εἶδον καινὸν θέαμα καὶ παράδοξον: ὅτι σωτηρίον τῷ Ἰσραὴλ ἐγενήθη. καὶ παραχρῆμα ἡ νεφέλη ὑπεστέλλετο ἐκ τοῦ σπηλαίου, καὶ ἐφάνη φῶς μέγα ἐν τῷ σπηλαίῳ, ὥστε τοὺς ὀφθαλμοὺς ἡμῶν μὴ φέρειν. καὶ πρὸς ὀλίγον τὸ φῶς ἐκεῖνο ὑπεστέλλετο, ἕως ἐφάνη τὸ βρέφος (καὶ ἦλθεν) καὶ ἔλαβεν μασθὸν ἐκ τῆς μητρὸς αὐτοῦ Μαρίας. (καὶ ἀνεβόησεν ἡ μαῖα: ὡς μεγάλη ἡ σήμερον ἡμέρα, ὅτι εἶδον τὸ καινὸν θέαμα τοῦτο.)

\pend\pstart
καὶ ἐξῆλθεν ἐκ τοῦ σπηλαίου ἡ μαῖα καὶ ἀπήντησεν Σαλώμην, καὶ εἶπεν αὐτῇ: Σαλώμη, Σαλώμη, καινόν σοι ἔχω διηγήσασθαι θέαμα: παρθένος ἐγέννησεν, ὅ οὐ χωρεῖ φύσις ἀνθρωπίνη. καὶ εἶπεν Σαλώμη: ζῇ κύριος ὁ θεός, ἐὰν μὴ κατανοήσω (ἐὰν μὴ βάλω τὴν χεῖρά μου εἰς αὐτήν), οὐ μὴ πιστεύσω, ὅτι παρθένος ἐγέννησεν.

\pend\pstart
\eledsection*{Κʹ}

\pend\setcounter{pstartL}{1}\pstart
Καὶ εἰσῆλθεν Σαλώμη καὶ εἶπεν: Μαρία, σχημάτισον σεαυτήν: οὐ γὰρ μικρὸς ἀγὼν περίκειται περὶ σοῦ. καὶ κατενόησεν αὐτήν. καὶ ἠλάλαξεν Σαλώμη καὶ ἐκραύγασε λέγουσα: οὐαὶ τῇ ἀνομίᾳ μου καὶ οὐαὶ τῇ ἀπιστίᾳ μου, ὅτι ἐξεπείρασα θεὸν ζῶντα: καὶ ἰδοὺ ἡ χείρ μου ἐν πυρὶ φλέγεται (ἀποπίπτει).

\pend\pstart
καὶ ἔκλινεν τὰ γόνατα αὐτῆς Σαλώμη πρὸς τὸν δεσπότην λέγουσα: ὁ θεὸς τῶν πατέρων μου, μνήσθητί μου, ὅτι σπέρμα εἰμὶ Ἀβραὰμ καὶ Ἰσαὰκ καὶ Ἰακώβ: μὴ παραδειγματίσῃς με τοῖς υἱοῖς Ἰσραήλ, ἀλλὰ ἀπόδος μοι ἐμὴν ὁλοκληρίαν.

\pend\pstart
καὶ ἰδοὺ ἄγγελος κυρίου ἔστη πρὸς Σαλώμην λέγων: Σαλώμη, Σαλώμη, ἐπήκουσε κύριος ὁ θεὸς τῆς δεήσεός σου: ἔγγισον πρὸς τὸ παιδίον καὶ βάστασον αὐτό, καὶ ἔσται σοι σωτηρία μεγάλη.

\pend\pstart
καὶ προσῆλθεν Σαλώμη καὶ ἐβάστασεν αὐτό, καὶ εἶπεν: ὄντως βασιλεὺς μέγας ἐγεννήθη τῷ Ἰσραήλ. καὶ εὐθέως ἰάθη Σαλώμη καὶ ἐξῆλθεν ἐκ τοῦ σπηλαίου δεδικαιωμένη, καὶ ἰδοὺ φωνὴ λέγουσα αὐτῇ: Σαλώμη, Σαλώμη, μὴ ἀναγγείλῃς, ὅσα εἶδες παράδοξα (ἕως ἔλθῃ εἰς Ἰερουσαλήμ).

\pend\pstart
\eledsection*{ΚΑʹ}

\pend\setcounter{pstartL}{1}\pstart
Καὶ ἰδοὺ Ἰωσὴφ ἡτοιμάσθη ἐξελθεῖν εἰς τὴν Ἰουδαίαν, καὶ θόρυβος ἐγένετο ἐν Βηθλεέμ. ἦλθαν γὰρ μάγοι ἀπὸ ἀνατολῶν (ἐκ Περσίδος) λέγοντες: ποῦ ἐστιν ὁ τεχθεὶς βασιλεὺς τῶν Ἰουδαίων; εἴδομεν γὰρ αὐτοῦ τὸν ἀστέρα ἐν τῇ ἀνατολῇ καὶ ἤλθομεν προσκυνῆσαι αὐτόν.

\pend\pstart
καὶ ἀκούσας Ἡρώδης ἐταράχθη καὶ ἔπεμψεν ὑπηρέτας πρὸ(ς) τοὺς μάγους, καὶ ἀπέστειλεν πρὸς τοὺς ἀρχιερεῖς καὶ ἀνέκρινεν αὐτοὺς λέγων: ποῦ ὁ χριστὸς γεννᾶται; οἱ δὲ εἶπον: ἐν Βηθλεὲμ τῆς Ἰουδαίας: οὕτως γὰρ γέγραπται. καὶ ἀπέλυσεν αὐτοὺς καὶ ἀνέκρινε τοὺς μάγους λέγων αὐτοῖς: τί εἴδετε σημεῖον ἐπὶ τὸν γεννηθέντα βασιλέα; καὶ εἶπον οἱ μάγοι: εἴδομεν ἀστέρα παμμεγέθη λάμψαντα ἐν τοῖς ἄστροις τούτοις καὶ ἀμβλύνοντα αὐτοὺς τοῦ (μὴ) φαίνειν καὶ ἔγνωμεν, ὅτι βασιλεὺς ἐγεννήθη τῷ Ἰσραήλ: καὶ διὰ τοῦτο ἤλθομεν προσκυνῆσαι αὐτόν. καὶ εἶπεν Ἡρώδης: πορευθέντες ἀκριβῶς ἐκζητήσατε περὶ τοῦ παιδίου: καὶ ἐπὰν εὕρηται, ἀπαγγείλατέ μοι, ὅπως κἀγὼ ἐλθὼν προσκυνήσω αὐτόν.

\pend\pstart
καὶ ἐξῆλθον οἱ μάγοι, καὶ ἰδοὺ ὁ ἀστήρ, ὅν εἶδον ἐν τῇ ἀνατολῇ, προῆγεν αὐτῶν, ἕως οὗ ἐλθὼν ἔστη εἰς τὸ σπήλαιον ἐπὶ τῆς κεφαλῆς τοῦ παιδίου. καὶ ἰδόντες αὐτὸ οἱ μάγοι μετὰ τῆς μητρὸς αὐτοῦ Μαρίας προσεκύνησαν αὐτὸ καὶ ἀνοίξαντες τοὺς θησαυροὺς αὐτῶν προσήνεγκαν αὐτῶν δῶρα, χρυσὸν καὶ λίβανον καὶ σμύρναν. καὶ χρηματισθέντες ὑπὸ ἁγίου ἀγγέλου (μὴ εἰσελθεῖν εἰς τὴν Ἰουδαίαν πρὸς Ἡρώδην) δι' ἄλλης ὁδοῦ ἐπορεύθησαν εἰς τὴν χώραν αὐτῶν.

\pend\pstart
\eledsection*{ΚΒʹ}

\pend\setcounter{pstartL}{1}\pstart
Γνοὺς δὲ ὁ Ἡρώδης, ὅτι ἐνεπαίχθη ὑπὸ τῶν μάγων, ὀργισθεὶς ἔπεμψεν τοὺς φονευτὰς κελεύσας αὐτοῖς ἀνελεῖν τὰ βρέφη ἀπὸ διετοῦς καὶ κατωτέρω.

\pend\pstart
ἀκούσασα δὲ Μαριάμ, ὅτι τὰ βρέφη ἀναιροῦνται, φοβηθεῖσα ἔλαβεν τὸ παιδίον μετὰ Ἰωσὴφ καὶ ἀπεδήμησεν εἰς Αἴγυπτον, καθὼς ἐχρηματίσθη αὐτοῖς.

\pend\pstart
ἡ δὲ Ἐλισάβετ λαβοῦσα τὸν Ἰωάννην ἀνέβη εἰς τὴν ὀρεινὴν καὶ περιεβλέπετο, ποῦ αὐτὸν ἀποκρύψει: καὶ οὐκ ἦν αὐτοῖς τόπος ἀποκρυβῆς. τότε στενάξασα λέγει: ὄρος, ὄρος, δέξαι μητέρα μετὰ τέκνου . οὐ γὰρ ἠδύνατο πορεύεσθαι. καὶ παραχρῆμα ἐδιχάσθη τὸ ὄρος καὶ ἐδέξατο αὐτήν. καὶ ἦν τὸ ὄρος ἐκεῖνο διαφαῖνον αὐτοῖς καὶ ἄγγελος κυρίου ὁδηγῶν αὐτούς.

\pend\pstart
\eledsection*{ΚΓʹ}

\pend\setcounter{pstartL}{1}\pstart
Ὁ δὲ Ἡρώδης ἐζήτει τὸν Ἰωάννην καὶ ἀπέστειλεν ὑπηρέτας εἰς τὸ θυσιαστήριον κυρίου πρὸς Σαχαρίαν λέγων: ποῦ ἀπέκρυψας τὸν υἱόν σου; ὁ δὲ εἶπεν αὐτοῖς: ἐγὼ λειτουργὸς ὑπάρχω κυρίου τοῦ θεοῦ καὶ παρεδρεύω τῷ ναῷ αὐτοῦ καὶ οὐ γινώσκω, ποῦ ἐστιν ὁ υἱός μου.

\pend\pstart
οἱ δὲ ὑπηρέται πορευθέντες ἀνήγγειλαν τῷ Ἡρώδῃ. καὶ ὀργισθεὶς ὁ Ἡρώδης ἀπέστειλεν ἐκ δευτέρου πρὸς Σαχαρίαν λέγων: εἰπέ μοι τὸ ἀληθές, ποῦ ἐστιν ὁ υἱός σου: οἶδας γάρ, ὅτι τὸ αἷμά σου ὑπὸ τὴν χεῖρά μού ἐστιν. οἱ δὲ ὑπηρέται ἀπῆλθον καὶ ἀνήγγειλαν τῷ Σαχαρίᾳ ταῦτα.

\pend\pstart
καὶ εἶπεν αὐτοῖς ὁ Σαχαρίας: εἴπατε τῷ Ἡρώδῃ: εἰ καὶ τὸ αἷμά μου ἐκχέεις, τὸ πνεῦμά μου ὁ δεσπότης λήψεται, πλὴν ὅτι ἀθῷον αἷμα ἐκχύνεις παρὰ τὰ πρόθυρα τοῦ ναοῦ κυρίου. οὐ γὰρ γινώσκω, ποῦ ἐστιν ὁ υἱός μου. καὶ περὶ τὸ διάφαυμα ἐφονεύθη Σαχαρίας. καὶ οὐκ ᾔδεισαν οἱ υἱοὶ Ἰσραήλ, πῶς ἐφονεύθη.

\pend\pstart
\eledsection*{ΚΔʹ}

\pend\setcounter{pstartL}{1}\pstart
Ἀλλὰ τῇ ὥρᾳ τοῦ ἀσπασμοῦ ἀπῆλθον οἱ ἱερεῖς, καὶ οὐκ ἀπήντησεν αὐτοῖς ὁ Σαχαρίας κατὰ τὸ εἰωθός, καὶ ἔστησαν οἱ ἱερεῖς προσδοκῶντες τὸν Σαχαρίαν τοῦ ἀσπάσασθαι αὐτὸν ἐν εὐχαῖς καὶ δοξάσαι τὸν θεόν.

\pend\pstart
χρονίσαντος δὲ αὐτοῦ ἐφοβήθησαν ἅπαντες. ἀποτολμήσας δὲ εἷς ἐξ αὐτῶν εἰσῆλθεν καὶ εἶδεν παρὰ τὸ θυσιαστήριον κυρίου αἷμα πεπηγός. καὶ ἰδοὺ φωνὴ λέγουσα: Σαχαρίας πεφόνευται καὶ οὐκ ἐξαλειφθήσεται τὸ αἷμα αὐτοῦ, ἕως οὗ ἔλθῃ ὁ ἔκδικος αὐτοῦ. ὁ δὲ ἀκούσας τὸν λόγον τοῦτον ἐφοβήθη καὶ ἐλθὼν ἀνήγγειλε τοῖς ἱερεῦσιν, ἅ εἶδεν καὶ ἤκουσεν.

\pend\pstart
καὶ τολμήσαντες εἰσῆλθον καὶ εἶδον τὸ γεγονός. καὶ τὰ δὲ φατνώματα τοῦ ναοῦ ὀλόλυξαν, καὶ αὐτοὶ διεσχίσαντο τὰ ἱμάτια αὐτῶν ἀπὸ ἄνωθεν ἕως κάτω. τὸ δὲ σῶμα αὐτοῦ οὐχ εὗρον, ἀλλ' εὗρον τὸ αἷμα αὐτοῦ ὡσεὶ λίθον γεγενημένον. ἐξελθόντες δὲ ἀνήγγειλαν τῷ λαῷ ὅτι Σαχαρίας πεφόνευται. καὶ ἤκουσαν πᾶσαι αἱ φυλαὶ τοῦ λαοῦ καὶ ἐπένθησαν αὐτὸν τρεῖς ἡμέρας καὶ τρεῖς νύκτας.

\pend\pstart
μετὰ δὲ τὰς ἡμέρας ἐκείνας ἐβουλεύσαντο οἱ ἱερεῖς, τίνα ἀναστήσωσιν εἰς τὸν τόπον Σαχαρίου, καὶ ἔβαλον κλήρους: καὶ ἔπεσεν ὁ κλῆρος ἐπὶ Συμεῶνα. αὐτὸς γὰρ ἦν χρηματισθεὶς ὑπὸ τοῦ ἁγίου πνεύματος τοῦ μὴ ἰδεῖν θάνατον, ἕως ἄν ἴδῃ τὸν χριστὸν κυρίου (ἐν σαρκί).

\pend\pstart
\eledsection*{ΚΕʹ}

\pend\setcounter{pstartL}{1}\pstart
Ἐγὼ δὲ Ἰάκωβος ὁ γράψας τὴν ἱστορίαν ταύτην, ἐν Ἱεροσολύμοις θορύβου γενομένου, ὅτε ἐτελεύτησεν Ἡρώδης, συνέστελλεν ἑαυτὸν ἐν τῇ ἐρήμῳ, ἕως παύσηται ὁ θόρυβος: δοξάσω δὲ τὸν δεσπότην τὸν δόντα μοι τὴν σοφίαν τοῦ γράψαι τὴν ἱστορίαν ταύτην.

\pend\pstart
καὶ ἔσται ἡ χάρις μετὰ πάντων τῶν φοβουμένων τὸν κύριον, ἀμήν.

\pend
