\documentclass[10pt]{book}

\usepackage{fontspec}
\usepackage{fullpage} % to reduce the margins
\usepackage{titlesec}
\usepackage{lettrine}
\usepackage{reledmac}
\usepackage{reledpar}
\usepackage{xcolor}

\setmainfont{EB Garamond}
\newfontfamily{\greekfont}{GFS Artemisia}

% hide reledpar line numbers
\numberlinefalse
% format reledpar paragraph/verse numbers
\renewcommand{\thepstartL}{\textnormal{\textcolor{benred8}{\arabic{pstartL}. }}}
\renewcommand{\thepstartR}{\textnormal{\textcolor{benred8}{\arabic{pstartR}. }}}

% RED
\definecolor{benred8}{HTML}{E82C00} 

% BLUE
\definecolor{benblue1}{HTML}{2B22C7}

% YELLOWS
\definecolor{benyellow1}{HTML}{FFD435}
\definecolor{benyellow2}{HTML}{7C6F3B}

\begin{document}

\title{\scshape{\greekfont ΠΡΩΤΕΥΑΓΓΕΛΙΟΝ ΙΑΚΩΒΟΥ}\\The Protevangelium of James}
\author{ed. Bernard Stockermans}
\date{A.D. MMXX}
\maketitle

\chapter*{}

{\center\Large
The Birth of Mary the Holy Mother of God, and Very Glorious Mother of Jesus Christ.\footnote{This title is taken by Tischendorf from a manuscript of the eleventh century (Paris). At least seventeen other forms exist. The book is variously named by ancient writers. In the decree of Gelasius (A.D. 495) he condemns it as \textit{Evangelium nomine Jacobi minoris apocryphum}. The text of Tischendorf, here translated, is somewhat less diffuse than that of Fabricius, and is based on manuscript evidence. The variations are verbal and formal rather than material.\textemdash R.}}

\begin{pairs}
\begin{Leftside}
\beginnumbering
\numberpstarttrue

{\greekfont
\pstart
\eledsection*{Αʹ}

\pend\pstart
Ἐν ταῖς ἱστορίαις τῶν δώδεκα φυλῶν τοῦ Ἰσραὴλ ἦν Ἰωακεὶμ πλούσιος σφόδρα, καὶ προσέφερε κυρίῳ τὰ δῶρα αὐτοῦ διπλᾶ λέγων ἐν ἑαυτῷ: Ἔσται τὸ τῆς περισσείας μου ἅπαντι τῷ λαῷ καὶ τὸ τῆς ἀφέσεως κυρίῳ τῷ θεῷ εἰς ἱλασμὸν ἐμοί.

\pend\pstart
ἐνἤγγισεν δὲ ἡ ἡμέρα κυρίου ἡ μεγάλη καὶ προσέφερον οἱ υἱοὶ Ἰσραὴλ τὰ δῶρα αὐτῶν, καὶ ἔστη κατενώπιον αὐτοῦ καὶ Ῥουβὴλ λέγων: οὐκ ἔξεστίν σοι πρώτῳ προσενεγκεῖν τὰ δῶρά σου, καθότι σπέρμα οὐκ ἐποίησας ἐν τῷ Ἰσραήλ.

\pend\pstart
καὶ ἐλυπήθη Ἰωακεὶμ καὶ ἀπίει εἰς τὸν οἶκον αὐτοῦ, καὶ ἐλθὼν εἰς τὴν δωδεκάφυλον τοῦ λαοῦ λέγει: ὄψομαι, εἰ ἐγὼ μόνος οὐκ ἐποίησα σπέρμα ἐν τῷ Ἰσραήλ. ἠρεύνησε δὲ καὶ εὗρε πάντας τοὺς δικαίους, ὅτι σπέρμα ἀνέστησαν ἐν τῷ Ἰσραὴλ, καὶ ἐμνήσθη τοῦ πατριάρχου Ἀβραάμ, ὅτι ἐν ταῖς ἐσχάταις αὐτοῦ ἡμέραις ἔδωκεν αὐτῷ ὁ θεὸς υἱὸν Ἰσαάκ.

\pend\pstart
καὶ ἐλυπεῖτο Ἰωακεὶμ σφόδρα καὶ οὐκ ἐφάνη τῇ γυναικὶ αὐτοῦ, ἀλλὰ ἔδωκεν ἑαυτὸν εἰς τὴν ἔρημον, καὶ ἔπηξε τὴν σκηνὴν αὐτοῦ ἐκεῖ καὶ ἐνήστευσεν ἡμέρας τεσσεράκοντα καὶ νύκτας τεσσεράκοντα λέγων ἑν ἑαυτῷ: οὐ καταβήσομαι οὔτε ἐπὶ βρωτὸν οὔτε ἐπὶ ποτόν, ἕως ἐπισκέψηταί με κύριος ὁ θεός μου, καὶ ἔσται μοι ἡ εὐχὴ βρόματα καὶ πόματα.

\pend\pstart
\eledsection*{Βʹ}

\pend\pstart
Ἡ δὲ γυνὴ δὲ αὐτοῦ Ἄννα δύο θρήνους ἐθρήνει καὶ δύο κοπετοὺς ἐκόπτετο λέγουσα: κόψομαι τὴν χηρίαν μου καὶ κόψομαι τὴν ἀτεκνίαν μου.

\pend\pstart
ἤγγισε δὲ ἡ ἡμέρα κυρίου ἡ μεγάλη καὶ εἶπεν Ἰουδὴθ ἡ παιδίσκη αὐτῆς πρὸς αὐτήν: ἕως πότε ταπεινοῖς τὴν ψυχήν σου; ἰδοὺ γὰρ ἤγγισεν ἡ ἡμέρα κυρίου ἡ μεγάλη καὶ οὐκ ἔξεστί σοι πενθεῖν. ἀλλὰ λάβε τοῦτο τὸ κεφαλο\-δέσμιον, ὅ ἔδωκέν μοι ἡ κυρία τοῦ ἔργου, καὶ οὐκ ἔξεστί μοι ἀναδήσασθαι αὐτό, καθότι παιδίσκη σού εἰμι καὶ χαρακτῆρα ἔχει βασιλικόν.

\pend\pstart
καὶ εἶπεν Ἄννα: ἀπόστηθι ἀπ' ἐμοῦ: καὶ ταῦτα οὐκ ἐποίησα, καὶ κύριος ὁ θεὸς ἐτα\-πείνωσέν με σφόδρα. μήπως πανοῦργος ἔδωκέν σοι τοῦτο καὶ ἦλθες κοινωνῆσαί με τῇ ἁμαρτίᾳ σου; εἶπεν δὲ αὐτῇ Ἰουδὴθ ἡ παιδίσκη αὐτῆς: τί ἀράσωμαί σοι, καθότι οὐκ ἤκουσας τῆς φωνῆς μου; ἀπέκλεισεν κύριος ὁ θεὸς τὴν μήτραν σου τοῦ μὴ δοῦναί σοι καρπὸν ἐν τῷ Ἰσραήλ.

\pend\pstart
καὶ ἐλυπήθη Ἄννα σφόδρα καὶ περιείλετο τὰ ἱμάτια αὐτῆς τὰ πενθικὰ καὶ ἐσμήξατο τὴν κεφαλὴν αὐτῆς καὶ ἐνεδύσατο τὰ ἱμάτια αὐτῆς τὰ νυμφικὰ καὶ περὶ ὥραν ἐννάτην κατέβη εἰς τὸν παράδεισον αὐτῆς (τοῦ περι\-πατῆσαι). καὶ εἶδεν δάφνην καὶ ἐκάθισεν ὑποκάτω αὐτῆς καὶ ἐλιτάνευσε τῷ δεσπότῃ λέγουσα: ὁ θεὸς τῶν πατέρων μου, εὐλόγησόν με καὶ ἐπάκουσον τῆς δεήσεός μου, καθὼς ἐπήκουσας καὶ εὐλόγησας τὴν μητέραν Σάραν καὶ ἔδωκας αὐτῇ υἱὸν τὸν Ἰσαάκ.

\pend\pstart
\eledsection*{Γʹ}

\pend\pstart
Καὶ ἀτενίσασα Ἄννα εἰς οὐρανὸν εἶδεν καλιὰν στρουθίων ἐν τῇ δάφνῃ καὶ εὐθέως ἐποίησε θρῆνον ἐν ἑαυτῇ λέγουσα: οἴμοι, τίς με ἐγέννησεν, ποία δὲ μήτρα ἐξέφυσέν με, ὅτι κατάρα ἐγεννήθην ἐνώπιον τῶν υἱῶν Ἰσραήλ καὶ ὠνειδίσθην καὶ ἐξεμυκτηρίσθην ἐκβληθεῖσα ἐκ ναοῦ κυρίου τοῦ θεοῦ μου;

\pend\pstart
οἴμοι, τίνι ὁμοιώθην ἐγώ; οὐχ ὁμοιώθην ἐγὼ τοῖς πετεινοῖς τοῦ οὐρανοῦ, ὅτι καὶ τὰ πετεινὰ γόνιμά εἰσιν ἐνώπιόν σου, κύριε. οἴμοι, τίνι ὁμοιώθην ἐγώ; οὐχ ὁμοιώθην ἐγὼ τοῖς ἀλόγοις ζώοις, καὶ τὰ ἄλογα ζῶα γόνιμά εἰσιν ἐνώπιόν σου, κύριε.

\pend\pstart
οἴμοι, τίνι ὁμοιώθην ἐγώ; οὐχ ὁμοιώθην ἐγὼ τοῖς ὕδασι τούτοις, ὅτι καὶ τὰ ὕδατα γόνιμά εἰσιν ἐνώπιόν σου, κύριε. οἴμοι, τίνι ὁμοιώθην ἐγώ; οὐχ ὁμοιώθην ἐγὼ τῇ γῇ, ὅτι καὶ ἡ γῆ προφέρει τοὺς καρποὺς αὐτῆς κατὰ καιρὸν καί σε εὐλογεῖ, κύριε.

\pend\pstart
\eledsection*{Δʹ}

\pend\pstart
Καὶ ἰδοὺ ἄγγελος κυρίου ἐπέστη λέγων: Ἄννα, Ἄννα, εἰσήκουσε κύριος ὁ θεὸς τῆς δεήσεός σου, καὶ λήψῃ καὶ λαληθήσεται τὸ σπέρμα σου ἐν ὅλῃ τῇ οἰκουμένῃ. εἶπεν δὲ Ἄννα: ζῇ κύριος ὁ θεός μου: ἐὰν γεννήσω εἴτε ἄρρεν εἴτε θῆλυ, προσάξω αὐτὸ δῶρον κυρίῳ τῷ θεῷ μου καὶ ἔσται λειτουργοῦν αὐτῷ πάσας ἡμέρας τῆς ζωῆς αὐτοῦ.

\pend\pstart
καὶ ἰδοὺ ἤλθοσαν ἄγγελοι δύοι λέγοντες αὐτῇ: ἰδοὺ Ἰωακεὶμ ὁ ἀνήρ σου ἔρχεται μετὰ τῶν ποιμνίων αὐτοῦ. ἄγγελος γὰρ κυρίου κατέβη πρὸς αὐτὸν λέγων: Ἰωακείμ, Ἰωακείμ, εἰσήκουσε κύριος ὁ θεὸς τῆς δεήσεός σου. κατάβηθι ἐντεῦθεν. ἰδοὺ Ἄννα ἡ γυνή σου ἐν γαστρὶ λήψεται (εἴληφεν).

\pend\pstart
καὶ εὐθέως κατέβη Ἰωακεὶμ καὶ ἐκάλεσεν τοὺς ποιμένας αὐτοῦ λέγων: φέρετέ μοι ὧδε δώδεκα ἀμνάδας ἀσπίλους καὶ ἀμόμους εἰς θυσίαν κυρίῳ τῷ θεῷ μου, καὶ φέρετέ μοι δώδεκα μόσχους ἀσπίλους καὶ ἔσονται τοῖς ἱερεῦσι καὶ τῇ γερουσίᾳ, καὶ φέρετέ μοι ἑκατὸν χιμάρους καὶ ἔσονται αἱ ἑκατὸν χίμαροι παντὶ τῷ λαῷ.

\pend\pstart
καὶ ἰδοὺ ἥκει Ἰωακεὶμ μετὰ τῶν ποιμνίων αὐτοῦ. καὶ ἔστη Ἄννα πρὸς τῇ πύλῃ τοῦ οἴκου αὐτῆς καὶ εἶδεν τὸν Ἰωακεὶμ ἐρχόμενον μετὰ τῶν ποιμνίων αὐτοῦ. καὶ ἔδραμεν Ἄννα καὶ ἐκρεμάσθη ἐπὶ τὸν τράχηλον αὐτοῦ λέγουσα: νῦν οἶδα, ὅτι κύριος ὁ θεὸς εὐλόγησέ με σφόδρα: ἰδοὺ γὰρ ἡ χήρα οὐκέτι χήρα καὶ ἡ ἄτεκνος ἰδοὺ ἐν γαστρὶ λήψομαι εἴληφα . καὶ ἀνεπαύσατο Ἰωακεὶμ τὴν πρώτην ἡμέραν εἰς τὸν οἶκον αὐτοῦ.

\pend\pstart
\eledsection*{Εʹ}

\pend\pstart
Τῇ δὲ ἐπαύριον προσέφερε τὰ δῶρα αὐτοῦ λέγων ἐν ἑαυτῷ: ἐὰν κύριος ὁ θεὸς ἱλασθῇ μοι, τὸ πέταλον τοῦ ἱερέως φανερών μοι ποιήσει. καὶ προσέφερεν τὰ δῶρα αὐτοῦ Ἰωακεὶμ καὶ προσεῖχε τῷ πετάλῳ τοῦ ἱερέως, ὡς ἐπέβη ἐπὶ τὸ θυσιαστήριον κυρίου, καὶ ἁμαρτία οὐχ εὑρέθη ἐν αὐτῷ. καὶ εἶπεν Ἰωακείμ: νῦν οἶδα, ὅτι κύριος ὁ θεὸς ἱλάσθη μοι καὶ ἀφεῖλέν μου πάντα τὰ ἁμαρτήματα. καὶ κατέβη ἐκ ναοῦ κυρίου δεδικαιωμένος καὶ ἀπῆλθεν εἰς τὸν οἶκον αὐτοῦ χαίρων καὶ δοξάζων τὸν θεόν.

\pend\pstart
ἐπληρώθησαν δὲ οἱ μῆνες αὐτῆς. ἐν δὲ τῷ ἐνάτῳ μηνὶ ἐγέννησεν Ἄννα καὶ εἶπεν τῇ μαίᾳ: τί ἐγέννησα; ἡ δὲ εἶπεν: θῆλυ. καὶ εἶπεν Ἄννα: ἐμεγάλυνεν ἡ ψυχή μου τὴν ἡμέραν ταύτην καὶ ἀνέκλινεν αὐτήν. πληρωθεισῶν δὲ τῶν ἡμερῶν ἀπεσμήξατο Ἄννα καὶ ἔδωκεν μασθὸν τῇ παιδί. ἐκάλεσεν δὲ τὸ ὄνομα αὐτῆς Μαριάμ.

\pend\pstart
% \eledsection*{Ϛʹ}
\eledsection*{Ϝʹ}

\pend\pstart
Ἡμέρᾳ δὲ καὶ ἡμέρᾳ ἐκραταιοῦτο ἡ παῖς. γενομένης δὲ αὐτῆς ἑξαμήνου ἔστησεν αὐτὴν ἡ μήτηρ αὐτῆς χαμαὶ τοῦ πειράσαι, εἰ ἵσταται: καὶ περιπατήσασα ἑπτὰ βήματα ἦλθεν εἰς τὸν κόλπον τῆς μητρὸς αὐτῆς, καὶ ἀνήρπασεν αὐτὴν ἡ μήτηρ αὐτῆς λέγουσα: ζῇ κύριος ὁ θεός μου: οὐ μὴ περιπατήσῃς ἐν τῇ γῇ ταύτῃ, ἕως οὗ ἀπάξω σε ἐν τῷ ναῷ κυρίου. καὶ ἐποίησεν ἁγίασμα ἐν τῷ κοιτῶνι αὐτῆς καὶ πᾶν κοινὸν ἤ ἀκάθαρτον οὐκ εἴα διέρχεσθαι δι' αὐτῆς. καὶ ἐκάλεσε τὰς θυγατέρας τῶν Ἑβραίων τὰς ἀμιάντους, καὶ διεπλάνων αὐτήν.

\pend\pstart
ἐγένετο δὲ πρῶτος ἐνιαυτὸς τῇ παιδί, καὶ ἐποίησεν Ἰωακεὶμ δοχὴν μεγάλην καὶ ἐκάλεσεν τοὺς ἱερεῖς καὶ τοὺς γραμματεῖς καὶ τὴν γερουσίαν καὶ πάντα τὸν λαὸν Ἰσραήλ. καὶ προσήνεγκεν Ἰωακεὶμ τὴν παῖδα τοῖς ἱερεῦσι καὶ εὐλόγησαν αὐτὴν οἱ ἱερεῖς λέγοντες: ὁ θεὸς τῶν πατέρων ἡμῶν, εὐλόγησον τὴν παῖδα ταύτην καὶ δὸς αὐτῇ ὄνομα ὀνομαστὸν αἰώνιον ἐν πάσαις ταῖς γενεαῖς. καὶ εἶπεν ὁ λαός: γένοιτο, γένοιτο, ἀμήν. καὶ προσήνεγκεν Ἰωακεὶμ τὴν παῖδα τοῖς ἀρχιερεῦσι, καὶ εὐλόγ\-ησαν αὐτὴν λέγοντες: ὁ θεὸς τῶν ὑψωμάτων, ἐπίβλεψον ἐπὶ τὴν παῖδα ταύτην καὶ εὐλόγησον αὐτὴν ἐσχάτην εὐλογίαν, ἥτις διαδοχὴν οὐχ ἕξει.

\pend\pstart
καὶ ἀπήγαγον αὐτὴν ἐν τῷ ἁγιάσματι τοῦ κοιτῶνος αὐτῆς: καὶ λαβοῦσα Ἄννα ἔδωκε μασθὸν τῇ παιδὶ καὶ ᾖσεν ᾆσμα κυρίῳ τῷ θεῷ λέγουσα: ᾄσω ὠδὴν κυρίῳ τῷ θεῷ μου, ὅτι ἐπεσκέψατό με καὶ ἀφεῖλεν ἀπ' ἐμοῦ τὸν ὀνειδισμὸν τῶν ἐχθρῶν μου καὶ ἔδωκέ μοι καρπὸν δικαιοσύνης μονοούσιον αὐτῷ καὶ πολυπλούσιον. τίς ἀναγγελεῖ τοῖς υἱοῖς Ῥουβίμ , ὅτι Ἄννα θηλάζει; καὶ ἀνέπαυσεν αὐτὴν ἡ μήτηρ αὐτῆς ἐν τῷ ἁγιάσματι τοῦ κοιτῶνος αὐτῆς καὶ ἐξῆλθε καὶ διηκόνει αὐτοῖς. τελεσ\-θέντος δὲ τοῦ δείπνου κατέβησαν εὐφραινόμενοι καὶ ἐδόξασαν τὸν θεὸν Ἰσραήλ.

\pend\pstart
\eledsection*{Ζʹ}

\pend\pstart
Τῇ δὲ παιδὶ προσετίθεντο οἱ μῆνες αὐτῆς. ἐγένετο δὲ διετὴς ἡ παῖς, καὶ εἶπεν Ἰωακείμ: ἀπάξωμεν αὐτὴν ἐν τῷ ναῷ κυρίου καὶ ἀποδῶμεν τὴν ἐπαγγελίαν, ἥν ἐπηγγειλάμεθα, μήπως ἀποστείλῃ κύριος ὁ θεὸς πρὸς ἡμᾶς καὶ γένηται ἀπρόσδεκτον τὸ δῶρον ἡμῶν. καὶ εἶπεν Ἄννα: ἀναμείνωμεν τὸ τρίτον ἔτος, ὅπως μὴ ζητήσῃ πατέρα ἤ μητέρα. καὶ εἶπεν Ἰωακείμ: ἀμήν, γένοιτο.

\pend\pstart
ἐγένετο δὲ τριετὴς ἡ παῖς, καὶ εἶπεν Ἰωακείμ: καλέσωμεν τὰς θυγατέρας τῶν Ἑβραίων τὰς ἀμιάντους, καὶ λαβέτωσαν ἀνὰ λαμπάδα, καὶ ἔστωσαν καιόμεναι, ἵνα μὴ ἐπιστραφῇ ἡ παῖς εἰς τὰ ὀπίσω καὶ αἰχμαλωτισθῇ ἡ καρδία αὐτῆς ἐκ ναοῦ κυρίου. καὶ ἐποίησαν οὕτως, ἕως οὗ ἀνέβησαν ἐν τῷ ναῷ κυρίου. καὶ ἐδέξατο αὐτὴν ὁ ἱερεὺς καὶ καταφιλήσας εὐλόγησε καὶ εἶπεν: ἐμεγάλυνε κύριος ὁ θεὸς τὸ ὄνομά σου ἐν πάσαις ταῖς γενεαῖς τῆς γῆς: (ἐπὶ σοὶ) ἐπ' ἐσχάτου τῶν ἡμερῶν φανερώσει κύριος ὁ θεὸς τὸ λύτρον τῶν υἱῶν Ἰσραήλ.

\pend\pstart
καὶ ἐκάθισεν αὐτὴν ἐπὶ τρίτου βαθμοῦ τοῦ θυσιαστηρίου, καὶ ἔβαλε κύριος ὁ θεὸς χάριν ἐπ' αὐτήν, καὶ κατεχόρευσε τοῖς ποσὶν αὐτοῖς, καὶ ἠγάπησεν αὐτὴν πᾶς οἶκος Ἰσραήλ.

\pend\pstart
\eledsection*{Ηʹ}

\pend\pstart
κατέβησαν δὲ οἱ γονεῖς αὐτῆς θαυμάζοντες καὶ ἐπαινοῦντες τὸν θεόν, ὅτι οὐκ ἐπεστράφη ἡ παῖς εἰς τὰ ὀπίσω. ἦν δὲ Μαριὰμ ὡσεὶ περιστερὰ νεμομένη ἐν τῷ ναῷ κυρίου καὶ ἐλάμβανε τροφὴν ἐκ χειρὸς ἀγγέλου.

\pend\pstart
γενομένης δὲ αὐτῆς δωδεκαετοῦς συμβούλιον ἐγένετο τῶν ἱερέων λεγόντων: ἰδοὺ Μαριὰμ γέγονε δωδεκαέτης ἐν τῷ ναῷ κυρίου: τί οὖν ποιήσωμεν αὐτήν, μήπως (ἐπέλθῃ αὐτῇ τὰ γυναικῶν καὶ) μιάνῃ τὸ ἁγίασμα κυρίου. καὶ εἶπον τῷ ἀρχιερεῖ: σὺ ἕστηκας ἐπὶ τὸ θυσιαστήριον θεοῦ: εἴσελθε καὶ πρόσευξαι περὶ αὐτῆς, καὶ ὅ ἄν φανερώσῃ σοι κύριος ὁ θεός, τοῦτο ποιήσωμεν.

\pend\pstart
καὶ εἰσῆλθεν ὁ ἱερεὺς λαβὼν τὸν δωδεκα\-κόδωνα (ἱεροπρεπὲς ἱμάτιον) εἰς τὰ ἅγια τῶν ἁγίων καὶ ηὔξατο περὶ αὐτῆς. καὶ ἰδοὺ ἄγγελος κυρίου ἐπέστη αὐτῷ λέγων: Ζαχαρία, Ζαχαρία, ἔξελθε καὶ ἐκκλησίασον τοὺς χηρεύοντας τοῦ λαοῦ, καὶ ἐνεγκάτωσαν ἀνὰ ῥάβδον, καὶ εἰς ὅν ἐὰν δείξῃ κύριος ὁ θεὸς σημεῖον, τούτου ἔσται γυνή. καὶ ἐξῆλθον οἱ κήρυκες καθ' ὅλης τῆς περιχώρου τῆς Ἰουδαίας, καὶ ἤχησεν ἡ σάλπιγξ κυρίου, καὶ ἔδραμον πάντες.

\pend\pstart
\eledsection*{Θʹ}

\pend\pstart
Ἰωσὴφ δὲ ῥίψας τὸ σκέπαρνον ἔδραμε καὶ αὐτὸς εἰς τὴν συναγωγήν, καὶ συναχθέντες ὁμοῦ ἀπῆλθαν πρὸς τὸν ἱερέα. ἔλαβε δὲ πάντων τὰς ῥάβδους ὁ ἱερεὺς καὶ εἰσῆλθεν εἰς τὸ ἱερὸν καὶ ηὔξατο. τελέσας δὲ τὴν εὐχὴν ἐξῆλθε καὶ ἐπέδωκεν ἑνὶ ἑκάστῳ τὴν ἑαυτοῦ ῥάβδον, καὶ σημεῖον οὐκ ἦν ἐν αὐτοῖς. τὴν δὲ ἐσχάτην ῥάβδον ἔλαβεν ὁ Ἰωσήφ, καὶ ἰδοὺ περιστερὰ ἐξῆλθεν ἐκ τῆς ῥάβδου καὶ ἐπετάσθη ἐπὶ τὴν κεφαλὴν Ἰωσήφ. καὶ εἶπεν αὐτῷ ὁ ἱερεύς: σὺ κεκλήρωσαι τὴν παρθένον κυρίου παραλαβεῖν. παράλαβε αὐτὴν εἰς τήρησιν σεαυτῷ.

\pend\pstart
ἀντεῖπε δὲ Ἰωσὴφ λέγων: υἱοὺς ἔχω καὶ πρεσβύτης εἰμί, αὕτη δὲ νεωτέρα. μήπως κατάγελως γένωμαι τοῖς υἱοῖς Ἰσραήλ; εἶπεν δὲ αὐτῷ ὁ ἱερεύς: Ἰωσήφ, φοβήθητι κύριον τὸν θεὸν καὶ ὅσα ἐποίησε Δαθὰμ καὶ Κορὲ καὶ Ἀβηρών, πῶς ἐδιχάσθη ἡ γῆ καὶ κατε\-ποντίσθησαν ἅπαντες διὰ τὴν ἀντιλογίαν αὐτῶν. καὶ νῦν φοβήθητι, Ἰωσήφ, μήπως ἔσται ταῦτα ἐν τῷ οἴκῳ σου.

\pend\pstart
καὶ φοβηθεὶς Ἰωσὴφ παρέλαβεν αὐτὴν εἰς τήρησιν. καὶ εἶπεν αὐτῇ: Μαρία, ἰδοὺ παρέλαβόν σε ἐκ ναοῦ κυρίου τοῦ θεοῦ μου καὶ νῦν καταλιμπάνω σε ἐν τῷ οἴκῳ μου, ἀπέρχομαι γὰρ οἰκοδομῆσαι τὰς οἰκοδομάς μου, καὶ ἐν τάχει ἥξω πρὸς σέ. κύριος ὁ θεὸς διαφυλάξει σε.

\pend\pstart
\eledsection*{Ιʹ}

\pend\pstart
Ἐγένετο δὲ συμβούλιον τῶν ἱερέων λεγόντων: ποιήσωμεν καταπέτασμα τῷ ναῷ κυρίου. καὶ εἶπεν ὁ ἱερεύς: καλέσατέ μοι ὧδε ἑπτὰ παρθένους ἀμιάντους ἐκ φυλῆς Δαυίδ. καὶ ἀπῆλθον οἱ ὑπηρέται καὶ εὕρησαν ἑπτά (εὗρον ἕξ). καὶ ἐμνήσθη ὁ ἱερεύς, ὅτι Μαρία ἐκ φυλῆς Δαυίδ ἐστι καὶ ἀμίαντός ἐστιν. καὶ ἀπῆλθαν οἱ ὑπηρέται καὶ ἤγαγον αὐτήν. καὶ εἰσήγαγεν αὐτὰς ὁ ἱερεὺς ἐν τῷ ναῷ κυρίου καὶ εἶπεν: λάχετέ μοι ὧδε, τίς νήσει τὸ χρυσίον καὶ τὸ ἀμίαντον καὶ τὸ βύσσινον καὶ τὸ σηρικοῦν καὶ τὸ ὑάκινθον καὶ τὸ κόκκινον καὶ τὴν ἀληθινὴν πορφύραν. καὶ ἔλαχεν τὴν Μαριὰμ τὸ κόκκινον καὶ ἡ ἀληθινὴ πορφύρα. καὶ λαβοῦσα ἀπῆλθεν εἰς τὸν οἶκον αὐτῆς. τῷ δὲ καιρῷ ἐκείνῳ Ζαχαρίας ἐσίγησεν. Μαριὰμ δὲ λαβοῦσα τὸ κόκκινον ἔκλωσεν.

\pend\pstart
\eledsection*{ΙΑʹ}

\pend\pstart
Καὶ λαβοῦσα κάλπιν ἐξῆλθεν γεμίσαι ὕδωρ, καὶ ἰδοὺ φωνὴ λέγουσα: χαῖρε κεχαριτωμένη, ὁ κύριος μετὰ σοῦ, εὐλογημένη σὺ ἐν γυναιξί. καὶ περιεβλέπετο δεξιὰ καὶ ἀριστερά, πόθεν αὕτη ἡ φωνὴ ὑπάρχει, καὶ ἔντρομος γενομένη ἀπῆλθεν εἰς τὸν οἶκον αὐτῆς. καὶ ἀναπαύσασα τὴν κάλπην ἔλαβε πάλιν τὴν πορφύραν καὶ ἐκάθισεν ἐπὶ τὸν θρόνον καὶ εἷλκεν αὐτήν.

\pend\pstart
καὶ ἰδοὺ ἄγγελος κυρίου ἐπέστη λέγων αὐτῇ: μὴ φοβοῦ, Μαριάμ, εὗρες γὰρ χάριν ἐνώπιον τοῦ θεοῦ καὶ συλλήψῃ ἐκ λόγου αὐτοῦ. ἀκούσασα δὲ Μαριὰμ διεκρίθη ἐν ἑαυτῇ λέγουσα: ἐγὼ συλλήψομαι, ὡς πᾶσα γυνὴ γεννᾷ;

\pend\pstart
καὶ λέγει πρὸς αὐτὴν ὁ ἄγγελος: οὐχ οὕτως, Μαριάμ: δύναμις γὰρ θεοῦ ἐπισκιάσει σοι, διὸ καὶ τὸ γεννόμενον (ἐκ σοῦ) ἅγιον κληθήσεται υἱὸς ὑψίστου, καὶ καλέσεις τὸ ὄνομα αὐτοῦ Ἰησοῦν: αὐτὸς γὰρ σώσει τὸν λαὸν αὐτοῦ ἀπὸ τῶν ἁμαρτιῶν αὐτῶν. καὶ εἶπεν Μαριάμ: ἰδοὺ ἡ δούλη κυρίου: γένοιτό μοι κατὰ τὸ ῥῆμά σου.

\pend\pstart
\eledsection*{ΙΒʹ}

\pend\pstart
Καὶ ἐποίησεν τὴν πορφύραν καὶ τὸ κόκκινον καὶ ἀπήνεγκεν αὐτὰ τῷ ἱερεῖ, καὶ εὐλόγησεν αὐτὴν ὁ ἱερεὺς καὶ εἶπεν: Μαριάμ, ἐμεγάλυνε κύριος ὁ θεὸς τὸ ὄνομά σου ἐν πάσαις ταῖς γενεαῖς τῆς γῆς καὶ ἔσῃ εὐλογημένη ὑπὸ κυρίου.

\pend\pstart
χαρὰν δὲ λαβοῦσα Μαριὰμ ἀπῆλθε πρὸς τὴν συγγενίδα αὐτῆς Ἐλισάβετ καὶ ἔκρουσε πρὸς τῇ θύρᾳ. καὶ ἀκούσασα Ἐλισάβετ ἔρριψε τὸ ἐν χερσὶν, καὶ δραμοῦσα ἤνοιξεν αὐτῇ καὶ εὐλόγησεν αὐτὴν καὶ εἶπεν: πόθεν μοι τοῦτο, ἵνα ἡ μήτηρ τοῦ κυρίου μου ἔλθῃ πρὸς ἐμέ; ἰδοὺ γὰρ τὸ ἐν ἐμοὶ βρέφος ἐσκίρτησε καὶ εὐλόγησέν σε. Μαριὰμ δὲ ἐπελάθετο τῶν μυστηρίων, ὧν εἶπεν πρὸς αὐτὴν Γαβριήλ, καὶ ἀτενίσασα εἰς τὸν οὐρανὸν εἶπεν: τίς εἰμι ἐγώ, ὅτι πᾶσαι αἱ γυναῖκες μακαριοῦσί με;

\pend\pstart
ἐποίησε δὲ τρεῖς μῆνας πρὸς τὴν Ἐλισάβετ καὶ ἀπῆλθεν εἰς τὸν οἶκον αὐτῆς. ἡμέρᾳ δὲ ἀφ' ἡμέρας ἡ γαστὴρ αὐτῆς ὀγκοῦτο, καὶ ἔκρυβεν ἑαυτὴν ἀπὸ τῶν υἱῶν Ἰσραήλ. ἦν δὲ ἐτῶν πεντεκαίδεκα, ὅτε τὰ μυστήρια ταῦτα ἐγένοντο.

\pend\pstart
\eledsection*{ΙΓʹ}

\pend\pstart
Ἐγένετο δὲ ἕκτος μὴν καὶ ἦλθεν Ἰωσὴφ ἀπὸ τῶν οἰκοδομῶν αὐτοῦ καὶ εἰσῆλθεν ἐν τῷ οἴκῳ αὐτοῦ καὶ εὗρε τὴν Μαριὰμ ὀγκωμένην. καὶ ἔτυψε τὸ πρόσωπον αὐτοῦ καὶ ἔρριψεν ἑαυτὸν χαμαὶ καὶ ἔκλαυσε λέγων: ποίῳ προσόπῳ ἀτενίσω πρὸς κύριον τὸν θεόν μου; τί δὴ εἴπω περὶ τῆς κόρης ταύτης, ὅτι παρθένον αὐτὴν παρέλαβον ἐκ ναοῦ κυρίου καὶ οὐκ ἐφύλαξα αὐτήν; τίς ὁ θηρεύσας με; τίς τὸ πονηρὸν τοῦτο ἐποίησεν ἐν τῷ οἴκῳ μου καὶ ἐμίανεν τὴν παρθένον; μήτι εἰς ἐμὲ ἀνεκεφαλαιόθη ἡ ἱστορία Ἀδάμ; ὥσπερ γὰρ Ἀδὰμ ἦν ἐν τῇ ὥρᾳ τῆς δοξολογίας αὐτοῦ καὶ ἦλθεν ὁ ὄφις καὶ εὗρεν τὴν Εὔαν μόνην καὶ ἐξηπάτησεν αὐτήν, οὕτως κἀμοί συνέβη.

\pend\pstart
καὶ ἀνέστη Ἰωσὴφ ἀπὸ τοῦ σάκκου καὶ ἐκάλεσε τὴν Μαριὰμ καὶ εἶπεν αὐτῇ: μεμελημένη τῷ θεῷ, τί τοῦτο ἐποίησας; τί ἐταπείνωσας τὴν ψυχήν σου; ἐπελάθου κυρίου τοῦ θεοῦ σου, ἡ ἀνατραφεῖσα εἰς τὰ ἅγια τῶν ἁγίων καὶ λαβοῦσα τροφὴν ἐκ χειρὸς ἀγγέλου καὶ χορεύσασα ἐν αὐτοῖς;

\pend\pstart
ἡ δὲ ἔκλαυσε πικρῶς λέγουσα: ζῇ κύριος ὁ θεός, καθότι καθαρά εἰμι ἐγὼ καὶ ἄνδρα οὐ γινώσκω. εἶπε δὲ αὐτῇ Ἰωσήφ: πόθεν οὖν ἐστι τοῦτο ἐν τῇ γαστρί σου; εἶπε δὲ αὐτῷ: ζῇ κύριος ὁ θεός μου, καθότι οὐ γινώσκω, πόθεν ἐστὶ τοῦτο τὸ ἐν τῇ γαστρί μου.

\pend\pstart
\eledsection*{ΙΔʹ}

\pend


}

\endnumbering
\end{Leftside}

\begin{Rightside}
\beginnumbering
\numberpstarttrue

\pstart
\eledsection*{I}

\pend\setcounter{pstartR}{1}\pstart
In the records of the twelve tribes of Israel was Joachim, a man rich exceedingly; and he brought his offerings double,\footnote{Susanna i. 4.} saying: There shall be of my superabundance to all the people, and there shall be the offering for my forgiveness\footnote{The readings vary, and the sense is doubtful. Thilo thinks that the sense is: What I offer over and above what the law requires is for the benefit of the whole people; but the offering I make for my own forgiveness (according to the law's requirements) shall be to the Lord, that He may be rendered merciful to me.} to the Lord for a propitiation\footnote{The Church of Rome appoints March 20 as the Feast of St. Joachim. His liberality is commemorated in prayers, and the lessons to be read are Wisd. xxxi. and Matt. i.} for me.\footnote{Translated by Alexander Walker, Esq., ``One of Her Majesty’s Inspectors of Schools for Scotland'' From \textit{Ante-Nicene Fathers}, Vol. VIII. Edited by\\Alexander Roberts, James Donaldson, and A. Cleveland Coxe. Published by Wm. B. Eerdmans Publishing Company, Grand Rapids, Michigan, 1885.\\Retrieved from \textit{Wikisource}, \texttt{\mbox{https://en.wikisource.org/wiki/Ante-Nicene\_Fathers/Volume\_VIII/Apocrypha\_of\_the\_New\_Testament/}\\The\_Protevangelium\_of\_James} (accessed 21 February, 2020).}

\pend\pstart
For the great day of the Lord was at hand, and the sons of Israel were bringing their offerings. And there stood over against him Rubim, saying: It is not meet for thee first to bring thine offerings, because thou hast not made seed in Israel.\footnote{1 Sam. i. 6, 7; Hos. ix. 14.}

\pend\pstart
And he searched, and found that all the righteous had raised up seed in Israel. And he called to mind the patriarch Abraham, that in the last day\footnote{Another reading is: In his last days.} God gave him a son Isaac.

\pend\pstart
And Joachim was exceedingly grieved, and did not come into the presence of his wife; but he retired to the desert,\footnote{Another reading is: Into the hill-country.} and there pitched his tent, and fasted forty days and forty nights,\footnote{Moses: Ex. xxiv. 18, xxxiv. 28; Deut. ix. 9. Elijah: 1 Kings xix. 8. Christ: Matt. iv. 2.} saying in himself: I will not go down either for food or for drink until the Lord my God shall look upon me, and prayer shall be my food and drink.

\pend\pstart
\eledsection*{II}

\pend\setcounter{pstartR}{1}\pstart
And his wife Anna\footnote{The 26th day of July is the Feast of St. Anna in the Church of Rome.} mourned in two mournings, and lamented in two lamentations, saying: I shall bewail my widowhood; I shall bewail my childlessness.

\pend\pstart
And the great day of the Lord was at hand; and Judith\footnote{Other forms of the name are Juth, Juthin.} her maid-servant said: How long dost thou humiliate thy soul? Behold, the great day of the Lord is at hand, and it is unlawful for thee to mourn. But take this head-band, which the woman that made it gave to me; for it is not proper that I should wear it, because I am a maid-servant, and it has a royal appearance.\footnote{Some MSS. have: For I am thy maid-servant, and thou hast a regal appearance.}

\pend\pstart
And Anna said: Depart from me; for I have not done such things, and the Lord has brought me very low. I fear that some wicked person has given it to thee, and thou hast come to make me a sharer in thy sin. And Judith said: Why should I curse thee, seeing that\footnote{Several MSS. insert: Thou hast not listened to my voice; for.} the Lord hath shut thy womb, so as not to give thee fruit in Israel?

\pend\pstart
And Anna was grieved exceedingly, and put off her garments of mourning, and cleaned her head, and put on her wedding garments, and about the ninth hour went down to the garden to walk. And she saw a laurel, and sat under it, and prayed to the Lord, saying: O God of our fathers, bless me and hear my prayer, as Thou didst bless the womb of Sarah, and didst give her a son Isaac.\footnote{Comp. 1 Sam. i. 9\textendash 18.}

\pend\pstart
\eledsection*{III}

\pend\setcounter{pstartR}{1}\pstart
And gazing towards the heaven, she saw a sparrow's nest in the laurel,\footnote{Tobit ii. 10.} and made a lamentation in herself, saying: Alas! who begot me? and what womb produced me? because I have become a curse in the presence of the sons of Israel, and I have been reproached, and they have driven me in derision out of the temple of the Lord.

\pend\pstart
Alas! to what have I been likened? I am not like the fowls of the heaven, because even the fowls of the heaven are productive before Thee, O Lord. Alas! to what have I been likened? I am not like the beasts of the earth, because even the beasts of the earth are productive before Thee, O Lord.

\pend\pstart
Alas! to what have I been likened? I am not like these waters, because even these waters are productive before Thee, O Lord. Alas! to what have I been likened? I am not like this earth, because even the earth bringeth forth its fruits in season, and blesseth Thee, O Lord.\footnote{Many of the MSS. here add: Alas! to what have I been likened? I am not like the waves of the sea, because even the waves of the sea, in calm and storm, and the fishes in them, bless Thee, O Lord.}

\pend\pstart
\eledsection*{IV}

\pend\setcounter{pstartR}{1}\pstart
And, behold, an angel of the Lord stood by, saying: Anna, Anna, the Lord hath heard thy prayer, and thou shalt conceive, and shall bring forth; and thy seed shall be spoken of in all the world. And Anna said: As the Lord my God liveth, if I beget either male or female, I will bring it as a gift to the Lord my God; and it shall minister to Him in holy things all the days of its life.\footnote{1 Sam. i. 11.}

\pend\pstart
And, behold, two angels came, saying to her: Behold, Joachim thy husband is coming with his
flocks.\footnote{One of the MSS.: With his shepherds, and sheep, and goats, and oxen.} For an angel of the Lord went down to him, saying: Joachim, Joachim, the Lord God hath heard thy prayer. Go down hence; for, behold, thy wife Anna shall conceive.

\pend\pstart
And Joachim went down and called his shepherds, saying: Bring me hither ten she-lambs without spot or blemish, and they shall be for the Lord my God; and bring me twelve tender calves, and they shall be for the priests and the elders; and a hundred goats for all the people.

\pend\pstart
And, behold, Joachim came with his flocks; and Anna stood by the gate, and saw Joachim coming, and she ran and hung upon his neck, saying: Now I know that the Lord God hath blessed me exceedingly; for, behold the widow no longer a widow, and I the childless shall conceive. And Joachim rested the first day in his house.

\pend\pstart
\eledsection*{V}

\pend\setcounter{pstartR}{1}\pstart
And on the following day he brought his offerings, saying in himself: If the Lord God has been rendered gracious to me, the plate\footnote{Ex. xxviii. 36\textendash 38. For traditions about the \textit{petalon}, see Euseb., \textit{H. E.}, ii. 23, iii. 31, v. 24; Epiph., \textit{Hær.}, 78.} on the priest's forehead will make it manifest to me. And Joachim brought his offerings, and observed attentively the priest's plate when he went up to the altar of the Lord, and he saw no sin in himself. And Joachim said: Now I know that the Lord has been gracious unto me, and has remitted all my sins. And he went down from the temple of the Lord justified, and departed to his own house.

\pend\pstart
And her months were fulfilled, and in the ninth\footnote{Various readings are: Sixth, seventh, eighth.} month Anna brought forth. And she said to the midwife: What have I brought forth? and she said: A girl. And said Anna: My soul has been magnified this day. And she laid her down. And the days having been fulfilled, Anna was purified, and gave the breast to the child,\footnote{One of the MSS. inserts: On the eighth day.} and called her name Mary.

\pend\pstart
\eledsection*{VI}

\pend\setcounter{pstartR}{1}\pstart
And the child grew strong day by day; and when she was six\footnote{One of the MSS. has nine.} months old, her mother set her on the ground to try whether she could stand, and she walked seven steps and came into her bosom; and she snatched her up, saying: As the Lord my God liveth, thou shalt not walk on this earth until I bring thee into the temple of the Lord. And she made a sanctuary in her bed-chamber, and allowed nothing common or unclean to pass through her. And she called the undefiled daughters of the Hebrews, and they led her astray.\footnote{This is the reading of most MSS.; but it is difficult to see any sense in it. One MS. reads: They attended on her. Fabricius proposed: They bathed her.}

\pend\pstart
And when she was a year old, Joachim made a great feast, and invited the priests, and the scribes, and the elders, and all the people of Israel. And Joachim brought the child to the priests; and they blessed her, saying: O God of our fathers, bless this child, and give her an everlasting name to be named in all generations. And all the people said: So be it, so be it, amen. And he brought her to the chief priests; and they blessed her, saying: O God most high, look upon this child, and bless her with the utmost blessing, which shall be for ever.

\pend\pstart
And her mother snatched her up, and took her into the sanctuary of her bed-chamber, and gave her the breast. And Anna made a song to the Lord God, saying: I will sing a song to the Lord my God, for He hath looked upon me, and hath taken away the reproach of mine enemies; and the Lord hath given the fruit of His righteousness, singular in its kind, and richly endowed before Him. Who will tell the sons of Rubim that Anna gives suck? Hear, hear, ye twelve tribes of Israel, that Anna gives suck. And she laid her to rest in the bed-chamber of her sanctuary, and went out and ministered unto them. And when the supper was ended, they went down rejoicing, and glorifying the God of Israel.\footnote{Two of the MSS. add: And they gave her the name of Mary, because her name shall not fade forever. This derivation of the name\textemdash from the root \textit{mar}, fade\textemdash is one of a dozen or so.}

\pend\pstart
\eledsection*{VII}

\pend\setcounter{pstartR}{1}\pstart
And her months were added to the child. And the child was two years old, and Joachim said: Let us take her up to the temple of the Lord, that we may pay the vow that we have vowed, lest perchance the Lord send to us,\footnote{This is taken to mean: Send someone to us to warn us that we have been too long in paying our vow. One MS. reads, lest the Lord depart from us; another, lest the Lord move away from us.} and our offering be not received. And Anna said: Let us wait for the third year, in order that the child may not seek for father or mother. And Joachim said: So let us wait.

\pend\pstart
And the child was three years old, and Joachim said: Invite the daughters of the Hebrews that are undefiled, and let them take each a lamp, and let them stand with the lamps burning, that the child may not turn back, and her heart be captivated from the temple of the Lord. And they did so until they went up into the temple of the Lord. And the priest received her, and kissed her, and blessed her, saying: The Lord has magnified thy name in all generations. In thee, on the last of the days, the Lord will manifest His redemption to the sons of Israel.

\pend\pstart
And he set her down upon the third step of the altar, and the Lord God sent grace upon her; and she danced with her feet, and all the house of Israel loved her.

\pend\pstart
\eledsection*{VIII}

\pend\setcounter{pstartR}{1}\pstart
And her parents went down marvelling, and praising the Lord God, because the child had not turned back. And Mary was in the temple of the Lord as if she were a dove that dwelt there, and she received food from the hand of an angel.

\pend\pstart
And when she was twelve\footnote{Or, fourteen. Postel's Latin version has \textit{ten}.} years old there was held a council of the priests, saying: Behold, Mary has reached the age of twelve years in the temple of the Lord. What then shall we do with her, lest perchance she defile the sanctuary of the Lord? And they said to the high priest: Thou standest by the altar of the Lord; go in, and pray concerning her; and whatever the Lord shall manifest unto thee, that also will we do.

\pend\pstart
And the high priest went in, taking the robe\footnote{Ex. xxviii. 28; Sirach xlv. 9; Justin, \textit{Tryph}., xlii.} with the twelve bells into the holy of holies; and he prayed concerning her. And behold an angel of the Lord stood by him, saying unto him: Zacharias, Zacharias, go out and assemble the widowers of the people, and let them bring each his rod; and to whomsoever the Lord shall show a sign, his wife shall she be. And the heralds went out through all the circuit of Judæa, and the trumpet of the Lord sounded, and all ran.

\pend\pstart
\eledsection*{IX}

\pend\setcounter{pstartR}{1}\pstart
And Joseph, throwing away his axe, went out to meet them; and when they had assembled, they went away to the high priest, taking with them their rods. And he, taking the rods of all of them, entered into the temple, and prayed; and having ended his prayer, he took the rods and came out, and gave them to them: but there was no sign in them, and Joseph took his rod last; and, behold, a dove came out of the rod, and flew upon Joseph's head. And the priest said to Joseph, Thou hast been chosen by lot to take into thy keeping the virgin of the Lord.

\pend\pstart
But Joseph refused, saying: I have children, and I am an old man, and she is a young girl. I am afraid lest I become a laughing-stock to the sons of Israel. And the priest said to Joseph: Fear the Lord thy God, and remember what the Lord did to Dathan, and Abiram, and Korah;\footnote{Num. xvi. 31\textendash 33.} how the earth opened, and they were swallowed up on account of their contradiction. And now fear, O Joseph, lest the same things happen in thy house.

\pend\pstart
And Joseph was afraid, and took her into his keeping. And Joseph said to Mary: Behold, I have received thee from the temple of the Lord; and now I leave thee in my house, and go away to build my buildings, and I shall come to thee. The Lord will protect thee.

\pend\pstart
\eledsection*{X}

\pend\setcounter{pstartR}{1}\pstart
And there was a council of the priests, saying: Let us make a veil for the temple of the Lord. And the priest said: Call to me the undefiled virgins of the family of David. And the officers went away, and sought, and found seven virgins. And the priest remembered the child Mary, that she was of the family of David, and undefiled before God. And the officers went away and brought her. And they brought them into the temple of the Lord. And the priest said: Choose for me by lot who shall spin the gold, and the white,\footnote{Lit., undefiled. It is difficult to say what colour is meant, or if it is a colour at all. The word is once used to mean the sea, but with no reference to colour. It is also the name of a stone of a greenish hue.} and the fine linen, and the silk, and the blue,\footnote{Lit., hyacinth.} and the scarlet, and the true purple.\footnote{Ex. xxv. 4.} And the true purple and the scarlet fell to the lot of Mary, and she took them, and went away to her house. And at that time Zacharias was dumb, and Samuel was in his place until the time that Zacharias spake. And Mary took the scarlet, and span it.

\pend\pstart
\eledsection*{XI}

\pend\setcounter{pstartR}{1}\pstart
And she took the pitcher, and went out to fill it with water. And, behold, a voice saying: Hail, thou who hast received grace; the Lord is with thee; blessed art thou among women!\footnote{Luke i. 28.} And she looked round, on the right hand and on the left, to see whence this voice came. And she went away, trembling, to her house, and put down the pitcher; and taking the purple, she sat down on her seat, and drew it out.

\pend\pstart
And, behold, an angel of the Lord stood before her, saying: Fear not, Mary; for thou hast found grace before the Lord of all, and thou shalt conceive, according to His word. And she hearing, reasoned with herself, saying: Shall I conceive by the Lord, the living God? and shall I bring forth as every woman brings forth?

\pend\pstart
And the angel of the Lord said: Not so, Mary; for the power of the Lord shall overshadow thee: wherefore also that holy thing which shall be born of thee shall be called the Son of the Most High. And thou shalt call His name Jesus, for He shall save His people from their sins. And Mary said: Behold, the servant of the Lord before His face: let it be unto me according to thy word.

\pend\pstart
\eledsection*{XII}

\pend\setcounter{pstartR}{1}\pstart
And she made the purple and the scarlet, and took them to the priest. And the priest blessed her, and said: Mary, the Lord God hath magnified thy name, and thou shalt be blessed in all the generations of the earth.

\pend\pstart
And Mary, with great
joy, went away to Elizabeth her kinswoman,\footnote{Luke i. 39, 40.} and knocked at the door. And when Elizabeth heard her, she threw away the scarlet,\footnote{Other readings are: the wool\textemdash what she had in her hand.} and ran to the door, and opened it; and seeing Mary, she blessed her, and said: Whence is this to me, that the mother of my Lord should come to me? for, behold, that which is in me leaped and blessed thee.\footnote{Luke i. 34, 44.} But Mary had forgotten the mysteries of which the archangel Gabriel had spoken, and gazed up into heaven, and said: Who am I, O Lord, that all the generations of the earth should bless me?\footnote{Luke i. 48.}

\pend\pstart
And she remained three months with Elizabeth; and day by day she grew bigger. And Mary being afraid, went away to her own house, and hid herself from the sons of Israel. And she was sixteen\footnote{Six MSS. have \textit{sixteen}; one, \textit{fourteen}; two, \textit{fifteen}; and one, \textit{seventeen}.} years old when these mysteries happened.

\pend\pstart
\eledsection*{XIII}

\pend\setcounter{pstartR}{1}\pstart
And she was in her sixth month; and, behold, Joseph came back from his building, and, entering into his house, he discovered that she was big with child. And he smote\footnote{The Latin translation has \textit{hung down}.} his face,\footnote{Ezek. xxi. 12; Jer. xxxi. 19.} and threw himself on the ground upon the sackcloth, and wept bitterly, saying: With what face shall I look upon the Lord my God? and what prayer shall I make about this maiden? because I received her a virgin out of the temple of the Lord, and I have not watched over her. Who is it that has hunted me\footnote{Two MSS.: \textit{her}.} down? Who has done this evil thing in my house, and defiled the virgin? Has not the history of Adam been repeated in me? For just as Adam was in the hour of his singing praise,\footnote{Another reading is: As Adam was in Paradise, and in the hour of the singing of praise (doxology) to God was with the angels, the serpent, etc.} and the serpent came, and found Eve alone, and completely deceived her, so it has happened to me also.

\pend\pstart
And Joseph stood up from the sackcloth, and called Mary, and said to her: O thou who hast been cared for by God, why hast thou done this and forgotten the Lord thy God? Why hast thou brought low thy soul, thou that wast brought up in the holy of holies, and that didst receive food from the hand of an angel?

\pend\pstart
And she wept bitterly, saying: I am innocent, and have known no man. And Joseph said to her: Whence then is that which is in thy womb? And she said: As the Lord my God liveth, I do not know whence it is to me.

\pend\pstart
\eledsection*{XIV}

\pend\setcounter{pstartR}{1}\pstart
And Joseph was greatly afraid, and retired from her, and considered what he should do in regard to her.\footnote{Matt. i. 19.} And Joseph said: If I conceal her sin, I find myself fighting against the law of the Lord; and if I expose her to the sons of Israel, I am afraid lest that which is in her be from an angel,\footnote{Lit., \textit{angelic}; one MS. has \textit{holy}; the Latin translation, following a slightly different reading, \textit{that it would not be fair to her}.} and I shall be found giving up innocent blood to the doom of death. What then shall I do with her? I will put her away from me secretly. And night came upon him;

\pend\pstart
and, behold, an angel of the Lord appears to him in a dream, saying: Be not afraid for this maiden, for that which is in her is of the Holy Spirit; and she will bring forth a Son, and thou shalt call His name Jesus, for He will save His people from their sins.\footnote{Matt. i. 20.} And Joseph arose from sleep, and glorified the God of Israel, who had given him this grace; and he kept her.

\pend\pstart
\eledsection*{XV}

\pend\setcounter{pstartR}{1}\pstart
And Annas the scribe came to him, and said: Why hast thou not appeared in our assembly? And Joseph said to him: Because I was weary from my journey, and rested the first day. And he turned, and saw that Mary was with child.

\pend\pstart
And he ran away to the priest,\footnote{Three MSS. have \textit{high priest}.} and said to him: Joseph, whom thou didst vouch for, has committed a grievous crime. And the priest said: How so? And he said: He has defiled the virgin whom he received out of the temple of the Lord, and has married her by stealth, and has not revealed it to the sons of Israel. And the priest answering, said: Has Joseph done this? Then said Annas the scribe: Send officers, and thou wilt find the virgin with child. And the officers went away, and found it as he had said; and they brought her along with Joseph to the tribunal.

\pend\pstart
And the priest said: Mary, why hast thou done this? and why hast thou brought thy soul low, and forgotten the Lord thy God? Thou that wast reared in the holy of holies, and that didst receive food from the hand of an angel, and didst hear the hymns, and didst dance before Him, why hast thou done this? And she wept bitterly, saying: As the Lord my God liveth, I am pure before Him, and know not a man.

\pend\pstart
And the priest said to Joseph: Why hast thou done this? And Joseph said: As the Lord liveth, I am pure concerning her. Then said the priest: Bear not false witness, but speak the truth. Thou hast married her by stealth, and hast not revealed it to the sons of Israel, and hast not bowed thy head under the strong hand, that thy seed might be blessed. And Joseph was silent.

\pend\pstart
\eledsection*{XVI}

\pend\setcounter{pstartR}{1}\pstart
And the priest said: Give up the virgin whom thou didst receive out of the temple of the Lord. And Joseph burst into tears. And the priest said: I will give you to drink of the water of the ordeal of the Lord,\footnote{Num. v. 11, ff.} and He shall make manifest your sins in your eyes.

\pend\pstart
And the priest took the water, and gave Joseph to drink and sent him away to the hill-country; and he returned unhurt. And he gave to Mary also to drink, and sent her away to the hill-country; and she returned unhurt. And all the people wondered that sin did not appear in them.

\pend\pstart
And the priest said: If the Lord God has not made manifest your sins, neither do I judge you. And he sent them away. And Joseph took Mary, and went away to his own house, rejoicing and glorifying the God of Israel.

\pend\pstart
\eledsection*{XVII}

\pend\setcounter{pstartR}{1}\pstart
And there was an order from the Emperor Augustus, that all in Bethlehem of Judæa should be enrolled.\footnote{Luke ii. 1.} And Joseph said: I shall enrol my sons, but what shall I do with this maiden? How shall I enrol her? As my wife? I am ashamed. As my daughter then? But all the sons of Israel know that she is not my daughter. The day of the Lord shall itself bring it to pass\footnote{Or: On this day of the Lord I will do, etc.} as the Lord will.

\pend\pstart
And he saddled the ass, and set her upon it; and his son led it, and Joseph followed.\footnote{Another reading is: And his son Samuel led it, and James and Simon followed.} And when they had come within three miles, Joseph turned and saw her sorrowful; and he said to himself: Likely that which is in her distresses her. And again Joseph turned and saw her laughing. And he said to her: Mary, how is it that I see in thy face at one time laughter, at another sorrow? And Mary said to Joseph: Because I see two peoples with my eyes; the one weeping and lamenting, and the other rejoicing and exulting.

\pend\pstart
And they came into the middle of the road, and Mary said to him: Take me down from off the ass, for that which is in me presses to come forth. And he took her down from off the ass, and said to her: Whither shall I lead thee, and cover thy disgrace? for the place is desert.

\pend\pstart
\eledsection*{XVIII}

\pend\setcounter{pstartR}{1}\pstart
And he found a cave\footnote{Bethlehem…used to be overshadowed by a grove of Thammuz, i.e., Adonis; and in the cave where Christ formerly wailed as an infant, they used to mourn for the beloved of Venus (\textit{Jerome to Paulinus}). In his letter to Sabinianus the cave is repeatedly mentioned: ``That cave in which the Son of God was born;'' ``that venerable cave,'' etc., ``within the door of what was once the Lord's manger, now the altar.'' ``Then you run to the place of the shepherds.'' There appears also to have been above the altar the figure of an angel, or angels. See also Justin, \textit{Tryph}., 78.} there, and led her into it; and leaving his two sons beside her, he went out to seek a midwife in the district of Bethlehem.

\pend\pstart
And I Joseph was walking, and was not walking; and I looked up into the sky, and saw the sky astonished; and I looked up to the pole of the heavens, and saw it standing, and the birds of the air keeping still. And I looked down upon the earth, and saw a trough lying, and work-people reclining: and their hands were in the trough. And those that were eating did not eat, and those that were rising did not carry it up, and those that were conveying anything to their mouths did not convey it; but the faces of all were looking upwards.

\pend\pstart
And I saw the sheep walking, and the sheep stood still; and the shepherd raised his hand to strike them, and his hand remained up. And I looked upon the current of the river, and I saw the mouths of the kids resting on the water and not drinking, and all things in a moment were driven from their course.

\pend\pstart
\eledsection*{XIX}

\pend\setcounter{pstartR}{1}\pstart
And I saw a woman coming down from the hill-country, and she said to me: O man, whither art thou going? And I said: I am seeking an Hebrew midwife. And she answered and said unto me: Art thou of Israel? And I said to her: Yes. And she said: And who is it that is bringing forth in the cave? And I said: A woman betrothed to me. And she said to me: Is she not thy wife? And I said to her: It is Mary that was reared in the temple of the Lord, and I obtained her by lot as my wife. And yet she is not my wife, but has conceived of the Holy Spirit. And the midwife said to him: Is this true? And Joseph said to her: Come and see. And the midwife went away with him.

\pend\pstart
And they stood in the place of the cave, and behold a luminous cloud overshadowed the cave. And the midwife said: My soul has been magnified this day, because mine eyes have seen strange things\textemdash because salvation has been brought forth to Israel. And immediately the cloud disappeared out of the cave, and a great light shone in the cave, so that the eyes could not bear it. And in a little that light gradually decreased, until the infant appeared, and went and took the breast from His mother Mary. And the midwife cried out, and said: This is a great day to me, because I have seen this strange sight.

\pend\pstart
And the midwife went forth out of the cave, and Salome met her. And she said to her: Salome, Salome, I have a strange sight to relate to thee: a virgin has brought forth\textemdash a thing which her nature admits not of. Then said Salome: As the Lord my God liveth, unless I thrust in my finger, and search the parts, I will not believe that a virgin has brought forth.

\pend\pstart
\eledsection*{XX}

\pend\setcounter{pstartR}{1}\pstart
And the midwife went in, and said to Mary: Show thyself; for no small controversy has arisen about thee. And Salome put in her finger, and cried out, and said: Woe is me for mine iniquity and mine unbelief, because I have tempted the living God; and, behold, my hand is dropping off as if burned with fire.

\pend\pstart
And she bent her knees before the Lord, saying: O God of my fathers, remember that I am the seed of Abraham, and Isaac, and Jacob; do not make a show of me to the sons of Israel, but restore me to the poor; for Thou knowest, O Lord, that in Thy name I have performed my services, and that I have received my reward at Thy hand.

\pend\pstart
And, behold, an angel of the Lord stood by her, saying to her: Salome, Salome, the Lord hath heard thee. Put thy hand to the infant, and carry it, and thou wilt have safety and joy.

\pend\pstart
And Salome went and carried it, saying: I will worship Him, because a great King has been born to Israel. And, behold, Salome was immediately cured, and she went forth out of the cave justified. And behold a voice saying: Salome, Salome, tell not the strange things thou hast seen, until the child has come into Jerusalem.

\pend\pstart
\eledsection*{XXI}

\pend\setcounter{pstartR}{1}\pstart
And, behold, Joseph was ready to go into Judæa. And there was a great commotion in Bethlehem of Judæa, for Magi came, saying: Where is he that is born king of the Jews? for we have seen his star in the east, and have come to worship him.

\pend\pstart
And when Herod heard, he was much disturbed, and sent officers to the Magi. And he sent for the priests, and examined them, saying: How is it written about the Christ? where is He to be born? And they said: In Bethlehem of Judæa, for so it is written.\footnote{Two MSS. here add: And thou Bethlehem, etc., from Mic. v. 2.} And he sent them away. And he examined the Magi, saying to them: What sign have you seen in reference to the king that has been born? And the Magi said: We have seen a star of great size shining among these stars, and obscuring their light, so that the stars did not appear; and we thus knew that a king has been born to Israel, and we have come to worship him. And Herod said: Go and seek him; and if you find him, let me know, in order that I also may go and worship him.

\pend\pstart
And the Magi went out. And, behold, the star which they had seen in the east went before them until they came to the cave, and it stood over the top of the cave. And the Magi saw the infant with His mother Mary; and they brought forth from their bag gold, and frankincense, and myrrh. And having been warned by the angel not to go into Judæa, they went into their own country by another road.\footnote{Matt. ii. 1\textendash 12. One of the MSS. here adds Matt. ii. 13\textendash 15, with two or three slight variations.}

\pend\pstart
\eledsection*{XXII}

\pend\setcounter{pstartR}{1}\pstart
And when Herod knew that he had been mocked by the Magi, in a rage he sent murderers, saying to them: Slay the children\footnote{Four MSS. have \textit{all the male children}, as in Matt. ii. 16.} from two years old and under.

\pend\pstart
And Mary, having heard that the children were being killed, was afraid, and took the infant and swaddled Him, and put Him into an ox-stall.

\pend\pstart
And Elizabeth, having heard that they were searching for John, took him and went up into the hill-country, and kept looking where to conceal him. And there was no place of concealment. And Elizabeth, groaning with a loud voice, says: O mountain of God, receive mother and child. And immediately the mountain was cleft, and received her. And a light shone about them, for an angel of the Lord was with them, watching over them.

\pend\pstart
\eledsection*{XXIII}

\pend\setcounter{pstartR}{1}\pstart
And Herod searched for John, and sent officers to Zacharias, saying: Where hast thou hid thy son? And he, answering, said to them: I am the servant of God in holy things, and I sit constantly in the temple of the Lord: I do not know where my son is.

\pend\pstart
And the officers went away, and reported all these things to Herod. And Herod was enraged, and said: His son is destined to be king over Israel. And he sent to him again, saying: Tell the truth; where is thy son? for thou knowest that thy life is in my hand.

\pend\pstart
And Zacharias said: I am God's martyr, if thou sheddest my blood; for the Lord will receive my spirit, because thou sheddest innocent blood at the vestibule of the temple of the Lord. And Zacharias was murdered about daybreak. And the sons of Israel did not know that he had been murdered.\footnote{Another reading is: And Herod, enraged at this, ordered him to be slain in the midst of the altar before the dawn, that the slaying of him might not be prevented by the people. [This incident was probably suggested by the reference to ``Zacharias the son of Barachias'' in Matt. xxiii. 35, Luke xi. 51; but comp. 2 Chron. xxiv. 20\textendash 22.\textemdash R.]}

\pend\pstart
\eledsection*{XXIV}

\pend\setcounter{pstartR}{1}\pstart
But at the hour of the salutation the priests went away, and Zacharias did not come forth to meet them with a blessing, according to his custom.\footnote{Lit., the blessing of Zacharias did not come forth, etc.} And the priests stood waiting for Zacharias to salute him at the prayer,\footnote{Or, with prayer.} and to glorify the Most High.

\pend\pstart
And he still delaying, they were all afraid. But one of them ventured to go in, and he saw clotted blood beside the altar; and he heard a voice saying: Zacharias has been murdered, and his blood shall not be wiped up until his avenger come. And hearing this saying, he was afraid, and went out and told it to the priests.

\pend\pstart
And they ventured in, and saw what had happened; and the fretwork of the temple made a wailing noise, and they rent their clothes\footnote{Another reading is: And was rent from the top, etc.} from the top even to the bottom. And they found not his body, but they found his blood turned into stone. And they were afraid, and went out and reported to the people that Zacharias had been murdered. And all the tribes of the people heard, and mourned, and lamented for him three days and three nights.

\pend\pstart
And after the three days, the priests consulted as to whom they should put in his place; and the lot fell upon Simeon. For it was he who had been warned by the Holy Spirit that he should not see death until he should see the Christ in the flesh.\footnote{Luke ii. 26. One of the MSS. here adds Matt. ii. 19\textendash 23, with two or three verbal changes.}

\pend\pstart
\eledsection*{XXV}

\pend\setcounter{pstartR}{1}\pstart
And I James that wrote this history in Jerusalem, a commotion having arisen when Herod died, withdrew myself to the wilderness until the commotion in Jerusalem ceased, glorifying the Lord God, who had given me the gift and the wisdom to write this history.\footnote{[Assuming that this is among the most ancient of the Apocryphal Gospels, it is noteworthy that the writer abstains from elaborating his statements on points fully narrated in the Canonical Gospels. The \textit{supplementary} character of the earliest of these writings is obvious. But what a contrast between the impressive silence of the New Testament narratives, and the garrulity, not to say indelicacy, of these detailed descriptions of the Nativity!\textemdash R.]}

\pend\pstart
And grace shall be with them that fear our Lord Jesus Christ, to whom be glory to ages of ages. Amen.\footnote{The MSS. vary much in the doxology, e.g. {\greekfont Ἐγὼ δὲ Ἰάκωβος ἔγραψα τὴν ἱστορίαν ταύτην ἐν Ἱερουσαλήμ, καὶ ἐδόξασα τὸν δεσπότην θεὸν τὸν ἀποκαλύψαντα ἡμῖν τὰ μυστήρια ταῦτα, ὅτι αὐτῷ πρέπει δόξα, κράτος εἰς τοὺς αἰῶνας τῶν αἰώνων, ἀμήν.}}

\pend


\endnumbering
\end{Rightside}
\end{pairs}
\Columns

{\center\Large
*

* *

{\greekfont
ΓΕΝΕΣΙΣ ΜΑΡΙΑΣ, ΑΠΟΚΑΛΥΨΙΣ ΙΑΚΩΒ.

Εἰρήνη τῷ γράψαντι καὶ τῷ ἀναγινώσκοντι.

}
}

\end{document}
