\documentclass[12pt]{book} % use larger type; default would be 10pt

% usual packages loading:
\usepackage{calc}
\usepackage{fontspec}
\usepackage{graphicx} % support the \includegraphics command and options
\usepackage{fullpage} % to reduce the margins
\usepackage{titlesec}
\usepackage[oldstyle]{libertine}
\usepackage{lettrine}
\usepackage{reledmac}
\usepackage{reledpar}
\usepackage{xcolor}

\eledsectnotoc{L}
\eledsectmark{R}

% RED
\definecolor{benred8}{HTML}{E82C00} 

% BLUE
\definecolor{benblue1}{HTML}{2B22C7}

% YELLOWS
\definecolor{benyellow1}{HTML}{FFD435}
\definecolor{benyellow2}{HTML}{7C6F3B}

\begin{document}

\chapter*{ΠΡΩΤΕΥΑΓΓΕΛΙΟΝ ΙΑΚΩΒΟΥ}

The Birth of Mary the Holy Mother of God, and Very Glorious Mother of Jesus Christ.\footnote{This title is taken by Tischendorf from a manuscript of the eleventh century (Paris). At least seventeen other forms exist. The book is variously named by ancient writers. In the decree of Gelasius (a.d. 495) he condemns it as \textit{Evangelium nomine Jacobi minoris apocryphum}. The text of Tischendorf, here translated, is somewhat less diffuse than that of Fabricius, and is based on manuscript evidence. The variations are verbal and formal rather than material. R.}

\begin{pairs}
\begin{Leftside}\beginnumbering\pstart
\eledsection*{Αʹ}

\pend\pstart
Ἐν ταῖς ἱστορίαις τῶν δώδεκα φυλῶν τοῦ Ἰσραὴλ ἦν Ἰωακεὶμ πλούσιος σφόδρα, καὶ προσέφερε κυρίῳ τὰ δῶρα αὐτοῦ διπλᾶ λέγων ἐν ἑαυτῷ: Ἔσται τὸ τῆς περισσείας μου ἅπαντι τῷ λαῷ καὶ τὸ τῆς ἀφέσεως κυρίῳ τῷ θεῷ εἰς ἱλασμὸν ἐμοί.

\pend\pstart
ἐνἤγγισεν δὲ ἡ ἡμέρα κυρίου ἡ μεγάλη καὶ προσέφερον οἱ υἱοὶ Ἰσραὴλ τὰ δῶρα αὐτῶν, καὶ ἔστη κατενώπιον αὐτοῦ καὶ Ῥουβὴλ λέγων: οὐκ ἔξεστίν σοι πρώτῳ προσενεγκεῖν τὰ δῶρά σου, καθότι σπέρμα οὐκ ἐποίησας ἐν τῷ Ἰσραήλ.

\pend\pstart
καὶ ἐλυπήθη Ἰωακεὶμ καὶ ἀπίει εἰς τὸν οἶκον αὐτοῦ, καὶ ἐλθὼν εἰς τὴν δωδεκάφυλον τοῦ λαοῦ λέγει: ὄψομαι, εἰ ἐγὼ μόνος οὐκ ἐποίησα σπέρμα ἐν τῷ Ἰσραήλ. ἠρεύνησε δὲ καὶ εὗρε πάντας τοὺς δικαίους, ὅτι σπέρμα ἀνέστησαν ἐν τῷ Ἰσραὴλ, καὶ ἐμνήσθη τοῦ πατριάρχου Ἀβραάμ, ὅτι ἐν ταῖς ἐσχάταις αὐτοῦ ἡμέραις ἔδωκεν αὐτῷ ὁ θεὸς υἱὸν Ἰσαάκ.

\pend\pstart
καὶ ἐλυπεῖτο Ἰωακεὶμ σφόδρα καὶ οὐκ ἐφάνη τῇ γυναικὶ αὐτοῦ, ἀλλὰ ἔδωκεν ἑαυτὸν εἰς τὴν ἔρημον, καὶ ἔπηξε τὴν σκηνὴν αὐτοῦ ἐκεῖ καὶ ἐνήστευσεν ἡμέρας τεσσεράκοντα καὶ νύκτας τεσσεράκοντα λέγων ἑν ἑαυτῷ: οὐ καταβήσομαι οὔτε ἐπὶ βρωτὸν οὔτε ἐπὶ ποτόν, ἕως ἐπισκέψηταί με κύριος ὁ θεός μου, καὶ ἔσται μοι ἡ εὐχὴ βρόματα καὶ πόματα.

\pend\pstart
\eledsection*{Βʹ}

\pend\pstart
Ἡ δὲ γυνὴ δὲ αὐτοῦ Ἄννα δύο θρήνους ἐθρήνει καὶ δύο κοπετοὺς ἐκόπτετο λέγουσα: κόψομαι τὴν χηρίαν μου καὶ κόψομαι τὴν ἀτεκνίαν μου.

\pend\pstart
ἤγγισε δὲ ἡ ἡμέρα κυρίου ἡ μεγάλη καὶ εἶπεν Ἰουδὴθ ἡ παιδίσκη αὐτῆς πρὸς αὐτήν: ἕως πότε ταπεινοῖς τὴν ψυχήν σου; ἰδοὺ γὰρ ἤγγισεν ἡ ἡμέρα κυρίου ἡ μεγάλη καὶ οὐκ ἔξεστί σοι πενθεῖν. ἀλλὰ λάβε τοῦτο τὸ κεφαλο\-δέσμιον, ὅ ἔδωκέν μοι ἡ κυρία τοῦ ἔργου, καὶ οὐκ ἔξεστί μοι ἀναδήσασθαι αὐτό, καθότι παιδίσκη σού εἰμι καὶ χαρακτῆρα ἔχει βασιλικόν.

\pend\pstart
καὶ εἶπεν Ἄννα: ἀπόστηθι ἀπ' ἐμοῦ: καὶ ταῦτα οὐκ ἐποίησα, καὶ κύριος ὁ θεὸς ἐτα\-πείνωσέν με σφόδρα. μήπως πανοῦργος ἔδωκέν σοι τοῦτο καὶ ἦλθες κοινωνῆσαί με τῇ ἁμαρτίᾳ σου; εἶπεν δὲ αὐτῇ Ἰουδὴθ ἡ παιδίσκη αὐτῆς: τί ἀράσωμαί σοι, καθότι οὐκ ἤκουσας τῆς φωνῆς μου; ἀπέκλεισεν κύριος ὁ θεὸς τὴν μήτραν σου τοῦ μὴ δοῦναί σοι καρπὸν ἐν τῷ Ἰσραήλ.

\pend\pstart
καὶ ἐλυπήθη Ἄννα σφόδρα καὶ περιείλετο τὰ ἱμάτια αὐτῆς τὰ πενθικὰ καὶ ἐσμήξατο τὴν κεφαλὴν αὐτῆς καὶ ἐνεδύσατο τὰ ἱμάτια αὐτῆς τὰ νυμφικὰ καὶ περὶ ὥραν ἐννάτην κατέβη εἰς τὸν παράδεισον αὐτῆς (τοῦ περι\-πατῆσαι). καὶ εἶδεν δάφνην καὶ ἐκάθισεν ὑποκάτω αὐτῆς καὶ ἐλιτάνευσε τῷ δεσπότῃ λέγουσα: ὁ θεὸς τῶν πατέρων μου, εὐλόγησόν με καὶ ἐπάκουσον τῆς δεήσεός μου, καθὼς ἐπήκουσας καὶ εὐλόγησας τὴν μητέραν Σάραν καὶ ἔδωκας αὐτῇ υἱὸν τὸν Ἰσαάκ.

\pend\pstart
\eledsection*{Γʹ}

\pend\pstart
Καὶ ἀτενίσασα Ἄννα εἰς οὐρανὸν εἶδεν καλιὰν στρουθίων ἐν τῇ δάφνῃ καὶ εὐθέως ἐποίησε θρῆνον ἐν ἑαυτῇ λέγουσα: οἴμοι, τίς με ἐγέννησεν, ποία δὲ μήτρα ἐξέφυσέν με, ὅτι κατάρα ἐγεννήθην ἐνώπιον τῶν υἱῶν Ἰσραήλ καὶ ὠνειδίσθην καὶ ἐξεμυκτηρίσθην ἐκβληθεῖσα ἐκ ναοῦ κυρίου τοῦ θεοῦ μου;

\pend\pstart
οἴμοι, τίνι ὁμοιώθην ἐγώ; οὐχ ὁμοιώθην ἐγὼ τοῖς πετεινοῖς τοῦ οὐρανοῦ, ὅτι καὶ τὰ πετεινὰ γόνιμά εἰσιν ἐνώπιόν σου, κύριε. οἴμοι, τίνι ὁμοιώθην ἐγώ; οὐχ ὁμοιώθην ἐγὼ τοῖς ἀλόγοις ζώοις, καὶ τὰ ἄλογα ζῶα γόνιμά εἰσιν ἐνώπιόν σου, κύριε.

\pend\pstart
οἴμοι, τίνι ὁμοιώθην ἐγώ; οὐχ ὁμοιώθην ἐγὼ τοῖς ὕδασι τούτοις, ὅτι καὶ τὰ ὕδατα γόνιμά εἰσιν ἐνώπιόν σου, κύριε. οἴμοι, τίνι ὁμοιώθην ἐγώ; οὐχ ὁμοιώθην ἐγὼ τῇ γῇ, ὅτι καὶ ἡ γῆ προφέρει τοὺς καρποὺς αὐτῆς κατὰ καιρὸν καί σε εὐλογεῖ, κύριε.

\pend\pstart
\eledsection*{Δʹ}

\pend\pstart
Καὶ ἰδοὺ ἄγγελος κυρίου ἐπέστη λέγων: Ἄννα, Ἄννα, εἰσήκουσε κύριος ὁ θεὸς τῆς δεήσεός σου, καὶ λήψῃ καὶ λαληθήσεται τὸ σπέρμα σου ἐν ὅλῃ τῇ οἰκουμένῃ. εἶπεν δὲ Ἄννα: ζῇ κύριος ὁ θεός μου: ἐὰν γεννήσω εἴτε ἄρρεν εἴτε θῆλυ, προσάξω αὐτὸ δῶρον κυρίῳ τῷ θεῷ μου καὶ ἔσται λειτουργοῦν αὐτῷ πάσας ἡμέρας τῆς ζωῆς αὐτοῦ.

\pend\pstart
καὶ ἰδοὺ ἤλθοσαν ἄγγελοι δύοι λέγοντες αὐτῇ: ἰδοὺ Ἰωακεὶμ ὁ ἀνήρ σου ἔρχεται μετὰ τῶν ποιμνίων αὐτοῦ. ἄγγελος γὰρ κυρίου κατέβη πρὸς αὐτὸν λέγων: Ἰωακείμ, Ἰωακείμ, εἰσήκουσε κύριος ὁ θεὸς τῆς δεήσεός σου. κατάβηθι ἐντεῦθεν. ἰδοὺ Ἄννα ἡ γυνή σου ἐν γαστρὶ λήψεται (εἴληφεν).

\pend\pstart
καὶ εὐθέως κατέβη Ἰωακεὶμ καὶ ἐκάλεσεν τοὺς ποιμένας αὐτοῦ λέγων: φέρετέ μοι ὧδε δώδεκα ἀμνάδας ἀσπίλους καὶ ἀμόμους εἰς θυσίαν κυρίῳ τῷ θεῷ μου, καὶ φέρετέ μοι δώδεκα μόσχους ἀσπίλους καὶ ἔσονται τοῖς ἱερεῦσι καὶ τῇ γερουσίᾳ, καὶ φέρετέ μοι ἑκατὸν χιμάρους καὶ ἔσονται αἱ ἑκατὸν χίμαροι παντὶ τῷ λαῷ.

\pend\pstart
καὶ ἰδοὺ ἥκει Ἰωακεὶμ μετὰ τῶν ποιμνίων αὐτοῦ. καὶ ἔστη Ἄννα πρὸς τῇ πύλῃ τοῦ οἴκου αὐτῆς καὶ εἶδεν τὸν Ἰωακεὶμ ἐρχόμενον μετὰ τῶν ποιμνίων αὐτοῦ. καὶ ἔδραμεν Ἄννα καὶ ἐκρεμάσθη ἐπὶ τὸν τράχηλον αὐτοῦ λέγουσα: νῦν οἶδα, ὅτι κύριος ὁ θεὸς εὐλόγησέ με σφόδρα: ἰδοὺ γὰρ ἡ χήρα οὐκέτι χήρα καὶ ἡ ἄτεκνος ἰδοὺ ἐν γαστρὶ λήψομαι εἴληφα . καὶ ἀνεπαύσατο Ἰωακεὶμ τὴν πρώτην ἡμέραν εἰς τὸν οἶκον αὐτοῦ.

\pend\pstart
\eledsection*{Εʹ}

\pend\pstart
Τῇ δὲ ἐπαύριον προσέφερε τὰ δῶρα αὐτοῦ λέγων ἐν ἑαυτῷ: ἐὰν κύριος ὁ θεὸς ἱλασθῇ μοι, τὸ πέταλον τοῦ ἱερέως φανερών μοι ποιήσει. καὶ προσέφερεν τὰ δῶρα αὐτοῦ Ἰωακεὶμ καὶ προσεῖχε τῷ πετάλῳ τοῦ ἱερέως, ὡς ἐπέβη ἐπὶ τὸ θυσιαστήριον κυρίου, καὶ ἁμαρτία οὐχ εὑρέθη ἐν αὐτῷ. καὶ εἶπεν Ἰωακείμ: νῦν οἶδα, ὅτι κύριος ὁ θεὸς ἱλάσθη μοι καὶ ἀφεῖλέν μου πάντα τὰ ἁμαρτήματα. καὶ κατέβη ἐκ ναοῦ κυρίου δεδικαιωμένος καὶ ἀπῆλθεν εἰς τὸν οἶκον αὐτοῦ χαίρων καὶ δοξάζων τὸν θεόν.

\pend\pstart
ἐπληρώθησαν δὲ οἱ μῆνες αὐτῆς. ἐν δὲ τῷ ἐνάτῳ μηνὶ ἐγέννησεν Ἄννα καὶ εἶπεν τῇ μαίᾳ: τί ἐγέννησα; ἡ δὲ εἶπεν: θῆλυ. καὶ εἶπεν Ἄννα: ἐμεγάλυνεν ἡ ψυχή μου τὴν ἡμέραν ταύτην καὶ ἀνέκλινεν αὐτήν. πληρωθεισῶν δὲ τῶν ἡμερῶν ἀπεσμήξατο Ἄννα καὶ ἔδωκεν μασθὸν τῇ παιδί. ἐκάλεσεν δὲ τὸ ὄνομα αὐτῆς Μαριάμ.

\pend\pstart
% \eledsection*{Ϛʹ}
\eledsection*{Ϝʹ}

\pend\pstart
Ἡμέρᾳ δὲ καὶ ἡμέρᾳ ἐκραταιοῦτο ἡ παῖς. γενομένης δὲ αὐτῆς ἑξαμήνου ἔστησεν αὐτὴν ἡ μήτηρ αὐτῆς χαμαὶ τοῦ πειράσαι, εἰ ἵσταται: καὶ περιπατήσασα ἑπτὰ βήματα ἦλθεν εἰς τὸν κόλπον τῆς μητρὸς αὐτῆς, καὶ ἀνήρπασεν αὐτὴν ἡ μήτηρ αὐτῆς λέγουσα: ζῇ κύριος ὁ θεός μου: οὐ μὴ περιπατήσῃς ἐν τῇ γῇ ταύτῃ, ἕως οὗ ἀπάξω σε ἐν τῷ ναῷ κυρίου. καὶ ἐποίησεν ἁγίασμα ἐν τῷ κοιτῶνι αὐτῆς καὶ πᾶν κοινὸν ἤ ἀκάθαρτον οὐκ εἴα διέρχεσθαι δι' αὐτῆς. καὶ ἐκάλεσε τὰς θυγατέρας τῶν Ἑβραίων τὰς ἀμιάντους, καὶ διεπλάνων αὐτήν.

\pend\pstart
ἐγένετο δὲ πρῶτος ἐνιαυτὸς τῇ παιδί, καὶ ἐποίησεν Ἰωακεὶμ δοχὴν μεγάλην καὶ ἐκάλεσεν τοὺς ἱερεῖς καὶ τοὺς γραμματεῖς καὶ τὴν γερουσίαν καὶ πάντα τὸν λαὸν Ἰσραήλ. καὶ προσήνεγκεν Ἰωακεὶμ τὴν παῖδα τοῖς ἱερεῦσι καὶ εὐλόγησαν αὐτὴν οἱ ἱερεῖς λέγοντες: ὁ θεὸς τῶν πατέρων ἡμῶν, εὐλόγησον τὴν παῖδα ταύτην καὶ δὸς αὐτῇ ὄνομα ὀνομαστὸν αἰώνιον ἐν πάσαις ταῖς γενεαῖς. καὶ εἶπεν ὁ λαός: γένοιτο, γένοιτο, ἀμήν. καὶ προσήνεγκεν Ἰωακεὶμ τὴν παῖδα τοῖς ἀρχιερεῦσι, καὶ εὐλόγ\-ησαν αὐτὴν λέγοντες: ὁ θεὸς τῶν ὑψωμάτων, ἐπίβλεψον ἐπὶ τὴν παῖδα ταύτην καὶ εὐλόγησον αὐτὴν ἐσχάτην εὐλογίαν, ἥτις διαδοχὴν οὐχ ἕξει.

\pend\pstart
καὶ ἀπήγαγον αὐτὴν ἐν τῷ ἁγιάσματι τοῦ κοιτῶνος αὐτῆς: καὶ λαβοῦσα Ἄννα ἔδωκε μασθὸν τῇ παιδὶ καὶ ᾖσεν ᾆσμα κυρίῳ τῷ θεῷ λέγουσα: ᾄσω ὠδὴν κυρίῳ τῷ θεῷ μου, ὅτι ἐπεσκέψατό με καὶ ἀφεῖλεν ἀπ' ἐμοῦ τὸν ὀνειδισμὸν τῶν ἐχθρῶν μου καὶ ἔδωκέ μοι καρπὸν δικαιοσύνης μονοούσιον αὐτῷ καὶ πολυπλούσιον. τίς ἀναγγελεῖ τοῖς υἱοῖς Ῥουβίμ , ὅτι Ἄννα θηλάζει; καὶ ἀνέπαυσεν αὐτὴν ἡ μήτηρ αὐτῆς ἐν τῷ ἁγιάσματι τοῦ κοιτῶνος αὐτῆς καὶ ἐξῆλθε καὶ διηκόνει αὐτοῖς. τελεσ\-θέντος δὲ τοῦ δείπνου κατέβησαν εὐφραινόμενοι καὶ ἐδόξασαν τὸν θεὸν Ἰσραήλ.

\pend\pstart
\eledsection*{Ζʹ}

\pend\endnumbering\end{Leftside}
\begin{Rightside}\beginnumbering\pstart
\eledsection*{I}

\pend\pstart
In the records of the twelve tribes of Israel was Joachim, a man rich exceedingly; and he brought his offerings double\footnote{Susanna i. 4.}, saying: There shall be of my superabundance to all the people, and there shall be the offering for my forgiveness\footnote{The readings vary, and the sense is doubtful. Thilo thinks that the sense is: What I offer over and above what the law requires is for the benefit of the whole people; but the offering I make for my own forgiveness (according to the law's requirements) shall be to the Lord, that He may be rendered merciful to me.} to the Lord for a propitiation\footnote{The Church of Rome appoints March 20 as the Feast of St. Joachim. His liberality is commemorated in prayers, and the lessons to be read are Wisd. xxxi. and Matt. i.} for me.

\pend\pstart
For the great day of the Lord was at hand, and the sons of Israel were bringing their offerings. And there stood over against him Rubim, saying: It is not meet for thee first to bring thine offerings, because thou hast not made seed in Israel\footnote{1 Sam. i. 6, 7; Hos. ix. 14.}.

\pend\pstart
And he searched, and found that all the righteous had raised up seed in Israel. And he called to mind the patriarch Abraham, that in the last day\footnote{Another reading is: In his last days.} God gave him a son Isaac.

\pend\pstart
And Joachim was exceedingly grieved, and did not come into the presence of his wife; but he retired to the desert\footnote{Another reading is: Into the hill-country.}, and there pitched his tent, and fasted forty days and forty nights,\footnote{Moses: Ex. xxiv. 18, xxxiv. 28; Deut. ix. 9. Elijah: 1 Kings xix. 8. Christ: Matt. iv. 2.} saying in himself: I will not go down either for food or for drink until the Lord my God shall look upon me, and prayer shall be my food and drink.

\pend\pstart
\eledsection*{II}

\pend\pstart
And his wife Anna\footnote{The 26th day of July is the Feast of St. Anna in the Church of Rome.} mourned in two mournings, and lamented in two lamentations, saying: I shall bewail my widowhood; I shall bewail my childlessness.

\pend\pstart
And the great day of the Lord was at hand; and Judith\footnote{Other forms of the name are Juth, Juthin.} her maid-servant said: How long dost thou humiliate thy soul? Behold, the great day of the Lord is at hand, and it is unlawful for thee to mourn. But take this head-band, which the woman that made it gave to me; for it is not proper that I should wear it, because I am a maid-servant, and it has a royal appearance\footnote{Some MSS. have: For I am thy maid-servant, and thou hast a regal appearance.}.

\pend\pstart
And Anna said: Depart from me; for I have not done such things, and the Lord has brought me very low. I fear that some wicked person has given it to thee, and thou hast come to make me a sharer in thy sin. And Judith said: Why should I curse thee, seeing that\footnote{Several MSS. insert: Thou hast not listened to my voice; for.} the Lord hath shut thy womb, so as not to give thee fruit in Israel?

\pend\pstart
And Anna was grieved exceedingly, and put off her garments of mourning, and cleaned her head, and put on her wedding garments, and about the ninth hour went down to the garden to walk. And she saw a laurel, and sat under it, and prayed to the Lord, saying: O God of our fathers, bless me and hear my prayer, as Thou didst bless the womb of Sarah, and didst give her a son Isaac\footnote{Comp. 1 Sam. i. 9\textendash 18.}.

\pend\pstart
\eledsection*{III}

\pend\pstart
And gazing towards the heaven, she saw a sparrow's nest in the laurel\footnote{Tobit ii. 10.}, and made a lamentation in herself, saying: Alas! who begot me? and what womb produced me? because I have become a curse in the presence of the sons of Israel, and I have been reproached, and they have driven me in derision out of the temple of the Lord.

\pend\pstart
Alas! to what have I been likened? I am not like the fowls of the heaven, because even the fowls of the heaven are productive before Thee, O Lord. Alas! to what have I been likened? I am not like the beasts of the earth, because even the beasts of the earth are productive before Thee, O Lord.

\pend\pstart
Alas! to what have I been likened? I am not like these waters, because even these waters are productive before Thee, O Lord. Alas! to what have I been likened? I am not like this earth, because even the earth bringeth forth its fruits in season, and blesseth Thee, O Lord\footnote{Many of the MSS. here add: Alas! to what have I been likened? I am not like the waves of the sea, because even the waves of the sea, in calm and storm, and the fishes in them, bless Thee, O Lord.}.

\pend\pstart
\eledsection*{IV}

\pend\pstart
And, behold, an angel of the Lord stood by, saying: Anna, Anna, the Lord hath heard thy prayer, and thou shalt conceive, and shall bring forth; and thy seed shall be spoken of in all the world. And Anna said: As the Lord my God liveth, if I beget either male or female, I will bring it as a gift to the Lord my God; and it shall minister to Him in holy things all the days of its life\footnote{1 Sam. i. 11.}.

\pend\pstart
And, behold, two angels came, saying to her: Behold, Joachim thy husband is coming with his
flocks\footnote{One of the MSS.: With his shepherds, and sheep, and goats, and oxen.}. For an angel of the Lord went down to him, saying: Joachim, Joachim, the Lord God hath heard thy prayer. Go down hence; for, behold, thy wife Anna shall conceive.

\pend\pstart
And Joachim went down and called his shepherds, saying: Bring me hither ten she-lambs without spot or blemish, and they shall be for the Lord my God; and bring me twelve tender calves, and they shall be for the priests and the elders; and a hundred goats for all the people.

\pend\pstart
And, behold, Joachim came with his flocks; and Anna stood by the gate, and saw Joachim coming, and she ran and hung upon his neck, saying: Now I know that the Lord God hath blessed me exceedingly; for, behold the widow no longer a widow, and I the childless shall conceive. And Joachim rested the first day in his house.

\pend\pstart
\eledsection*{V}

\pend\pstart
And on the following day he brought his offerings, saying in himself: If the Lord God has been rendered gracious to me, the plate\footnote{Ex. xxviii. 36\textendash 38. For traditions about the \textit{petalon}, see Euseb., \textit{H. E.}, ii. 23, iii. 31, v. 24; Epiph., \textit{Hær.}, 78.} on the priest's forehead will make it manifest to me. And Joachim brought his offerings, and observed attentively the priest's plate when he went up to the altar of the Lord, and he saw no sin in himself. And Joachim said: Now I know that the Lord has been gracious unto me, and has remitted all my sins. And he went down from the temple of the Lord justified, and departed to his own house.

\pend\pstart
And her months were fulfilled, and in the ninth\footnote{Various readings are: Sixth, seventh, eighth.} month Anna brought forth. And she said to the midwife: What have I brought forth? and she said: A girl. And said Anna: My soul has been magnified this day. And she laid her down. And the days having been fulfilled, Anna was purified, and gave the breast to the child\footnote{One of the MSS. inserts: On the eighth day.}, and called her name Mary.

\pend\pstart
\eledsection*{VI}

\pend\pstart
And the child grew strong day by day; and when she was six\footnote{One of the MSS. has nine.} months old, her mother set her on the ground to try whether she could stand, and she walked seven steps and came into her bosom; and she snatched her up, saying: As the Lord my God liveth, thou shalt not walk on this earth until I bring thee into the temple of the Lord. And she made a sanctuary in her bed-chamber, and allowed nothing common or unclean to pass through her. And she called the undefiled daughters of the Hebrews, and they led her astray\footnote{This is the reading of most mss.; but it is difficult to see any sense in it. One MS. reads: They attended on her. Fabricius proposed: They bathed her.}.

\pend\pstart
And when she was a year old, Joachim made a great feast, and invited the priests, and the scribes, and the elders, and all the people of Israel. And Joachim brought the child to the priests; and they blessed her, saying: O God of our fathers, bless this child, and give her an everlasting name to be named in all generations. And all the people said: So be it, so be it, amen. And he brought her to the chief priests; and they blessed her, saying: O God most high, look upon this child, and bless her with the utmost blessing, which shall be for ever.

\pend\pstart
And her mother snatched her up, and took her into the sanctuary of her bed-chamber, and gave her the breast. And Anna made a song to the Lord God, saying: I will sing a song to the Lord my God, for He hath looked upon me, and hath taken away the reproach of mine enemies; and the Lord hath given the fruit of His righteousness, singular in its kind, and richly endowed before Him. Who will tell the sons of Rubim that Anna gives suck? Hear, hear, ye twelve tribes of Israel, that Anna gives suck. And she laid her to rest in the bed-chamber of her sanctuary, and went out and ministered unto them. And when the supper was ended, they went down rejoicing, and glorifying the God of Israel\footnote{Two of the MSS. add: And they gave her the name of Mary, because her name shall not fade forever. This derivation of the name\textemdash from the root \textit{mar}, fade\textemdash is one of a dozen or so.}.

\pend\pstart
\eledsection*{VII}

\pend\endnumbering\end{Rightside}
\end{pairs}
\Columns

\end{document}
