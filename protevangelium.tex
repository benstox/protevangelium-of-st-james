\documentclass[10pt]{book}

\usepackage{fontspec}
\usepackage{fullpage} % to reduce the margins
\usepackage{reledmac}
\usepackage{reledpar}
\usepackage{titlesec}
\usepackage{xcolor}

\setmainfont{EB Garamond}
\newfontfamily{\greekfont}{GFS Artemisia}

% hide reledpar line numbers
\numberlinefalse
% format reledpar paragraph/verse numbers
\renewcommand{\thepstartL}{\textnormal{\textcolor{benred8}{\arabic{pstartL}. }}}
\renewcommand{\thepstartR}{\textnormal{\textcolor{benred8}{\arabic{pstartR}. }}}

% RED
\definecolor{benred8}{HTML}{E82C00} 

% BLUE
\definecolor{benblue1}{HTML}{2B22C7}

% YELLOWS
\definecolor{benyellow1}{HTML}{FFD435}
\definecolor{benyellow2}{HTML}{7C6F3B}

\begin{document}
% Had a problem with Greek words extending beyond the text width and even crossing over
% into the English text on the right. This command fixes that.
\sloppy

\title{\scshape{\greekfont ΠΡΩΤΕΥΑΓΓΕΛΙΟΝ ΙΑΚΩΒΟΥ}\\The Protevangelium of James}
\author{ed. Bernard Stockermans}
\date{A.D. MMXX}
\maketitle

\chapter*{}

{\center\Large
The Birth of Mary the Holy Mother of God, and Very Glorious Mother of Jesus Christ.\footnote{This title is taken by Tischendorf from a manuscript of the eleventh century (Paris). At least seventeen other forms exist. The book is variously named by ancient writers. In the decree of Gelasius (A.D. 495) he condemns it as \textit{Evangelium nomine Jacobi minoris apocryphum}. The text of Tischendorf, here translated, is somewhat less diffuse than that of Fabricius, and is based on manuscript evidence. The variations are verbal and formal rather than material.\textemdash R.}}

\begin{pairs}
\begin{Leftside}
\beginnumbering
\numberpstarttrue

{\greekfont
\pstart
\eledsection*{Αʹ}

\pend\pstart
Ἐν ταῖς ἱστορίαις τῶν δώδεκα φυλῶν τοῦ Ἰσραὴλ ἦν Ἰωακεὶμ πλούσιος σφόδρα, καὶ προσέφερε κυρίῳ τὰ δῶρα αὐτοῦ διπλᾶ λέγων ἐν ἑαυτῷ: Ἔσται τὸ τῆς περισσείας μου ἅπαντι τῷ λαῷ καὶ τὸ τῆς ἀφέσεως κυρίῳ τῷ θεῷ εἰς ἱλασμὸν ἐμοί.

\pend\pstart
ἐνἤγγισεν δὲ ἡ ἡμέρα κυρίου ἡ μεγάλη καὶ προσέφερον οἱ υἱοὶ Ἰσραὴλ τὰ δῶρα αὐτῶν, καὶ ἔστη κατενώπιον αὐτοῦ καὶ Ῥουβὴλ λέγων: οὐκ ἔξεστίν σοι πρώτῳ προσενεγκεῖν τὰ δῶρά σου, καθότι σπέρμα οὐκ ἐποίησας ἐν τῷ Ἰσραήλ.

\pend\pstart
καὶ ἐλυπήθη Ἰωακεὶμ καὶ ἀπίει εἰς τὸν οἶκον αὐτοῦ, καὶ ἐλθὼν εἰς τὴν δωδεκάφυλον τοῦ λαοῦ λέγει: ὄψομαι, εἰ ἐγὼ μόνος οὐκ ἐποίησα σπέρμα ἐν τῷ Ἰσραήλ. ἠρεύνησε δὲ καὶ εὗρε πάντας τοὺς δικαίους, ὅτι σπέρμα ἀνέστησαν ἐν τῷ Ἰσραὴλ, καὶ ἐμνήσθη τοῦ πατριάρχου Ἀβραάμ, ὅτι ἐν ταῖς ἐσχάταις αὐτοῦ ἡμέραις ἔδωκεν αὐτῷ ὁ θεὸς υἱὸν Ἰσαάκ.

\pend\pstart
καὶ ἐλυπεῖτο Ἰωακεὶμ σφόδρα καὶ οὐκ ἐφάνη τῇ γυναικὶ αὐτοῦ, ἀλλὰ ἔδωκεν ἑαυτὸν εἰς τὴν ἔρημον, καὶ ἔπηξε τὴν σκηνὴν αὐτοῦ ἐκεῖ καὶ ἐνήστευσεν ἡμέρας τεσσεράκοντα καὶ νύκτας τεσσεράκοντα λέγων ἑν ἑαυτῷ: οὐ καταβήσομαι οὔτε ἐπὶ βρωτὸν οὔτε ἐπὶ ποτόν, ἕως ἐπισκέψηταί με κύριος ὁ θεός μου, καὶ ἔσται μοι ἡ εὐχὴ βρόματα καὶ πόματα.

\pend\pstart
\eledsection*{Βʹ}

\pend\pstart
Ἡ δὲ γυνὴ δὲ αὐτοῦ Ἄννα δύο θρήνους ἐθρήνει καὶ δύο κοπετοὺς ἐκόπτετο λέγουσα: κόψομαι τὴν χηρίαν μου καὶ κόψομαι τὴν ἀτεκνίαν μου.

\pend\pstart
ἤγγισε δὲ ἡ ἡμέρα κυρίου ἡ μεγάλη καὶ εἶπεν Ἰουδὴθ ἡ παιδίσκη αὐτῆς πρὸς αὐτήν: ἕως πότε ταπεινοῖς τὴν ψυχήν σου; ἰδοὺ γὰρ ἤγγισεν ἡ ἡμέρα κυρίου ἡ μεγάλη καὶ οὐκ ἔξεστί σοι πενθεῖν. ἀλλὰ λάβε τοῦτο τὸ κεφαλο\-δέσμιον, ὅ ἔδωκέν μοι ἡ κυρία τοῦ ἔργου, καὶ οὐκ ἔξεστί μοι ἀναδήσασθαι αὐτό, καθότι παιδίσκη σού εἰμι καὶ χαρακτῆρα ἔχει βασιλικόν.

\pend\pstart
καὶ εἶπεν Ἄννα: ἀπόστηθι ἀπ' ἐμοῦ: καὶ ταῦτα οὐκ ἐποίησα, καὶ κύριος ὁ θεὸς ἐτα\-πείνωσέν με σφόδρα. μήπως πανοῦργος ἔδωκέν σοι τοῦτο καὶ ἦλθες κοινωνῆσαί με τῇ ἁμαρτίᾳ σου; εἶπεν δὲ αὐτῇ Ἰουδὴθ ἡ παιδίσκη αὐτῆς: τί ἀράσωμαί σοι, καθότι οὐκ ἤκουσας τῆς φωνῆς μου; ἀπέκλεισεν κύριος ὁ θεὸς τὴν μήτραν σου τοῦ μὴ δοῦναί σοι καρπὸν ἐν τῷ Ἰσραήλ.

\pend\pstart
καὶ ἐλυπήθη Ἄννα σφόδρα καὶ περιείλετο τὰ ἱμάτια αὐτῆς τὰ πενθικὰ καὶ ἐσμήξατο τὴν κεφαλὴν αὐτῆς καὶ ἐνεδύσατο τὰ ἱμάτια αὐτῆς τὰ νυμφικὰ καὶ περὶ ὥραν ἐννάτην κατέβη εἰς τὸν παράδεισον αὐτῆς (τοῦ περι\-πατῆσαι). καὶ εἶδεν δάφνην καὶ ἐκάθισεν ὑποκάτω αὐτῆς καὶ ἐλιτάνευσε τῷ δεσπότῃ λέγουσα: ὁ θεὸς τῶν πατέρων μου, εὐλόγησόν με καὶ ἐπάκουσον τῆς δεήσεός μου, καθὼς ἐπήκουσας καὶ εὐλόγησας τὴν μητέραν Σάραν καὶ ἔδωκας αὐτῇ υἱὸν τὸν Ἰσαάκ.

\pend\pstart
\eledsection*{Γʹ}

\pend\pstart
Καὶ ἀτενίσασα Ἄννα εἰς οὐρανὸν εἶδεν καλιὰν στρουθίων ἐν τῇ δάφνῃ καὶ εὐθέως ἐποίησε θρῆνον ἐν ἑαυτῇ λέγουσα: οἴμοι, τίς με ἐγέννησεν, ποία δὲ μήτρα ἐξέφυσέν με, ὅτι κατάρα ἐγεννήθην ἐνώπιον τῶν υἱῶν Ἰσραήλ καὶ ὠνειδίσθην καὶ ἐξεμυκτηρίσθην ἐκβληθεῖσα ἐκ ναοῦ κυρίου τοῦ θεοῦ μου;

\pend\pstart
οἴμοι, τίνι ὁμοιώθην ἐγώ; οὐχ ὁμοιώθην ἐγὼ τοῖς πετεινοῖς τοῦ οὐρανοῦ, ὅτι καὶ τὰ πετεινὰ γόνιμά εἰσιν ἐνώπιόν σου, κύριε. οἴμοι, τίνι ὁμοιώθην ἐγώ; οὐχ ὁμοιώθην ἐγὼ τοῖς ἀλόγοις ζώοις, καὶ τὰ ἄλογα ζῶα γόνιμά εἰσιν ἐνώπιόν σου, κύριε.

\pend\pstart
οἴμοι, τίνι ὁμοιώθην ἐγώ; οὐχ ὁμοιώθην ἐγὼ τοῖς ὕδασι τούτοις, ὅτι καὶ τὰ ὕδατα γόνιμά εἰσιν ἐνώπιόν σου, κύριε. οἴμοι, τίνι ὁμοιώθην ἐγώ; οὐχ ὁμοιώθην ἐγὼ τῇ γῇ, ὅτι καὶ ἡ γῆ προφέρει τοὺς καρποὺς αὐτῆς κατὰ καιρὸν καί σε εὐλογεῖ, κύριε.

\pend\pstart
\eledsection*{Δʹ}

\pend\pstart
Καὶ ἰδοὺ ἄγγελος κυρίου ἐπέστη λέγων: Ἄννα, Ἄννα, εἰσήκουσε κύριος ὁ θεὸς τῆς δεήσεός σου, καὶ λήψῃ καὶ λαληθήσεται τὸ σπέρμα σου ἐν ὅλῃ τῇ οἰκουμένῃ. εἶπεν δὲ Ἄννα: ζῇ κύριος ὁ θεός μου: ἐὰν γεννήσω εἴτε ἄρρεν εἴτε θῆλυ, προσάξω αὐτὸ δῶρον κυρίῳ τῷ θεῷ μου καὶ ἔσται λειτουργοῦν αὐτῷ πάσας ἡμέρας τῆς ζωῆς αὐτοῦ.

\pend\pstart
καὶ ἰδοὺ ἤλθοσαν ἄγγελοι δύοι λέγοντες αὐτῇ: ἰδοὺ Ἰωακεὶμ ὁ ἀνήρ σου ἔρχεται μετὰ τῶν ποιμνίων αὐτοῦ. ἄγγελος γὰρ κυρίου κατέβη πρὸς αὐτὸν λέγων: Ἰωακείμ, Ἰωακείμ, εἰσήκουσε κύριος ὁ θεὸς τῆς δεήσεός σου. κατάβηθι ἐντεῦθεν. ἰδοὺ Ἄννα ἡ γυνή σου ἐν γαστρὶ λήψεται (εἴληφεν).

\pend\pstart
καὶ εὐθέως κατέβη Ἰωακεὶμ καὶ ἐκάλεσεν τοὺς ποιμένας αὐτοῦ λέγων: φέρετέ μοι ὧδε δώδεκα ἀμνάδας ἀσπίλους καὶ ἀμόμους εἰς θυσίαν κυρίῳ τῷ θεῷ μου, καὶ φέρετέ μοι δώδεκα μόσχους ἀσπίλους καὶ ἔσονται τοῖς ἱερεῦσι καὶ τῇ γερουσίᾳ, καὶ φέρετέ μοι ἑκατὸν χιμάρους καὶ ἔσονται αἱ ἑκατὸν χίμαροι παντὶ τῷ λαῷ.

\pend\pstart
καὶ ἰδοὺ ἥκει Ἰωακεὶμ μετὰ τῶν ποιμνίων αὐτοῦ. καὶ ἔστη Ἄννα πρὸς τῇ πύλῃ τοῦ οἴκου αὐτῆς καὶ εἶδεν τὸν Ἰωακεὶμ ἐρχόμενον μετὰ τῶν ποιμνίων αὐτοῦ. καὶ ἔδραμεν Ἄννα καὶ ἐκρεμάσθη ἐπὶ τὸν τράχηλον αὐτοῦ λέγουσα: νῦν οἶδα, ὅτι κύριος ὁ θεὸς εὐλόγησέ με σφόδρα: ἰδοὺ γὰρ ἡ χήρα οὐκέτι χήρα καὶ ἡ ἄτεκνος ἰδοὺ ἐν γαστρὶ λήψομαι εἴληφα . καὶ ἀνεπαύσατο Ἰωακεὶμ τὴν πρώτην ἡμέραν εἰς τὸν οἶκον αὐτοῦ.

\pend\pstart
\eledsection*{Εʹ}

\pend\pstart
Τῇ δὲ ἐπαύριον προσέφερε τὰ δῶρα αὐτοῦ λέγων ἐν ἑαυτῷ: ἐὰν κύριος ὁ θεὸς ἱλασθῇ μοι, τὸ πέταλον τοῦ ἱερέως φανερών μοι ποιήσει. καὶ προσέφερεν τὰ δῶρα αὐτοῦ Ἰωακεὶμ καὶ προσεῖχε τῷ πετάλῳ τοῦ ἱερέως, ὡς ἐπέβη ἐπὶ τὸ θυσιαστήριον κυρίου, καὶ ἁμαρτία οὐχ εὑρέθη ἐν αὐτῷ. καὶ εἶπεν Ἰωακείμ: νῦν οἶδα, ὅτι κύριος ὁ θεὸς ἱλάσθη μοι καὶ ἀφεῖλέν μου πάντα τὰ ἁμαρτήματα. καὶ κατέβη ἐκ ναοῦ κυρίου δεδικαιωμένος καὶ ἀπῆλθεν εἰς τὸν οἶκον αὐτοῦ χαίρων καὶ δοξάζων τὸν θεόν.

\pend\pstart
ἐπληρώθησαν δὲ οἱ μῆνες αὐτῆς. ἐν δὲ τῷ ἐνάτῳ μηνὶ ἐγέννησεν Ἄννα καὶ εἶπεν τῇ μαίᾳ: τί ἐγέννησα; ἡ δὲ εἶπεν: θῆλυ. καὶ εἶπεν Ἄννα: ἐμεγάλυνεν ἡ ψυχή μου τὴν ἡμέραν ταύτην καὶ ἀνέκλινεν αὐτήν. πληρωθεισῶν δὲ τῶν ἡμερῶν ἀπεσμήξατο Ἄννα καὶ ἔδωκεν μασθὸν τῇ παιδί. ἐκάλεσεν δὲ τὸ ὄνομα αὐτῆς Μαριάμ.

\pend\pstart
% \eledsection*{Ϛʹ}
\eledsection*{Ϝʹ}

\pend\pstart
Ἡμέρᾳ δὲ καὶ ἡμέρᾳ ἐκραταιοῦτο ἡ παῖς. γενομένης δὲ αὐτῆς ἑξαμήνου ἔστησεν αὐτὴν ἡ μήτηρ αὐτῆς χαμαὶ τοῦ πειράσαι, εἰ ἵσταται: καὶ περιπατήσασα ἑπτὰ βήματα ἦλθεν εἰς τὸν κόλπον τῆς μητρὸς αὐτῆς, καὶ ἀνήρπασεν αὐτὴν ἡ μήτηρ αὐτῆς λέγουσα: ζῇ κύριος ὁ θεός μου: οὐ μὴ περιπατήσῃς ἐν τῇ γῇ ταύτῃ, ἕως οὗ ἀπάξω σε ἐν τῷ ναῷ κυρίου. καὶ ἐποίησεν ἁγίασμα ἐν τῷ κοιτῶνι αὐτῆς καὶ πᾶν κοινὸν ἤ ἀκάθαρτον οὐκ εἴα διέρχεσθαι δι' αὐτῆς. καὶ ἐκάλεσε τὰς θυγατέρας τῶν Ἑβραίων τὰς ἀμιάντους, καὶ διεπλάνων αὐτήν.

\pend\pstart
ἐγένετο δὲ πρῶτος ἐνιαυτὸς τῇ παιδί, καὶ ἐποίησεν Ἰωακεὶμ δοχὴν μεγάλην καὶ ἐκάλεσεν τοὺς ἱερεῖς καὶ τοὺς γραμματεῖς καὶ τὴν γερουσίαν καὶ πάντα τὸν λαὸν Ἰσραήλ. καὶ προσήνεγκεν Ἰωακεὶμ τὴν παῖδα τοῖς ἱερεῦσι καὶ εὐλόγησαν αὐτὴν οἱ ἱερεῖς λέγοντες: ὁ θεὸς τῶν πατέρων ἡμῶν, εὐλόγησον τὴν παῖδα ταύτην καὶ δὸς αὐτῇ ὄνομα ὀνομαστὸν αἰώνιον ἐν πάσαις ταῖς γενεαῖς. καὶ εἶπεν ὁ λαός: γένοιτο, γένοιτο, ἀμήν. καὶ προσήνεγκεν Ἰωακεὶμ τὴν παῖδα τοῖς ἀρχιερεῦσι, καὶ εὐλόγ\-ησαν αὐτὴν λέγοντες: ὁ θεὸς τῶν ὑψωμάτων, ἐπίβλεψον ἐπὶ τὴν παῖδα ταύτην καὶ εὐλόγησον αὐτὴν ἐσχάτην εὐλογίαν, ἥτις διαδοχὴν οὐχ ἕξει.

\pend\pstart
καὶ ἀπήγαγον αὐτὴν ἐν τῷ ἁγιάσματι τοῦ κοιτῶνος αὐτῆς: καὶ λαβοῦσα Ἄννα ἔδωκε μασθὸν τῇ παιδὶ καὶ ᾖσεν ᾆσμα κυρίῳ τῷ θεῷ λέγουσα: ᾄσω ὠδὴν κυρίῳ τῷ θεῷ μου, ὅτι ἐπεσκέψατό με καὶ ἀφεῖλεν ἀπ' ἐμοῦ τὸν ὀνειδισμὸν τῶν ἐχθρῶν μου καὶ ἔδωκέ μοι καρπὸν δικαιοσύνης μονοούσιον αὐτῷ καὶ πολυπλούσιον. τίς ἀναγγελεῖ τοῖς υἱοῖς Ῥουβίμ , ὅτι Ἄννα θηλάζει; καὶ ἀνέπαυσεν αὐτὴν ἡ μήτηρ αὐτῆς ἐν τῷ ἁγιάσματι τοῦ κοιτῶνος αὐτῆς καὶ ἐξῆλθε καὶ διηκόνει αὐτοῖς. τελεσ\-θέντος δὲ τοῦ δείπνου κατέβησαν εὐφραινόμενοι καὶ ἐδόξασαν τὸν θεὸν Ἰσραήλ.

\pend\pstart
\eledsection*{Ζʹ}

\pend\pstart
Τῇ δὲ παιδὶ προσετίθεντο οἱ μῆνες αὐτῆς. ἐγένετο δὲ διετὴς ἡ παῖς, καὶ εἶπεν Ἰωακείμ: ἀπάξωμεν αὐτὴν ἐν τῷ ναῷ κυρίου καὶ ἀποδῶμεν τὴν ἐπαγγελίαν, ἥν ἐπηγγειλάμεθα, μήπως ἀποστείλῃ κύριος ὁ θεὸς πρὸς ἡμᾶς καὶ γένηται ἀπρόσδεκτον τὸ δῶρον ἡμῶν. καὶ εἶπεν Ἄννα: ἀναμείνωμεν τὸ τρίτον ἔτος, ὅπως μὴ ζητήσῃ πατέρα ἤ μητέρα. καὶ εἶπεν Ἰωακείμ: ἀμήν, γένοιτο.

\pend\pstart
ἐγένετο δὲ τριετὴς ἡ παῖς, καὶ εἶπεν Ἰωακείμ: καλέσωμεν τὰς θυγατέρας τῶν Ἑβραίων τὰς ἀμιάντους, καὶ λαβέτωσαν ἀνὰ λαμπάδα, καὶ ἔστωσαν καιόμεναι, ἵνα μὴ ἐπιστραφῇ ἡ παῖς εἰς τὰ ὀπίσω καὶ αἰχμαλωτισθῇ ἡ καρδία αὐτῆς ἐκ ναοῦ κυρίου. καὶ ἐποίησαν οὕτως, ἕως οὗ ἀνέβησαν ἐν τῷ ναῷ κυρίου. καὶ ἐδέξατο αὐτὴν ὁ ἱερεὺς καὶ καταφιλήσας εὐλόγησε καὶ εἶπεν: ἐμεγάλυνε κύριος ὁ θεὸς τὸ ὄνομά σου ἐν πάσαις ταῖς γενεαῖς τῆς γῆς: (ἐπὶ σοὶ) ἐπ' ἐσχάτου τῶν ἡμερῶν φανερώσει κύριος ὁ θεὸς τὸ λύτρον τῶν υἱῶν Ἰσραήλ.

\pend\pstart
καὶ ἐκάθισεν αὐτὴν ἐπὶ τρίτου βαθμοῦ τοῦ θυσιαστηρίου, καὶ ἔβαλε κύριος ὁ θεὸς χάριν ἐπ' αὐτήν, καὶ κατεχόρευσε τοῖς ποσὶν αὐτοῖς, καὶ ἠγάπησεν αὐτὴν πᾶς οἶκος Ἰσραήλ.

\pend\pstart
\eledsection*{Ηʹ}

\pend\pstart
κατέβησαν δὲ οἱ γονεῖς αὐτῆς θαυμάζοντες καὶ ἐπαινοῦντες τὸν θεόν, ὅτι οὐκ ἐπεστράφη ἡ παῖς εἰς τὰ ὀπίσω. ἦν δὲ Μαριὰμ ὡσεὶ περιστερὰ νεμομένη ἐν τῷ ναῷ κυρίου καὶ ἐλάμβανε τροφὴν ἐκ χειρὸς ἀγγέλου.

\pend\pstart
γενομένης δὲ αὐτῆς δωδεκαετοῦς συμβούλιον ἐγένετο τῶν ἱερέων λεγόντων: ἰδοὺ Μαριὰμ γέγονε δωδεκαέτης ἐν τῷ ναῷ κυρίου: τί οὖν ποιήσωμεν αὐτήν, μήπως (ἐπέλθῃ αὐτῇ τὰ γυναικῶν καὶ) μιάνῃ τὸ ἁγίασμα κυρίου. καὶ εἶπον τῷ ἀρχιερεῖ: σὺ ἕστηκας ἐπὶ τὸ θυσιαστήριον θεοῦ: εἴσελθε καὶ πρόσευξαι περὶ αὐτῆς, καὶ ὅ ἄν φανερώσῃ σοι κύριος ὁ θεός, τοῦτο ποιήσωμεν.

\pend\pstart
καὶ εἰσῆλθεν ὁ ἱερεὺς λαβὼν τὸν δωδεκα\-κόδωνα (ἱεροπρεπὲς ἱμάτιον) εἰς τὰ ἅγια τῶν ἁγίων καὶ ηὔξατο περὶ αὐτῆς. καὶ ἰδοὺ ἄγγελος κυρίου ἐπέστη αὐτῷ λέγων: Ζαχαρία, Ζαχαρία, ἔξελθε καὶ ἐκκλησίασον τοὺς χηρεύοντας τοῦ λαοῦ, καὶ ἐνεγκάτωσαν ἀνὰ ῥάβδον, καὶ εἰς ὅν ἐὰν δείξῃ κύριος ὁ θεὸς σημεῖον, τούτου ἔσται γυνή. καὶ ἐξῆλθον οἱ κήρυκες καθ' ὅλης τῆς περιχώρου τῆς Ἰουδαίας, καὶ ἤχησεν ἡ σάλπιγξ κυρίου, καὶ ἔδραμον πάντες.

\pend\pstart
\eledsection*{Θʹ}

\pend\pstart
Ἰωσὴφ δὲ ῥίψας τὸ σκέπαρνον ἔδραμε καὶ αὐτὸς εἰς τὴν συναγωγήν, καὶ συναχθέντες ὁμοῦ ἀπῆλθαν πρὸς τὸν ἱερέα. ἔλαβε δὲ πάντων τὰς ῥάβδους ὁ ἱερεὺς καὶ εἰσῆλθεν εἰς τὸ ἱερὸν καὶ ηὔξατο. τελέσας δὲ τὴν εὐχὴν ἐξῆλθε καὶ ἐπέδωκεν ἑνὶ ἑκάστῳ τὴν ἑαυτοῦ ῥάβδον, καὶ σημεῖον οὐκ ἦν ἐν αὐτοῖς. τὴν δὲ ἐσχάτην ῥάβδον ἔλαβεν ὁ Ἰωσήφ, καὶ ἰδοὺ περιστερὰ ἐξῆλθεν ἐκ τῆς ῥάβδου καὶ ἐπετάσθη ἐπὶ τὴν κεφαλὴν Ἰωσήφ. καὶ εἶπεν αὐτῷ ὁ ἱερεύς: σὺ κεκλήρωσαι τὴν παρθένον κυρίου παραλαβεῖν. παράλαβε αὐτὴν εἰς τήρησιν σεαυτῷ.

\pend\pstart
ἀντεῖπε δὲ Ἰωσὴφ λέγων: υἱοὺς ἔχω καὶ πρεσβύτης εἰμί, αὕτη δὲ νεωτέρα. μήπως κατάγελως γένωμαι τοῖς υἱοῖς Ἰσραήλ; εἶπεν δὲ αὐτῷ ὁ ἱερεύς: Ἰωσήφ, φοβήθητι κύριον τὸν θεὸν καὶ ὅσα ἐποίησε Δαθὰμ καὶ Κορὲ καὶ Ἀβηρών, πῶς ἐδιχάσθη ἡ γῆ καὶ κατε\-ποντίσθησαν ἅπαντες διὰ τὴν ἀντιλογίαν αὐτῶν. καὶ νῦν φοβήθητι, Ἰωσήφ, μήπως ἔσται ταῦτα ἐν τῷ οἴκῳ σου.

\pend\pstart
καὶ φοβηθεὶς Ἰωσὴφ παρέλαβεν αὐτὴν εἰς τήρησιν. καὶ εἶπεν αὐτῇ: Μαρία, ἰδοὺ παρέλαβόν σε ἐκ ναοῦ κυρίου τοῦ θεοῦ μου καὶ νῦν καταλιμπάνω σε ἐν τῷ οἴκῳ μου, ἀπέρχομαι γὰρ οἰκοδομῆσαι τὰς οἰκοδομάς μου, καὶ ἐν τάχει ἥξω πρὸς σέ. κύριος ὁ θεὸς διαφυλάξει σε.

\pend\pstart
\eledsection*{Ιʹ}

\pend\pstart
Ἐγένετο δὲ συμβούλιον τῶν ἱερέων λεγόντων: ποιήσωμεν καταπέτασμα τῷ ναῷ κυρίου. καὶ εἶπεν ὁ ἱερεύς: καλέσατέ μοι ὧδε ἑπτὰ παρθένους ἀμιάντους ἐκ φυλῆς Δαυίδ. καὶ ἀπῆλθον οἱ ὑπηρέται καὶ εὕρησαν ἑπτά (εὗρον ἕξ). καὶ ἐμνήσθη ὁ ἱερεύς, ὅτι Μαρία ἐκ φυλῆς Δαυίδ ἐστι καὶ ἀμίαντός ἐστιν. καὶ ἀπῆλθαν οἱ ὑπηρέται καὶ ἤγαγον αὐτήν. καὶ εἰσήγαγεν αὐτὰς ὁ ἱερεὺς ἐν τῷ ναῷ κυρίου καὶ εἶπεν: λάχετέ μοι ὧδε, τίς νήσει τὸ χρυσίον καὶ τὸ ἀμίαντον καὶ τὸ βύσσινον καὶ τὸ σηρικοῦν καὶ τὸ ὑάκινθον καὶ τὸ κόκκινον καὶ τὴν ἀληθινὴν πορφύραν. καὶ ἔλαχεν τὴν Μαριὰμ τὸ κόκκινον καὶ ἡ ἀληθινὴ πορφύρα. καὶ λαβοῦσα ἀπῆλθεν εἰς τὸν οἶκον αὐτῆς. τῷ δὲ καιρῷ ἐκείνῳ Ζαχαρίας ἐσίγησεν. Μαριὰμ δὲ λαβοῦσα τὸ κόκκινον ἔκλωσεν.

\pend\pstart
\eledsection*{ΙΑʹ}

\pend\pstart
Καὶ λαβοῦσα κάλπιν ἐξῆλθεν γεμίσαι ὕδωρ, καὶ ἰδοὺ φωνὴ λέγουσα: χαῖρε κεχαριτωμένη, ὁ κύριος μετὰ σοῦ, εὐλογημένη σὺ ἐν γυναιξί. καὶ περιεβλέπετο δεξιὰ καὶ ἀριστερά, πόθεν αὕτη ἡ φωνὴ ὑπάρχει, καὶ ἔντρομος γενομένη ἀπῆλθεν εἰς τὸν οἶκον αὐτῆς. καὶ ἀναπαύσασα τὴν κάλπην ἔλαβε πάλιν τὴν πορφύραν καὶ ἐκάθισεν ἐπὶ τὸν θρόνον καὶ εἷλκεν αὐτήν.

\pend\pstart
καὶ ἰδοὺ ἄγγελος κυρίου ἐπέστη λέγων αὐτῇ: μὴ φοβοῦ, Μαριάμ, εὗρες γὰρ χάριν ἐνώπιον τοῦ θεοῦ καὶ συλλήψῃ ἐκ λόγου αὐτοῦ. ἀκούσασα δὲ Μαριὰμ διεκρίθη ἐν ἑαυτῇ λέγουσα: ἐγὼ συλλήψομαι, ὡς πᾶσα γυνὴ γεννᾷ;

\pend\pstart
καὶ λέγει πρὸς αὐτὴν ὁ ἄγγελος: οὐχ οὕτως, Μαριάμ: δύναμις γὰρ θεοῦ ἐπισκιάσει σοι, διὸ καὶ τὸ γεννόμενον (ἐκ σοῦ) ἅγιον κληθήσεται υἱὸς ὑψίστου, καὶ καλέσεις τὸ ὄνομα αὐτοῦ Ἰησοῦν: αὐτὸς γὰρ σώσει τὸν λαὸν αὐτοῦ ἀπὸ τῶν ἁμαρτιῶν αὐτῶν. καὶ εἶπεν Μαριάμ: ἰδοὺ ἡ δούλη κυρίου: γένοιτό μοι κατὰ τὸ ῥῆμά σου.

\pend\pstart
\eledsection*{ΙΒʹ}

\pend\pstart
Καὶ ἐποίησεν τὴν πορφύραν καὶ τὸ κόκκινον καὶ ἀπήνεγκεν αὐτὰ τῷ ἱερεῖ, καὶ εὐλόγησεν αὐτὴν ὁ ἱερεὺς καὶ εἶπεν: Μαριάμ, ἐμεγάλυνε κύριος ὁ θεὸς τὸ ὄνομά σου ἐν πάσαις ταῖς γενεαῖς τῆς γῆς καὶ ἔσῃ εὐλογημένη ὑπὸ κυρίου.

\pend\pstart
χαρὰν δὲ λαβοῦσα Μαριὰμ ἀπῆλθε πρὸς τὴν συγγενίδα αὐτῆς Ἐλισάβετ καὶ ἔκρουσε πρὸς τῇ θύρᾳ. καὶ ἀκούσασα Ἐλισάβετ ἔρριψε τὸ ἐν χερσὶν, καὶ δραμοῦσα ἤνοιξεν αὐτῇ καὶ εὐλόγησεν αὐτὴν καὶ εἶπεν: πόθεν μοι τοῦτο, ἵνα ἡ μήτηρ τοῦ κυρίου μου ἔλθῃ πρὸς ἐμέ; ἰδοὺ γὰρ τὸ ἐν ἐμοὶ βρέφος ἐσκίρτησε καὶ εὐλόγησέν σε. Μαριὰμ δὲ ἐπελάθετο τῶν μυστηρίων, ὧν εἶπεν πρὸς αὐτὴν Γαβριήλ, καὶ ἀτενίσασα εἰς τὸν οὐρανὸν εἶπεν: τίς εἰμι ἐγώ, ὅτι πᾶσαι αἱ γυναῖκες μακαριοῦσί με;

\pend\pstart
ἐποίησε δὲ τρεῖς μῆνας πρὸς τὴν Ἐλισάβετ καὶ ἀπῆλθεν εἰς τὸν οἶκον αὐτῆς. ἡμέρᾳ δὲ ἀφ' ἡμέρας ἡ γαστὴρ αὐτῆς ὀγκοῦτο, καὶ ἔκρυβεν ἑαυτὴν ἀπὸ τῶν υἱῶν Ἰσραήλ. ἦν δὲ ἐτῶν πεντεκαίδεκα, ὅτε τὰ μυστήρια ταῦτα ἐγένοντο.

\pend\pstart
\eledsection*{ΙΓʹ}

\pend\pstart
Ἐγένετο δὲ ἕκτος μὴν καὶ ἦλθεν Ἰωσὴφ ἀπὸ τῶν οἰκοδομῶν αὐτοῦ καὶ εἰσῆλθεν ἐν τῷ οἴκῳ αὐτοῦ καὶ εὗρε τὴν Μαριὰμ ὀγκωμένην. καὶ ἔτυψε τὸ πρόσωπον αὐτοῦ καὶ ἔρριψεν ἑαυτὸν χαμαὶ καὶ ἔκλαυσε λέγων: ποίῳ προσόπῳ ἀτενίσω πρὸς κύριον τὸν θεόν μου; τί δὴ εἴπω περὶ τῆς κόρης ταύτης, ὅτι παρθένον αὐτὴν παρέλαβον ἐκ ναοῦ κυρίου καὶ οὐκ ἐφύλαξα αὐτήν; τίς ὁ θηρεύσας με; τίς τὸ πονηρὸν τοῦτο ἐποίησεν ἐν τῷ οἴκῳ μου καὶ ἐμίανεν τὴν παρθένον; μήτι εἰς ἐμὲ ἀνεκεφα\-λαιόθη ἡ ἱστορία Ἀδάμ; ὥσπερ γὰρ Ἀδὰμ ἦν ἐν τῇ ὥρᾳ τῆς δοξολογίας αὐτοῦ καὶ ἦλθεν ὁ ὄφις καὶ εὗρεν τὴν Εὔαν μόνην καὶ ἐξηπάτησεν αὐτήν, οὕτως κἀμοί συνέβη.

\pend\pstart
καὶ ἀνέστη Ἰωσὴφ ἀπὸ τοῦ σάκκου καὶ ἐκάλεσε τὴν Μαριὰμ καὶ εἶπεν αὐτῇ: μεμελημένη τῷ θεῷ, τί τοῦτο ἐποίησας; τί ἐταπείνωσας τὴν ψυχήν σου; ἐπελάθου κυρίου τοῦ θεοῦ σου, ἡ ἀνατραφεῖσα εἰς τὰ ἅγια τῶν ἁγίων καὶ λαβοῦσα τροφὴν ἐκ χειρὸς ἀγγέλου καὶ χορεύσασα ἐν αὐτοῖς;

\pend\pstart
ἡ δὲ ἔκλαυσε πικρῶς λέγουσα: ζῇ κύριος ὁ θεός, καθότι καθαρά εἰμι ἐγὼ καὶ ἄνδρα οὐ γινώσκω. εἶπε δὲ αὐτῇ Ἰωσήφ: πόθεν οὖν ἐστι τοῦτο ἐν τῇ γαστρί σου; εἶπε δὲ αὐτῷ: ζῇ κύριος ὁ θεός μου, καθότι οὐ γινώσκω, πόθεν ἐστὶ τοῦτο τὸ ἐν τῇ γαστρί μου.

\pend\pstart
\eledsection*{ΙΔʹ}

\pend\pstart
Καὶ ἐφοβήθη Ἰωσὴφ σφόδρα καὶ ἠρέμησεν ἐξ αὐτῆς καὶ διελογίζετο, τί αὐτὴν ποιήσει, εἶπε δὲ ἐν ἑαυτῷ: ἐὰν αὐτῆς κρύψω τὸ ἁμάρτημα, εὑρεθήσομαι μαχόμενος τῷ νόμῳ κυρίου: καὶ ἐὰν αὐτὴν φανερὰν ποιήσω τοῖς υἱοῖς Ἰσραήλ, φοβοῦμαι, μήπως ἀγγελικόν ἐστι τὸ ἐν αὐτῇ καὶ εὑρεθήσομαι παραδιδοὺς αἷμα ἀθῷον εἰς κρίμα θανάτου. τί οὖν αὐτὴν ποιήσω; λάθρᾳ αὐτὴν ἀπολύσω ἀπ' ἐμοῦ. καὶ ταῦτα αὐτοῦ ἐνθυμουμένου κατέλαβεν αὐτὸν ἡ νύξ.

\pend\pstart
καὶ ἰδοὺ ἄγγελος κυρίου φαίνεται αὐτῷ κατ' ὄναρ λέγων: Ἰωσήφ (υἱὸς Δαυίδ), μὴ φοβηθῇς τὴν παῖδα ταύτην. τὸ γὰρ ἐν αὐτῇ γεννηθὲν ἐκ πνεύματός ἐστιν ἁγίου, καὶ καλέσεις τὸ ὄνομα αὐτοῦ Ἰησοῦν: αὐτὸς γὰρ σώσει τὸν λαὸν αὐτοῦ ἀπὸ τῶν ἁμαρτιῶν αὐτῶν. καὶ ἀνέστη Ἰωσὴφ ἀπὸ τοῦ ὕπνου καὶ ἐδόξασε τὸν θεὸν Ἰσραὴλ τὸν δόντα αὐτῷ τὴν χάριν ταύτην, καὶ ἐφύλασσε τὴν παῖδα.

\pend\pstart
\eledsection*{ΙΕʹ}

\pend\pstart
Ἠλθεν δὲ Ἄννας ὁ γραμματεὺς πρὸς αὐτὸν καὶ εἶπεν αὐτῷ: διὰ τί οὐκ ἐφάνης ἐν τῇ συναγωγῇ (συνόδῳ) ἡμῶν; καὶ εἶπεν αὐτῷ Ἰωσήφ: ὅτι κεκμηκὼς ἤμην ἐκ τῆς ὁδοῦ καὶ ἀνεπαυσάμην ἡμέραν μίαν . καὶ ἐστράφη Ἄννας καὶ εἶδεν τὴν παρθένον ὀγκωμένην.

\pend\pstart
καὶ ἀπελθὼν δρομαίως πρὸς τὸν (ἀρχ-)ἱερέα εἶπεν αὐτῷ: Ἰωσήφ, ὅν σὺ μαρτυρεῖς, ἠνόμησε σφόδρα. καὶ εἶπεν ὁ ἱερεύς: τί τοῦτο; καὶ εἶπεν Ἄννας: τὴν παρθένον, ἥν παρέλαβεν ἐκ ναοῦ κυρίου, ἐμίανεν αὐτήν. καὶ ἀποκριθεὶς ὁ ἱερεὺς εἶπεν αὐτῷ: Ἰωσὴφ; Ἰωσὴφ τοῦτο ἐποίησεν; καὶ εἶπεν Ἄννας: ἀπόστειλον ὑπηρέτας καὶ εὑρέσεις τὴν παρθένον ὀγκωμένην. καὶ ἀπῆλθον οἱ ὑπηρέται καὶ εὗρον αὐτήν, καθὼς εἶπεν, καὶ ἀπήγαγον ἅμα τῷ Ἰωσὴφ εἰς τὸ κριτήριον.

\pend\pstart
καὶ εἶπεν ὁ ἱερεύς: Μαριάμ, τί τοῦτο ἐποί\-ησας καὶ ἐταπείνωσας τὴν ψυχήν σου καὶ ἐπελάθου κυρίου τοῦ θεοῦ σου, ἡ ἀνατραφεῖσα εἰς τὰ ἅγια τῶν ἁγίων καὶ λαβοῦσα τροφὴν ἐκ χειρὸς ἀγγέλων, σὺ ἡ ἀκούσασα τὸν ὕμνον αὐτῶν καὶ χορεύσασα ἐνώπιον αὐτῶν; τί τοῦτο ἐποί\-ησας; ἡ δὲ ἔκλαυσε πικρῶς λέγουσα: ζῇ κύριος ὁ θεός, ὅτι καθαρά εἰμι ἐγὼ ἐνώπιον αὐτοῦ καὶ ἄνδρα οὐ γινώσκω.

\pend\pstart
καὶ εἶπεν ὁ ἀρχιερεύς: Ἰωσήφ, τί τοῦτο ἐποίησας; καὶ εἶπεν Ἰωσήφ: ζῇ κύριος ὁ θεός μου, ὅτι καθαρός εἰμι ἐξ αὐτῆς. καὶ εἶπεν ὁ ἀρχιερεύς: μὴ ψευδομαρτύρει, ἀλλὰ λέγε τὸ ἀληθές: ἔκλεψας τοὺς γάμους καὶ οὐκ ἐφανέρωσας τοῖς υἱοῖς Ἰσραήλ, καὶ οὐκ ἔκλινας τὴν κεφαλήν σου ὑπὸ τὴν κραταιὰν χεῖρα, ὅπως εὐλογηθῇ τὸ σπέρμα σου. καὶ Ἰωσὴφ ἐσίγησεν.

\pend\pstart
\eledsection*{ΙϜʹ}

\pend\pstart
Καὶ εἶπεν ὁ ἱερεύς: ἀπόδος τὴν παρθένον, ἥν παρέλαβες ἐκ ναοῦ κυρίου. καὶ περίδακρυς γενόμενος ὁ Ἰωσὴφ ἔστη. καὶ εἶπεν ὁ ἱερεύς: ποτιῶ ὑμᾶς τὸ ὕδωρ τῆς ἐλέγξεως κυρίου καὶ φανερώσει τὰ ἁμαρτήματα ὑμῶν ἐν ὀφθαλ\-μοῖς ὑμῶν.

\pend\pstart
καὶ λαβὼν ὁ ἱερεὺς ἐπότισε τὸν Ἰωσὴφ καὶ ἔπεμψεν αὐτὸν εἰς τὴν ὀρεινήν: καὶ ἦλθεν ὁλόκληρος. ἐπότισεν δὲ καὶ τὴν παρθένον καὶ ἔπεμψεν καὶ αὐτὴν εἰς τὴν ὀρεινήν: καὶ ἦλθεν ὁλόκληρος, καὶ ἐθαύμασε πᾶς ὁ λαός, ὅτι ἁμαρτία οὐχ εὑρέθη ἐν αὐτοῖς.

\pend\pstart
καὶ εἶπεν ὁ ἱερεύς: εἰ κύριος ὁ θεὸς οὐκ ἐφανέρωσεν τὴν ἁμαρτίαν ὑμῶν, οὐδὲ ἐγὼ κρίνω ὑμᾶς καὶ ἀπέλυσεν αὐτούς. καὶ παρέλαβεν Ἰωσὴφ τὴν Μαριὰμ καὶ ἀπίει εἰς τὸν οἶκον αὐτοῦ χαίρων καὶ δοξάζων τὸν θεὸν τοῦ Ἰσραήλ.

\pend\pstart
\eledsection*{ΙΖʹ}

\pend\pstart
Κέλευσις δὲ ἐγένετο ἀπὸ (τοῦ Ἀόστου) Ἡρώδου τοῦ βασιλέως ἀπογράψασθαι, ὅσοι εἰσὶν ἐν Βηθλεὲμ τῆς Ἰουδαίας. (ἠναγκάζετο δὲ Ἰωσὴφ ἀπελθεῖν ἐκ Ναζαρὲτ εἰς τὴν Βηθλεὲμ καὶ εἶπεν) καὶ εἶπεν Ἰωσήφ: ἐγὼ ἀπογράψομαι τοὺς υἱούς μου. ταύτην δὲ τὴν παῖδα τί ποιήσω; πῶς αὐτὴν ἀπογράψομαι; γυναῖκα ἐμήν; ἐπαισχύνομαι. ἀλλὰ θυγατέρα; οἶδαν οἱ υἱοὶ Ἰσραήλ, ὅτι οὐκ ἔστιν θυγάτηρ μου. αὐτὴ ἡ ἡμέρα Κυρίου ποιήσει, ὡς βούλεται.

\pend\pstart
καὶ ἔστρωσεν τὸν ὄνον, καὶ ἐκάθισεν αὐτὴν καὶ ἧλκεν ὁ υἱὸς αὐτοῦ καὶ ἠκολούθησεν Σαμουήλ (αὐτός). καὶ ἤγγισαν ἐπὶ μίλιον τρίτον, καὶ ἐστράφη Ἰωσὴφ καὶ εἶδεν αὐτὴν στυγνὴν καὶ ἔλεγεν: ἴσως τὸ ἐν αὐτῇ χειμάζει αὐτήν. καὶ πάλιν ἐστράφη Ἰωσὴφ καὶ εἶδεν αὐτὴν γελοῦσαν καὶ εἶπεν: Μαριάμμη, τί ἐστίν σοι τοῦτο, ὅτι τὸ πρόσωπόν σου βλέπω ποτὲ μὲν γελοῦντα ποτὲ δὲ στυγνάζον; καὶ εἶπεν αὐτῷ: Ἰωσήφ, ὅτι δύο λαοὺς βλέπω ἐν τοῖς ὀφθαλμοῖς μου, ἔνα κλαίοντα καὶ κοπτόμενον καὶ ἔνα χαίροντα καὶ ἀγαλλιῶντα.

\pend\pstart
καὶ ἤλθωσεν ἀνὰ μέσον τῆς ὁδοῦ, καὶ εἶπεν αὐτῷ Μαριάμμη: κατάγαγέ με ἀπὸ τοῦ ὄνου, ὅτι (τ)ὸ ἐν ἐμοὶ ἐπείγει με προελθεῖν. καὶ κατήγαγεν αὐτὴν ἐκεῖ καὶ εἶπεν αὐτῇ: ποῦ σε ἀπάξω καὶ σκεπάσω σου τὴν ἀσχημοσύνην, ὅτι ὁ τόπος ἔρημός ἐστιν;

\pend\pstart
\eledsection*{ΙΗʹ}

\pend\pstart
Καὶ εὗρεν ἐκεῖ σπήλαιον καὶ εἰσήγαγεν αὐτὴν καὶ παρέστησεν αὐτῇ τοὺς υἱοὺς αὐτοῦ καὶ ἐξῆλθεν ζητῆσαι μαῖαν ( Ἑβραίαν) ἐν χώρᾳ Βηθλεέμ.

\pend\pstart
ἐγὼ δὲ Ἰωσὴφ περιεπάτουν καὶ οὐ περιεπάτουν. καὶ ἀνέβλεψα εἰς τὸν πόλον τοῦ οὐρανοῦ καὶ εἶδον αὐτὸν ἑστῶτα, καὶ εἰς τὸν ἀέρα καὶ εἶδον αὐτὸν ἔκθαμβον, καὶ τὰ πετεινὰ τοῦ οὐρανοῦ ἠρεμοῦντα. καὶ ἐπέβλεψα ἐπὶ τὴν γῆν καὶ εἶδον σκάφην κειμένην καὶ ἐργάτας ἀνακειμένους, καὶ ἦσαν αἱ χεῖρες αὐτῶν ἐν τῇ σκάφῃ. καὶ οἱ μασόμενοι οὐκ ἐμασῶντο, καὶ οἱ αἴροντες οὐκ ἀνέφερον, καὶ οἱ προσφέροντες τῷ στόματι αὐτῶν οὐ προσέφερον. ἀλλὰ πάντων ἦν τὰ πρόσωπα ἄνω βλέποντα.

\pend\pstart
καὶ εἶδον ἐλαυνόμενα πρόβατα, καὶ τὰ πρόβατα ἑστήκει: καὶ ἐπῆρεν ὁ ποιμὴν τὴν χεῖρα αὐτοῦ τοῦ πατάξαι αὐτά, καὶ ἡ χεὶρ αὐτοῦ ἔστη ἄνω. καὶ ἀνέβλεψα ἐπὶ τὸν χείμαρρον τοῦ ποταμοῦ καὶ εἶδον ἐρίφους καὶ τὰ στόματα αὐτῶν ἐπικείμενα τῷ ὕδατι καὶ μὴ πίνοντα. καὶ πάντα ὑπὸ θῆξιν (θήζει, θίζει, θρίζιν, ἔκπληξιν) τῷ δρόμῳ ἀπηλαύνοντο.

\pend\pstart
\eledsection*{ΙΘʹ}

\pend


}

\endnumbering
\end{Leftside}

\begin{Rightside}
\beginnumbering
\numberpstarttrue

\pstart
\eledsection*{I}

\pend\pstart
In the records of the twelve tribes of Israel was Joachim, a man rich exceedingly; and he brought his offerings double\footnote{Susanna i. 4.}, saying: There shall be of my superabundance to all the people, and there shall be the offering for my forgiveness\footnote{The readings vary, and the sense is doubtful. Thilo thinks that the sense is: What I offer over and above what the law requires is for the benefit of the whole people; but the offering I make for my own forgiveness (according to the law's requirements) shall be to the Lord, that He may be rendered merciful to me.} to the Lord for a propitiation\footnote{The Church of Rome appoints March 20 as the Feast of St. Joachim. His liberality is commemorated in prayers, and the lessons to be read are Wisd. xxxi. and Matt. i.} for me.

\pend\pstart
For the great day of the Lord was at hand, and the sons of Israel were bringing their offerings. And there stood over against him Rubim, saying: It is not meet for thee first to bring thine offerings, because thou hast not made seed in Israel\footnote{1 Sam. i. 6, 7; Hos. ix. 14.}.

\pend\pstart
And he searched, and found that all the righteous had raised up seed in Israel. And he called to mind the patriarch Abraham, that in the last day\footnote{Another reading is: In his last days.} God gave him a son Isaac.

\pend\pstart
And Joachim was exceedingly grieved, and did not come into the presence of his wife; but he retired to the desert\footnote{Another reading is: Into the hill-country.}, and there pitched his tent, and fasted forty days and forty nights,\footnote{Moses: Ex. xxiv. 18, xxxiv. 28; Deut. ix. 9. Elijah: 1 Kings xix. 8. Christ: Matt. iv. 2.} saying in himself: I will not go down either for food or for drink until the Lord my God shall look upon me, and prayer shall be my food and drink.

\pend\pstart
\eledsection*{II}

\pend\pstart
And his wife Anna\footnote{The 26th day of July is the Feast of St. Anna in the Church of Rome.} mourned in two mournings, and lamented in two lamentations, saying: I shall bewail my widowhood; I shall bewail my childlessness.

\pend\pstart
And the great day of the Lord was at hand; and Judith\footnote{Other forms of the name are Juth, Juthin.} her maid-servant said: How long dost thou humiliate thy soul? Behold, the great day of the Lord is at hand, and it is unlawful for thee to mourn. But take this head-band, which the woman that made it gave to me; for it is not proper that I should wear it, because I am a maid-servant, and it has a royal appearance\footnote{Some MSS. have: For I am thy maid-servant, and thou hast a regal appearance.}.

\pend\pstart
And Anna said: Depart from me; for I have not done such things, and the Lord has brought me very low. I fear that some wicked person has given it to thee, and thou hast come to make me a sharer in thy sin. And Judith said: Why should I curse thee, seeing that\footnote{Several MSS. insert: Thou hast not listened to my voice; for.} the Lord hath shut thy womb, so as not to give thee fruit in Israel?

\pend\pstart
And Anna was grieved exceedingly, and put off her garments of mourning, and cleaned her head, and put on her wedding garments, and about the ninth hour went down to the garden to walk. And she saw a laurel, and sat under it, and prayed to the Lord, saying: O God of our fathers, bless me and hear my prayer, as Thou didst bless the womb of Sarah, and didst give her a son Isaac\footnote{Comp. 1 Sam. i. 9\textendash 18.}.

\pend\pstart
\eledsection*{III}

\pend\pstart
And gazing towards the heaven, she saw a sparrow's nest in the laurel\footnote{Tobit ii. 10.}, and made a lamentation in herself, saying: Alas! who begot me? and what womb produced me? because I have become a curse in the presence of the sons of Israel, and I have been reproached, and they have driven me in derision out of the temple of the Lord.

\pend\pstart
Alas! to what have I been likened? I am not like the fowls of the heaven, because even the fowls of the heaven are productive before Thee, O Lord. Alas! to what have I been likened? I am not like the beasts of the earth, because even the beasts of the earth are productive before Thee, O Lord.

\pend\pstart
Alas! to what have I been likened? I am not like these waters, because even these waters are productive before Thee, O Lord. Alas! to what have I been likened? I am not like this earth, because even the earth bringeth forth its fruits in season, and blesseth Thee, O Lord\footnote{Many of the MSS. here add: Alas! to what have I been likened? I am not like the waves of the sea, because even the waves of the sea, in calm and storm, and the fishes in them, bless Thee, O Lord.}.

\pend\pstart
\eledsection*{IV}

\pend\pstart
And, behold, an angel of the Lord stood by, saying: Anna, Anna, the Lord hath heard thy prayer, and thou shalt conceive, and shall bring forth; and thy seed shall be spoken of in all the world. And Anna said: As the Lord my God liveth, if I beget either male or female, I will bring it as a gift to the Lord my God; and it shall minister to Him in holy things all the days of its life\footnote{1 Sam. i. 11.}.

\pend\pstart
And, behold, two angels came, saying to her: Behold, Joachim thy husband is coming with his
flocks\footnote{One of the MSS.: With his shepherds, and sheep, and goats, and oxen.}. For an angel of the Lord went down to him, saying: Joachim, Joachim, the Lord God hath heard thy prayer. Go down hence; for, behold, thy wife Anna shall conceive.

\pend\pstart
And Joachim went down and called his shepherds, saying: Bring me hither ten she-lambs without spot or blemish, and they shall be for the Lord my God; and bring me twelve tender calves, and they shall be for the priests and the elders; and a hundred goats for all the people.

\pend\pstart
And, behold, Joachim came with his flocks; and Anna stood by the gate, and saw Joachim coming, and she ran and hung upon his neck, saying: Now I know that the Lord God hath blessed me exceedingly; for, behold the widow no longer a widow, and I the childless shall conceive. And Joachim rested the first day in his house.

\pend\pstart
\eledsection*{V}

\pend\pstart
And on the following day he brought his offerings, saying in himself: If the Lord God has been rendered gracious to me, the plate\footnote{Ex. xxviii. 36\textendash 38. For traditions about the \textit{petalon}, see Euseb., \textit{H. E.}, ii. 23, iii. 31, v. 24; Epiph., \textit{Hær.}, 78.} on the priest's forehead will make it manifest to me. And Joachim brought his offerings, and observed attentively the priest's plate when he went up to the altar of the Lord, and he saw no sin in himself. And Joachim said: Now I know that the Lord has been gracious unto me, and has remitted all my sins. And he went down from the temple of the Lord justified, and departed to his own house.

\pend\pstart
And her months were fulfilled, and in the ninth\footnote{Various readings are: Sixth, seventh, eighth.} month Anna brought forth. And she said to the midwife: What have I brought forth? and she said: A girl. And said Anna: My soul has been magnified this day. And she laid her down. And the days having been fulfilled, Anna was purified, and gave the breast to the child\footnote{One of the MSS. inserts: On the eighth day.}, and called her name Mary.

\pend\pstart
\eledsection*{VI}

\pend\pstart
And the child grew strong day by day; and when she was six\footnote{One of the MSS. has nine.} months old, her mother set her on the ground to try whether she could stand, and she walked seven steps and came into her bosom; and she snatched her up, saying: As the Lord my God liveth, thou shalt not walk on this earth until I bring thee into the temple of the Lord. And she made a sanctuary in her bed-chamber, and allowed nothing common or unclean to pass through her. And she called the undefiled daughters of the Hebrews, and they led her astray\footnote{This is the reading of most MSS.; but it is difficult to see any sense in it. One MS. reads: They attended on her. Fabricius proposed: They bathed her.}.

\pend\pstart
And when she was a year old, Joachim made a great feast, and invited the priests, and the scribes, and the elders, and all the people of Israel. And Joachim brought the child to the priests; and they blessed her, saying: O God of our fathers, bless this child, and give her an everlasting name to be named in all generations. And all the people said: So be it, so be it, amen. And he brought her to the chief priests; and they blessed her, saying: O God most high, look upon this child, and bless her with the utmost blessing, which shall be for ever.

\pend\pstart
And her mother snatched her up, and took her into the sanctuary of her bed-chamber, and gave her the breast. And Anna made a song to the Lord God, saying: I will sing a song to the Lord my God, for He hath looked upon me, and hath taken away the reproach of mine enemies; and the Lord hath given the fruit of His righteousness, singular in its kind, and richly endowed before Him. Who will tell the sons of Rubim that Anna gives suck? Hear, hear, ye twelve tribes of Israel, that Anna gives suck. And she laid her to rest in the bed-chamber of her sanctuary, and went out and ministered unto them. And when the supper was ended, they went down rejoicing, and glorifying the God of Israel\footnote{Two of the MSS. add: And they gave her the name of Mary, because her name shall not fade forever. This derivation of the name\textemdash from the root \textit{mar}, fade\textemdash is one of a dozen or so.}.

\pend\pstart
\eledsection*{VII}

\pend\pstart
And her months were added to the child. And the child was two years old, and Joachim said: Let us take her up to the temple of the Lord, that we may pay the vow that we have vowed, lest perchance the Lord send to us\footnote{This is taken to mean: Send someone to us to warn us that we have been too long in paying our vow. One MS. reads, lest the Lord depart from us; another, lest the Lord move away from us.}, and our offering be not received. And Anna said: Let us wait for the third year, in order that the child may not seek for father or mother. And Joachim said: So let us wait.

\pend\pstart
And the child was three years old, and Joachim said: Invite the daughters of the Hebrews that are undefiled, and let them take each a lamp, and let them stand with the lamps burning, that the child may not turn back, and her heart be captivated from the temple of the Lord. And they did so until they went up into the temple of the Lord. And the priest received her, and kissed her, and blessed her, saying: The Lord has magnified thy name in all generations. In thee, on the last of the days, the Lord will manifest His redemption to the sons of Israel.

\pend\pstart
And he set her down upon the third step of the altar, and the Lord God sent grace upon her; and she danced with her feet, and all the house of Israel loved her.

\pend\pstart
\eledsection*{VIII}

\pend\pstart
And her parents went down marvelling, and praising the Lord God, because the child had not turned back. And Mary was in the temple of the Lord as if she were a dove that dwelt there, and she received food from the hand of an angel.

\pend\pstart
And when she was twelve\footnote{Or, fourteen. Postel's Latin version has \textit{ten}.} years old there was held a council of the priests, saying: Behold, Mary has reached the age of twelve years in the temple of the Lord. What then shall we do with her, lest perchance she defile the sanctuary of the Lord? And they said to the high priest: Thou standest by the altar of the Lord; go in, and pray concerning her; and whatever the Lord shall manifest unto thee, that also will we do.

\pend\pstart
And the high priest went in, taking the robe\footnote{Ex. xxviii. 28; Sirach xlv. 9; Justin, \textit{Tryph}., xlii.} with the twelve bells into the holy of holies; and he prayed concerning her. And behold an angel of the Lord stood by him, saying unto him: Zacharias, Zacharias, go out and assemble the widowers of the people, and let them bring each his rod; and to whomsoever the Lord shall show a sign, his wife shall she be. And the heralds went out through all the circuit of Judæa, and the trumpet of the Lord sounded, and all ran.

\pend\pstart
\eledsection*{IX}

\pend\pstart
And Joseph, throwing away his axe, went out to meet them; and when they had assembled, they went away to the high priest, taking with them their rods. And he, taking the rods of all of them, entered into the temple, and prayed; and having ended his prayer, he took the rods and came out, and gave them to them: but there was no sign in them, and Joseph took his rod last; and, behold, a dove came out of the rod, and flew upon Joseph's head. And the priest said to Joseph, Thou hast been chosen by lot to take into thy keeping the virgin of the Lord.

\pend\pstart
But Joseph refused, saying: I have children, and I am an old man, and she is a young girl. I am afraid lest I become a laughing-stock to the sons of Israel. And the priest said to Joseph: Fear the Lord thy God, and remember what the Lord did to Dathan, and Abiram, and Korah\footnote{Num. xvi. 31\textendash 33.}; how the earth opened, and they were swallowed up on account of their contradiction. And now fear, O Joseph, lest the same things happen in thy house.

\pend\pstart
And Joseph was afraid, and took her into his keeping. And Joseph said to Mary: Behold, I have received thee from the temple of the Lord; and now I leave thee in my house, and go away to build my buildings, and I shall come to thee. The Lord will protect thee.

\pend\pstart
\eledsection*{X}

\pend\pstart
And there was a council of the priests, saying: Let us make a veil for the temple of the Lord. And the priest said: Call to me the undefiled virgins of the family of David. And the officers went away, and sought, and found seven virgins. And the priest remembered the child Mary, that she was of the family of David, and undefiled before God. And the officers went away and brought her. And they brought them into the temple of the Lord. And the priest said: Choose for me by lot who shall spin the gold, and the white\footnote{Lit., undefiled. It is difficult to say what colour is meant, or if it is a colour at all. The word is once used to mean the sea, but with no reference to colour. It is also the name of a stone of a greenish hue.}, and the fine linen, and the silk, and the blue\footnote{Lit., hyacinth.}, and the scarlet, and the true purple\footnote{Ex. xxv. 4.}. And the true purple and the scarlet fell to the lot of Mary, and she took them, and went away to her house. And at that time Zacharias was dumb, and Samuel was in his place until the time that Zacharias spake. And Mary took the scarlet, and span it.

\pend\pstart
\eledsection*{XI}

\pend\pstart
And she took the pitcher, and went out to fill it with water. And, behold, a voice saying: Hail, thou who hast received grace; the Lord is with thee; blessed art thou among women!\footnote{Luke i. 28.} And she looked round, on the right hand and on the left, to see whence this voice came. And she went away, trembling, to her house, and put down the pitcher; and taking the purple, she sat down on her seat, and drew it out.

\pend\pstart
And, behold, an angel of the Lord stood before her, saying: Fear not, Mary; for thou hast found grace before the Lord of all, and thou shalt conceive, according to His word. And she hearing, reasoned with herself, saying: Shall I conceive by the Lord, the living God? and shall I bring forth as every woman brings forth?

\pend\pstart
And the angel of the Lord said: Not so, Mary; for the power of the Lord shall overshadow thee: wherefore also that holy thing which shall be born of thee shall be called the Son of the Most High. And thou shalt call His name Jesus, for He shall save His people from their sins. And Mary said: Behold, the servant of the Lord before His face: let it be unto me according to thy word.

\pend\pstart
\eledsection*{XII}

\pend\pstart
And she made the purple and the scarlet, and took them to the priest. And the priest blessed her, and said: Mary, the Lord God hath magnified thy name, and thou shalt be blessed in all the generations of the earth.

\pend\pstart
And Mary, with great
joy, went away to Elizabeth her kinswoman,\footnote{Luke i. 39, 40.} and knocked at the door. And when Elizabeth heard her, she threw away the scarlet,\footnote{Other readings are: the wool\textemdash what she had in her hand.} and ran to the door, and opened it; and seeing Mary, she blessed her, and said: Whence is this to me, that the mother of my Lord should come to me? for, behold, that which is in me leaped and blessed thee.\footnote{Luke i. 34, 44.} But Mary had forgotten the mysteries of which the archangel Gabriel had spoken, and gazed up into heaven, and said: Who am I, O Lord, that all the generations of the earth should bless me?\footnote{Luke i. 48.}

\pend\pstart
And she remained three months with Elizabeth; and day by day she grew bigger. And Mary being afraid, went away to her own house, and hid herself from the sons of Israel. And she was sixteen\footnote{Six MSS. have \textit{sixteen}; one, \textit{fourteen}; two, \textit{fifteen}; and one, \textit{seventeen}.} years old when these mysteries happened.

\pend\pstart
\eledsection*{XIII}

\pend


\endnumbering
\end{Rightside}
\end{pairs}
\Columns

{\center
*

* *

{\Large\greekfont
ΓΕΝΕΣΙΣ ΜΑΡΙΑΣ, ΑΠΟΚΑΛΥΨΙΣ ΙΑΚΩΒ.

Εἰρήνη τῷ γράψαντι καὶ τῷ ἀναγινώσκοντι.

}
}

\end{document}
